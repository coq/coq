\achapter{Extraction of programs in Objective Caml and Haskell}
\label{Extraction}
\aauthor{Jean-Christophe Filli�tre and Pierre Letouzey}
\index{Extraction}

\begin{flushleft}
  \em The status of extraction is experimental.
\end{flushleft}
We present here the \Coq\ extraction commands, used to build certified
and relatively efficient functional programs, extracting them from the
proofs of their specifications.  The functional languages available as
output are currently \ocaml{}, \textsc{Haskell} and \textsc{Scheme}.
In the following, ``ML'' will be used (abusively) to refer to any of
the three.

\paragraph{Differences with old versions.}
The current extraction mechanism is new for version 7.0 of {\Coq}.
In particular, the \FW\ toplevel used as an intermediate step between 
\Coq\ and ML has been withdrawn.  It is also not possible 
any more to import ML objects in this \FW\ toplevel.
The current mechanism also differs from
the one in previous versions of \Coq: there is no more
an explicit toplevel for the language (formerly called \textsc{Fml}). 

\asection{Generating ML code}
\comindex{Extraction}
\comindex{Recursive Extraction}
\comindex{Extraction Module}
\comindex{Recursive Extraction Module}

The next two commands are meant to be used for rapid preview of
extraction. They both display extracted term(s) inside \Coq.

\begin{description}
\item {\tt Extraction \qualid.} ~\par
  Extracts one constant or module in the \Coq\ toplevel.

\item {\tt Recursive Extraction  \qualid$_1$ \dots\ \qualid$_n$.} ~\par
  Recursive extraction of all the globals (or modules) \qualid$_1$ \dots\
  \qualid$_n$ and all their dependencies in the \Coq\ toplevel.
\end{description}

%% TODO error messages

All the following commands produce real ML files. User can choose to produce
one monolithic file or one file per \Coq\ library. 

\begin{description}
\item {\tt Extraction "{\em file}"}  
      \qualid$_1$ \dots\ \qualid$_n$. ~\par
  Recursive extraction of all the globals (or modules) \qualid$_1$ \dots\
  \qualid$_n$ and all their dependencies in one monolithic file {\em file}.
  Global and local identifiers are renamed according to the choosen ML
  language to fullfill its syntactic conventions, keeping original
  names as much as possible.
  
\item {\tt Extraction Library} \ident. ~\par 
  Extraction of the whole \Coq\ library {\tt\ident.v} to an ML module
  {\tt\ident.ml}.  In case of name clash, identifiers are here renamed
  using prefixes \verb!coq_!  or \verb!Coq_! to ensure a
  session-independent renaming.

\item {\tt Recursive Extraction Library} \ident. ~\par
  Extraction of the \Coq\ library {\tt\ident.v} and all other modules 
  {\tt\ident.v} depends on. 
\end{description}

The list of globals \qualid$_i$ does not need to be
exhaustive: it is automatically completed into a complete and minimal
environment. 

\asection{Extraction options}

\asubsection{Setting the target language}
\comindex{Extraction Language}

The ability to fix target language is the first and more important
of the extraction options. Default is Ocaml. Besides Haskell and
Scheme, another language called Toplevel is provided. It is a pseudo-Ocaml,
with no renaming on global names: so names are printed as in \Coq.
This third language is available only at the \Coq\ Toplevel.
\begin{description}
\item {\tt Extraction Language Ocaml}.
\item {\tt Extraction Language Haskell}.
\item {\tt Extraction Language Scheme}.
\item {\tt Extraction Language Toplevel}.
\end{description}

\asubsection{Inlining and optimizations}

Since Objective Caml is a strict language, the extracted
code has to be optimized in order to be efficient (for instance, when
using induction principles we do not want to compute all the recursive
calls but only the needed ones). So the extraction mechanism provides
an automatic optimization routine that will be
called each time the user want to generate Ocaml programs. Essentially,
it performs constants inlining and reductions.  Therefore some
constants may not appear in resulting monolithic Ocaml program (a warning is
printed for each such constant). In the case of modular extraction, 
even if some inlining is done, the inlined constant are nevertheless
printed, to ensure session-independent programs. 

Concerning Haskell, such optimizations are less useful because of
lazyness. We still make some optimizations, for example in order to
produce more readable code. 

All these optimizations are controled by the following \Coq\ options: 

\begin{description}

\item \comindex{Set Extraction Optimize}
{\tt Set Extraction Optimize.}

\item \comindex{Unset Extraction Optimize}
{\tt Unset Extraction Optimize.}

Default is Set. This control all optimizations made on the ML terms 
(mostly reduction of dummy beta/iota redexes, but also simplications on 
Cases, etc). Put this option to Unset if you want a ML term as close as 
possible to the Coq term.

\item \comindex{Set Extraction AutoInline}
{\tt Set Extraction AutoInline.} 

\item \comindex{Unset Extraction AutoInline}
{\tt Unset Extraction AutoInline.} 

Default is Set, so by default, the extraction mechanism feels free to 
inline the bodies of some defined constants, according to some heuristics 
like size of bodies, useness of some arguments, etc. Those heuristics are 
not always perfect, you may want to disable this feature, do it by Unset. 

\item \comindex{Extraction Inline}
{\tt Extraction Inline} \qualid$_1$ \dots\ \qualid$_n$. 

\item \comindex{Extraction NoInline}
{\tt Extraction NoInline} \qualid$_1$ \dots\ \qualid$_n$. 

In addition to the automatic inline feature, you can now tell precisely to 
inline some more constants by the {\tt Extraction Inline} command. Conversely, 
you can forbid the automatic inlining of some specific constants by
the {\tt Extraction NoInline} command.
Those two commands enable a precise control of what is inlined and what is not. 

\item \comindex{Print Extraction Inline}
{\tt Print Extraction Inline}. 

Prints the current state of the table recording the custom inlinings 
declared by the two previous commands. 

\item \comindex{Reset Extraction Inline}
{\tt Reset Extraction Inline}. 

Puts the table recording the custom inlinings back to empty. 

\end{description}


\paragraph{Inlining and printing of a constant declaration.}

A user can explicitely asks a constant to be extracted by two means: 
\begin{itemize}
\item by mentioning it on the extraction command line
\item by extracting the whole \Coq\ module of this constant.
\end{itemize}
In both cases, the declaration of this constant will be present in the
produced file. 
But this same constant may or may not be inlined in the following
terms, depending on the automatic/custom inlining mechanism.  


For the constants non-explicitely required but needed for dependancy
reasons, there are two cases: 
\begin{itemize}
\item If an inlining decision is taken, wether automatically or not, 
all occurences of this constant are replaced by its extracted body, and
this constant is not declared in the generated file.
\item If no inlining decision is taken, the constant is normally
  declared in the produced file. 
\end{itemize}

\asubsection{Realizing axioms}\label{extraction:axioms}

Extraction will fail if it encounters an informative
axiom not realized (see Section~\ref{extraction:axioms}). 
A warning will be issued if it encounters an logical axiom, to remind 
user that inconsistant logical axioms may lead to incorrect or 
non-terminating extracted terms. 

It is possible to assume some axioms while developing a proof. Since
these axioms can be any kind of proposition or object or type, they may
perfectly well have some computational content. But a program must be
a closed term, and of course the system cannot guess the program which
realizes an axiom.  Therefore, it is possible to tell the system
what ML term corresponds to a given axiom. 

\comindex{Extract Constant}
\begin{description}
\item{\tt Extract Constant \qualid\ => \str.} ~\par
  Give an ML extraction for the given constant.
  The \str\ may be an identifier or a quoted string.
\item{\tt Extract Inlined Constant \qualid\ => \str.} ~\par
  Same as the previous one, except that the given ML terms will
  be inlined everywhere instead of being declared via a let.
\end{description}

Note that the {\tt Extract Inlined Constant} command is sugar
for an {\tt Extract Constant} followed by a {\tt Extraction Inline}. 
Hence a {\tt Reset Extraction Inline} will have an effect on the
realized and inlined xaxiom. 

Of course, it is the responsability of the user to ensure that the ML
terms given to realize the axioms do have the expected types.  In
fact, the strings containing realizing code are just copied in the
extracted files. The extraction recognize whether the realized axiom
should become a ML type constant or a ML object declaration.

\Example
\begin{coq_example}
Axiom X:Set.
Axiom x:X.
Extract Constant X => "int".
Extract Constant x => "0".
\end{coq_example}   

Notice that in the case of type scheme axiom (i.e. whose type is an
arity, that is a sequence of product finished by a sort), then some type
variables has to be given. The syntax is then: 

\begin{description}
\item{\tt Extract Constant \qualid\ \str$_1$ \ldots \str$_n$ => \str.} ~\par
\end{description}

The number of type variables is checked by the system. 

\Example
\begin{coq_example}
Axiom Y : Set -> Set -> Set.
Extract Constant Y "'a" "'b" => " 'a*'b ".
\end{coq_example}

Realizing an axiom via {\tt Extract Constant} is only useful in the
case of an informative axiom (of sort Type or Set). A logical axiom
have no computational content and hence will not appears in extracted
terms. But a warning is nonetheless issued if extraction encounters a
logical axiom. This warning reminds user that inconsistant logical
axioms may lead to incorrect or non-terminating extracted terms.

If an informative axiom has not been realized before an extraction, a
warning is also issued and the definition of the axiom is filled with
an exception labelled {\tt AXIOM TO BE REALIZED}. The user must then
search these exceptions inside the extracted file and replace them by
real code.

\comindex{Extract Inductive} 

The system also provides a mechanism to specify ML terms for inductive
types and constructors.  For instance, the user may want to use the ML
native boolean type instead of \Coq\ one.  The syntax is the following:

\begin{description}
\item{\tt Extract Inductive \qualid\ => \str\ [ \str\ \dots \str\ ].} ~\par
  Give an ML extraction for the given inductive type. You must specify
  extractions for the type itself (first \str) and all its
  constructors (between square brackets). The ML extraction must be an
  ML recursive datatype.
\end{description}

\Example
Typical examples are the following:
\begin{coq_example}
Extract Inductive unit => "unit" [ "()" ].
Extract Inductive bool => "bool" [ "true" "false" ].
Extract Inductive sumbool => "bool" [ "true" "false" ].
\end{coq_example}

If an inductive constructor or type has arity 2 and the corresponding 
string is enclosed by parenthesis, then the rest of the string is used
as infix constructor or type. 
\begin{coq_example}
Extract Inductive list => "list" [ "[]" "(::)" ].
Extract Inductive prod => "(*)"  [ "(,)" ].
\end{coq_example}


\asection{Differences between \Coq\ and ML type systems}


Due to differences between \Coq\ and ML type systems, 
some extracted programs are not directly typable in ML. 
We now solve this problem (at least in Ocaml) by adding 
when needed some unsafe casting {\tt Obj.magic}, which give
a generic type {\tt 'a} to any term.

For example, Here are two kinds of problem that can occur: 

\begin{itemize}
  \item If some part of the program is {\em very} polymorphic, there
    may be no ML type for it. In that case the extraction to ML works
    all right but the generated code may be refused by the ML
    type-checker. A very well known example is the {\em distr-pair}
    function:
\begin{verbatim}
Definition dp := 
 fun (A B:Set)(x:A)(y:B)(f:forall C:Set, C->C) => (f A x, f B y).
\end{verbatim}

In Ocaml, for instance, the direct extracted term would be:

\begin{verbatim}
let dp x y f = Pair((f () x),(f () y))
\end{verbatim}

and would have type:
\begin{verbatim}
dp : 'a -> 'a -> (unit -> 'a -> 'b) -> ('b,'b) prod
\end{verbatim}

which is not its original type, but a restriction.

We now produce the following correct version:
\begin{verbatim}
let dp x y f = Pair ((Obj.magic f () x), (Obj.magic f () y))
\end{verbatim}

  \item Some definitions of \Coq\ may have no counterpart in ML. This
    happens when there is a quantification over types inside the type
    of a constructor; for example:
\begin{verbatim}
Inductive anything : Set := dummy : forall A:Set, A -> anything.
\end{verbatim}

which corresponds to the definition of an ML dynamic type.
In Ocaml, we must cast any argument of the constructor dummy.

\end{itemize}

Even with those unsafe castings, you should never get error like
``segmentation fault''. In fact even if your program may seem
ill-typed to the Ocaml type-checker, it can't go wrong: it comes 
from a Coq well-typed terms, so for example inductives will always 
have the correct number of arguments, etc. 

More details about the correctness of the extracted programs can be 
found in \cite{Let02}.

We have to say, though, that in most ``realistic'' programs, these
problems do not occur. For example all the programs of Coq library are
accepted by Caml type-checker without any {\tt Obj.magic} (see examples below).



\asection{Some examples}

We present here two examples of extractions, taken from the 
\Coq\ Standard Library. We choose \ocaml\ as target language, 
but all can be done in the other dialects with slight modifications.
We then indicate where to find other examples and tests of Extraction.

\asubsection{A detailed example: Euclidean division}

The file {\tt Euclid} contains the proof of Euclidean division
(theorem {\tt eucl\_dev}). The natural numbers defined in the example
files are unary integers defined by two constructors $O$ and $S$:
\begin{coq_example*}
Inductive nat : Set :=
  | O : nat
  | S : nat -> nat.
\end{coq_example*}

This module contains a theorem {\tt eucl\_dev}, and its extracted term
is of type 
\begin{verbatim}
forall b:nat, b > 0 -> forall a:nat, diveucl a b
\end{verbatim}
where {\tt diveucl} is a type for the pair of the quotient and the modulo.
We can now extract this program to \ocaml:

\begin{coq_eval}
Reset Initial.
\end{coq_eval}
\begin{coq_example}
Require Import Euclid.
Extraction Inline Wf_nat.gt_wf_rec Wf_nat.lt_wf_rec.
Recursive Extraction  eucl_dev.
\end{coq_example}

The inlining of {\tt gt\_wf\_rec} and {\tt lt\_wf\_rec} is not
mandatory. It only enhances readability of extracted code. 
You can then copy-paste the output to a file {\tt euclid.ml} or let 
\Coq\ do it for you with the following command: 

\begin{coq_example}
Extraction "euclid" eucl_dev.
\end{coq_example}

Let us play the resulting program:

\begin{verbatim}
# #use "euclid.ml";;
type sumbool = Left | Right
type nat = O | S of nat
type diveucl = Divex of nat * nat
val minus : nat -> nat -> nat = <fun>
val le_lt_dec : nat -> nat -> sumbool = <fun>
val le_gt_dec : nat -> nat -> sumbool = <fun>
val eucl_dev : nat -> nat -> diveucl = <fun>
# eucl_dev (S (S O)) (S (S (S (S (S O)))));;
- : diveucl = Divex (S (S O), S O)
\end{verbatim}
It is easier to test on \ocaml\ integers:
\begin{verbatim}
# let rec i2n = function 0 -> O | n -> S (i2n (n-1));;
val i2n : int -> nat = <fun>
# let rec n2i = function O -> 0 | S p -> 1+(n2i p);;
val n2i : nat -> int = <fun>
# let div a b = 
     let Divex (q,r) = eucl_dev (i2n b) (i2n a) in (n2i q, n2i r);;
div : int -> int -> int * int = <fun>
# div 173 15;;
- : int * int = 11, 8
\end{verbatim}

\asubsection{Another detailed example: Heapsort}

The file {\tt Heap.v}
contains the proof of an efficient list sorting algorithm described by
Bjerner. Is is an adaptation of the well-known {\em heapsort}
algorithm to functional languages. The main function is {\tt
treesort}, whose type is shown below: 


\begin{coq_eval}
Reset Initial.
Require Import Relation_Definitions.
Require Import List.
Require Import Sorting.
Require Import Permutation.
\end{coq_eval}
\begin{coq_example}
Require Import Heap.
Check treesort.
\end{coq_example}

Let's now extract this function: 

\begin{coq_example}
Extraction Inline sort_rec is_heap_rec.
Extraction NoInline list_to_heap.
Extraction "heapsort" treesort.
\end{coq_example}

One more time, the {\tt Extraction Inline} and {\tt NoInline}
directives are cosmetic. Without it, everything goes right, 
but the output is less readable.
Here is the produced file {\tt heapsort.ml}: 

\begin{verbatim}
type nat =
  | O
  | S of nat

type 'a sig2 =
  'a
  (* singleton inductive, whose constructor was exist2 *)
  
type sumbool =
  | Left
  | Right

type 'a list =
  | Nil
  | Cons of 'a * 'a list

type 'a multiset =
  'a -> nat
  (* singleton inductive, whose constructor was Bag *)
  
type 'a merge_lem =
  'a list
  (* singleton inductive, whose constructor was merge_exist *)
  
(** val merge : ('a1 -> 'a1 -> sumbool) -> ('a1 -> 'a1 -> sumbool) ->
                'a1 list -> 'a1 list -> 'a1 merge_lem **)

let rec merge leA_dec eqA_dec l1 l2 =
  match l1 with
    | Nil -> l2
    | Cons (a, l) ->
        let rec f = function
          | Nil -> Cons (a, l)
          | Cons (a0, l3) ->
              (match leA_dec a a0 with
                 | Left -> Cons (a,
                     (merge leA_dec eqA_dec l (Cons (a0, l3))))
                 | Right -> Cons (a0, (f l3)))
        in f l2

type 'a tree =
  | Tree_Leaf
  | Tree_Node of 'a * 'a tree * 'a tree

type 'a insert_spec =
  'a tree
  (* singleton inductive, whose constructor was insert_exist *)
  
(** val insert : ('a1 -> 'a1 -> sumbool) -> ('a1 -> 'a1 -> sumbool) ->
                 'a1 tree -> 'a1 -> 'a1 insert_spec **)

let rec insert leA_dec eqA_dec t a =
  match t with
    | Tree_Leaf -> Tree_Node (a, Tree_Leaf, Tree_Leaf)
    | Tree_Node (a0, t0, t1) ->
        let h3 = fun x -> insert leA_dec eqA_dec t0 x in
        (match leA_dec a0 a with
           | Left -> Tree_Node (a0, t1, (h3 a))
           | Right -> Tree_Node (a, t1, (h3 a0)))

type 'a build_heap =
  'a tree
  (* singleton inductive, whose constructor was heap_exist *)
  
(** val list_to_heap : ('a1 -> 'a1 -> sumbool) -> ('a1 -> 'a1 ->
                       sumbool) -> 'a1 list -> 'a1 build_heap **)

let rec list_to_heap leA_dec eqA_dec = function
  | Nil -> Tree_Leaf
  | Cons (a, l0) ->
      insert leA_dec eqA_dec (list_to_heap leA_dec eqA_dec l0) a

type 'a flat_spec =
  'a list
  (* singleton inductive, whose constructor was flat_exist *)
  
(** val heap_to_list : ('a1 -> 'a1 -> sumbool) -> ('a1 -> 'a1 ->
                       sumbool) -> 'a1 tree -> 'a1 flat_spec **)

let rec heap_to_list leA_dec eqA_dec = function
  | Tree_Leaf -> Nil
  | Tree_Node (a, t0, t1) -> Cons (a,
      (merge leA_dec eqA_dec (heap_to_list leA_dec eqA_dec t0)
        (heap_to_list leA_dec eqA_dec t1)))

(** val treesort : ('a1 -> 'a1 -> sumbool) -> ('a1 -> 'a1 -> sumbool)
                   -> 'a1 list -> 'a1 list sig2 **)

let treesort leA_dec eqA_dec l =
  heap_to_list leA_dec eqA_dec (list_to_heap leA_dec eqA_dec l)

\end{verbatim}

Let's test it: 
% Format.set_margin 72;;
\begin{verbatim}
# #use "heapsort.ml";;
type sumbool = Left | Right
type nat = O | S of nat
type 'a tree = Tree_Leaf | Tree_Node of 'a * 'a tree * 'a tree
type 'a list = Nil | Cons of 'a * 'a list
val merge : 
  ('a -> 'a -> sumbool) -> 'b -> 'a list -> 'a list -> 'a list = <fun>
val heap_to_list : 
  ('a -> 'a -> sumbool) -> 'b -> 'a tree -> 'a list = <fun>
val insert : 
  ('a -> 'a -> sumbool) -> 'b -> 'a tree -> 'a -> 'a tree = <fun>
val list_to_heap : 
  ('a -> 'a -> sumbool) -> 'b -> 'a list -> 'a tree = <fun>
val treesort : 
  ('a -> 'a -> sumbool) -> 'b -> 'a list -> 'a list = <fun>
\end{verbatim}

One can remark that the argument of {\tt treesort} corresponding to 
{\tt eqAdec} is never used in the informative part of the terms, 
only in the logical parts. So the extracted {\tt treesort} never use
it, hence this {\tt 'b} argument. We will use {\tt ()} for this
argument. Only remains the {\tt leAdec}
argument (of type {\tt 'a -> 'a -> sumbool}) to really provide.

\begin{verbatim}
# let leAdec x y = if x <= y then Left else Right;;
val leAdec : 'a -> 'a -> sumbool = <fun>
# let rec listn = function 0 -> Nil
                         | n -> Cons(Random.int 10000,listn (n-1));;
val listn : int -> int list = <fun>
# treesort leAdec () (listn 9);;
- : int list = Cons (160, Cons (883, Cons (1874, Cons (3275, Cons 
  (5392, Cons (7320, Cons (8512, Cons (9632, Cons (9876, Nil)))))))))
\end{verbatim}

Some tests on longer lists (10000 elements) show that the program is
quite efficient for Caml code.


\asubsection{The Standard Library} 

As a test, we propose an automatic extraction of the 
Standard Library of \Coq. In particular, we will find back the
two previous examples, {\tt Euclid} and {\tt Heapsort}. 
Go to directory {\tt contrib/extraction/test} of the sources of \Coq,
and run commands: 
\begin{verbatim}
make tree; make
\end{verbatim}
This will extract all Standard Library files and compile them. 
It is done via many {\tt Extraction Module}, with some customization
(see subdirectory {\tt custom}).

%The result of this extraction of the Standard Library can be browsed
%at URL 
%\begin{flushleft}
%\url{http://www.lri.fr/~letouzey/extraction}.
%\end{flushleft}
                                
%Reals theory is normally not extracted, since it is an axiomatic 
%development. We propose nonetheless a dummy realization of those
%axioms, to test, run: \\
%
%\mbox{\tt make reals}\\

This test works also with Haskell. In the same directory, run:
\begin{verbatim}
make tree; make -f Makefile.haskell
\end{verbatim}
The haskell compiler currently used is {\tt hbc}.  Any other should
also work, just adapt the {\tt Makefile.haskell}. In particular {\tt
 ghc} is known to work.

\asubsection{Extraction's horror museum}

Some pathological examples of extraction are grouped in the file
\begin{verbatim}
contrib/extraction/test_extraction.v
\end{verbatim}
of the sources of \Coq.

\asubsection{Users' Contributions}

 Several of the \Coq\ Users' Contributions use extraction to produce 
 certified programs. In particular the following ones have an automatic 
 extraction test (just run {\tt make} in those directories): 

 \begin{itemize}
 \item Bordeaux/Additions
 \item Bordeaux/EXCEPTIONS
 \item Bordeaux/SearchTrees
 \item Dyade/BDDS
 \item Lannion
 \item Lyon/CIRCUITS
 \item Lyon/FIRING-SQUAD
 \item Marseille/CIRCUITS
 \item Muenchen/Higman
 \item Nancy/FOUnify
 \item Rocq/ARITH/Chinese
 \item Rocq/COC
 \item Rocq/GRAPHS
 \item Rocq/HIGMAN
 \item Sophia-Antipolis/Stalmarck
 \item Suresnes/BDD
 \end{itemize}

 Lannion, Rocq/HIGMAN and Lyon/CIRCUITS are a bit particular. They are 
 the only examples of developments where {\tt Obj.magic} are needed.
 This is probably due to an heavy use of impredicativity.
 After compilation those two examples run nonetheless,
 thanks to the correction of the extraction~\cite{Let02}. 

% $Id$ 

%%% Local Variables: 
%%% mode: latex
%%% TeX-master: "Reference-Manual"
%%% End: 
