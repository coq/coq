\documentclass[11pt]{report}

\usepackage[utf8x]{inputenc}
\usepackage[T1]{fontenc}
\usepackage{fullpage}
\usepackage{amsfonts}
\usepackage[color]{../../coqdoc}

\input{../common/version}
%%%%%%%%%%%%%%%%%%%%%%%%%%%%%%%%
% File title.tex
% Page formatting commands
% Macro \coverpage
%%%%%%%%%%%%%%%%%%%%%%%%%%%%%%%%

%\setlength{\marginparwidth}{0pt}
%\setlength{\oddsidemargin}{0pt}
%\setlength{\evensidemargin}{0pt}
%\setlength{\marginparsep}{0pt}
%\setlength{\topmargin}{0pt}
%\setlength{\textwidth}{16.9cm}
%\setlength{\textheight}{22cm}
%\usepackage{fullpage}

%\newcommand{\printingdate}{\today}
%\newcommand{\isdraft}{\Large\bf\today\\[20pt]}
%\newcommand{\isdraft}{\vspace{20pt}}

\newcommand{\coverpage}[3]{
\thispagestyle{empty}
\begin{center}
\bfseries % for the rest of this page, until \end{center}
\Huge
The Coq Proof Assistant\\[12pt]
#1\\[20pt]
\Large\today\\[20pt]
Version \coqversion\footnote[1]{This research was partly supported by IST working group ``Types''}

\vspace{0pt plus .5fill}
#2
\par\vfill
$\pi r^2$ Project (formerly LogiCal, then TypiCal)

\vspace*{15pt}
\end{center}
\newpage

\thispagestyle{empty}
\hbox{}\vfill % without \hbox \vfill does not work at the top of the page
\begin{flushleft}
%BEGIN LATEX
V\coqversion, \today
\par\vspace{20pt}
%END LATEX
\copyright INRIA 1999-2004 ({\Coq} versions 7.x)

\copyright INRIA 2004-2017 ({\Coq} versions 8.x)

#3
\end{flushleft}
} % end of \coverpage definition


% \newcommand{\shorttitle}[1]{
% \begin{center}
% \begin{huge}
% \begin{bf}
% The Coq Proof Assistant\\
% \vspace{10pt}
%     #1\\
% \end{bf}
% \end{huge}
% \end{center}
% \vspace{5pt}
% }

% Local Variables: 
% mode: LaTeX
% TeX-master: ""
% End: 


%%%%%%%%%%%%%%%%%%%%%%%%%%%%%%%%%%%%%%%%%%
% MACROS FOR THE REFERENCE MANUAL OF COQ %
%%%%%%%%%%%%%%%%%%%%%%%%%%%%%%%%%%%%%%%%%%

% For commentaries (define \com as {} for the release manual)
%\newcommand{\com}[1]{{\it(* #1 *)}}
%\newcommand{\com}[1]{}

%%OPTIONS for HACHA
%\renewcommand{\cuttingunit}{section}


%BEGIN LATEX
\newenvironment{centerframe}%
{\bgroup
\dimen0=\textwidth
\advance\dimen0 by -2\fboxrule
\advance\dimen0 by -2\fboxsep
\setbox0=\hbox\bgroup
\begin{minipage}{\dimen0}%
\begin{center}}%
{\end{center}%
\end{minipage}\egroup
\centerline{\fbox{\box0}}\egroup
}
%END LATEX
%HEVEA \newenvironment{centerframe}{\begin{center}}{\end{center}}

%HEVEA \renewcommand{\vec}[1]{\mathbf{#1}}
%\renewcommand{\ominus}{-} % Hevea does a good job translating these commands
%\renewcommand{\oplus}{+}
%\renewcommand{\otimes}{\times}
%\newcommand{\land}{\wedge}
%\newcommand{\lor}{\vee}
%HEVEA \renewcommand{\k}[1]{#1} % \k{a} is supposed to produce a with a little stroke
%HEVEA \newcommand{\phantom}[1]{\qquad}

%%%%%%%%%%%%%%%%%%%%%%%
% Formatting commands %
%%%%%%%%%%%%%%%%%%%%%%%

\newcommand{\ErrMsg}{\medskip \noindent {\bf Error message: }}
\newcommand{\ErrMsgx}{\medskip \noindent {\bf Error messages: }}
\newcommand{\variant}{\medskip \noindent {\bf Variant: }}
\newcommand{\variants}{\medskip \noindent {\bf Variants: }}
\newcommand{\SeeAlso}{\medskip \noindent {\bf See also: }}
\newcommand{\Rem}{\medskip \noindent {\bf Remark: }}
\newcommand{\Rems}{\medskip \noindent {\bf Remarks: }}
\newcommand{\Example}{\medskip \noindent {\bf Example: }}
\newcommand{\examples}{\medskip \noindent {\bf Examples: }}
\newcommand{\Warning}{\medskip \noindent {\bf Warning: }}
\newcommand{\Warns}{\medskip \noindent {\bf Warnings: }}
\newcounter{ex}
\newcommand{\firstexample}{\setcounter{ex}{1}}
\newcommand{\example}[1]{
\medskip \noindent \textbf{Example \arabic{ex}: }\textit{#1}
\addtocounter{ex}{1}}

\newenvironment{Variant}{\variant\begin{enumerate}}{\end{enumerate}}
\newenvironment{Variants}{\variants\begin{enumerate}}{\end{enumerate}}
\newenvironment{ErrMsgs}{\ErrMsgx\begin{enumerate}}{\end{enumerate}}
\newenvironment{Remarks}{\Rems\begin{enumerate}}{\end{enumerate}}
\newenvironment{Warnings}{\Warns\begin{enumerate}}{\end{enumerate}}
\newenvironment{Examples}{\medskip\noindent{\bf Examples:}
\begin{enumerate}}{\end{enumerate}}

%\newcommand{\bd}{\noindent\bf}
%\newcommand{\sbd}{\vspace{8pt}\noindent\bf}
%\newcommand{\sdoll}[1]{\begin{small}$ #1~ $\end{small}}
%\newcommand{\sdollnb}[1]{\begin{small}$ #1 $\end{small}}
\newcommand{\kw}[1]{\textsf{#1}}
%\newcommand{\spec}[1]{\{\,#1\,\}}

% Building regular expressions
\newcommand{\zeroone}[1]{\mbox{\sl [}#1\mbox{\sl ]}}
%\newcommand{\zeroonemany}[1]{$\{$#1$\}$*}
%\newcommand{\onemany}[1]{$\{$#1$\}$+}
\newcommand{\nelistnosep}[1]{{#1} \mbox{\dots} {#1}}
\newcommand{\nelist}[2]{{#1} {\tt #2} \mbox{\dots} {\tt #2} {#1}}
\newcommand{\sequence}[2]{{\sl [}{#1} {\tt #2} \mbox{\dots} {\tt #2} {#1}{\sl ]}}
\newcommand{\nelistwithoutblank}[2]{#1{\tt #2}\mbox{\dots}{\tt #2}#1}
\newcommand{\sequencewithoutblank}[2]{$[$#1{\tt #2}\mbox{\dots}{\tt #2}#1$]$}

% Used for RefMan-gal
%\newcommand{\ml}[1]{\hbox{\tt{#1}}}
%\newcommand{\op}{\,|\,}

%%%%%%%%%%%%%%%%%%%%%%%%
% Trademarks and so on %
%%%%%%%%%%%%%%%%%%%%%%%%

\newcommand{\Coq}{\textsc{Coq}}
\newcommand{\gallina}{\textsc{Gallina}}
\newcommand{\Gallina}{\textsc{Gallina}}
\newcommand{\CoqIDE}{\textsc{CoqIDE}}
\newcommand{\ocaml}{\textsc{Objective Caml}}
\newcommand{\camlpppp}{\textsc{Camlp4}}
\newcommand{\emacs}{\textsc{GNU Emacs}}
\newcommand{\ProofGeneral}{\textsc{Proof General}}
\newcommand{\CIC}{\textsc{Cic}}
\newcommand{\iCIC}{\textsc{Cic}}
\newcommand{\FW}{\ensuremath{F_{\omega}}}
\newcommand{\Program}{\textsc{Program}}
\newcommand{\Russell}{\textsc{Russell}}
\newcommand{\PVS}{\textsc{PVS}}
%\newcommand{\bn}{{\sf BNF}}

%%%%%%%%%%%%%%%%%%%
% Name of tactics %
%%%%%%%%%%%%%%%%%%%

%\newcommand{\Natural}{\mbox{\tt Natural}}

%%%%%%%%%%%%%%%%%
% \rm\sl series %
%%%%%%%%%%%%%%%%%

\newcommand{\nterm}[1]{\textrm{\textsl{#1}}}

\newcommand{\qstring}{\nterm{string}}

%% New syntax specific entries
\newcommand{\annotation}{\nterm{annotation}}
\newcommand{\assums}{\nterm{assums}} % vernac
\newcommand{\simpleassums}{\nterm{simple\_assums}} % assumptions
\newcommand{\binder}{\nterm{binder}}
\newcommand{\binders}{\nterm{binders}}
\newcommand{\caseitems}{\nterm{match\_items}}
\newcommand{\caseitem}{\nterm{match\_item}}
\newcommand{\eqn}{\nterm{equation}}
\newcommand{\ifitem}{\nterm{dep\_ret\_type}}
\newcommand{\hyplocation}{\nterm{hyp\_location}}
\newcommand{\convclause}{\nterm{conversion\_clause}}
\newcommand{\occclause}{\nterm{occurrence\_clause}}
\newcommand{\occgoalset}{\nterm{goal\_occurrences}}
\newcommand{\atoccurrences}{\nterm{at\_occurrences}}
\newcommand{\occlist}{\nterm{occurrences}}
\newcommand{\params}{\nterm{params}} % vernac
\newcommand{\returntype}{\nterm{return\_type}}
\newcommand{\idparams}{\nterm{ident\_with\_params}}
\newcommand{\statkwd}{\nterm{assertion\_keyword}} % vernac
\newcommand{\termarg}{\nterm{arg}}
\newcommand{\hintdef}{\nterm{hint\_definition}}

\newcommand{\typecstr}{\zeroone{{\tt :}~{\term}}}
\newcommand{\typecstrwithoutblank}{\zeroone{{\tt :}{\term}}}


\newcommand{\Fwterm}{\nterm{Fwterm}}
\newcommand{\Index}{\nterm{index}}
\newcommand{\abbrev}{\nterm{abbreviation}}
\newcommand{\atomictac}{\nterm{atomic\_tactic}}
\newcommand{\bindinglist}{\nterm{bindings\_list}}
\newcommand{\cast}{\nterm{cast}}
\newcommand{\cofixpointbodies}{\nterm{cofix\_bodies}}
\newcommand{\cofixpointbody}{\nterm{cofix\_body}}
\newcommand{\commandtac}{\nterm{tactic\_invocation}}
\newcommand{\constructor}{\nterm{constructor}}
\newcommand{\convtactic}{\nterm{conv\_tactic}}
\newcommand{\assumptionkeyword}{\nterm{assumption\_keyword}}
\newcommand{\assumption}{\nterm{assumption}}
\newcommand{\definition}{\nterm{definition}}
\newcommand{\digit}{\nterm{digit}}
\newcommand{\exteqn}{\nterm{ext\_eqn}}
\newcommand{\field}{\nterm{field}}
\newcommand{\firstletter}{\nterm{first\_letter}}
\newcommand{\fixpg}{\nterm{fix\_pgm}}
\newcommand{\fixpointbodies}{\nterm{fix\_bodies}}
\newcommand{\fixpointbody}{\nterm{fix\_body}}
\newcommand{\fixpoint}{\nterm{fixpoint}}
\newcommand{\flag}{\nterm{flag}}
\newcommand{\form}{\nterm{form}}
\newcommand{\entry}{\nterm{entry}} 
\newcommand{\proditem}{\nterm{prod\_item}}
\newcommand{\taclevel}{\nterm{tactic\_level}}
\newcommand{\tacargtype}{\nterm{tactic\_argument\_type}} 
\newcommand{\scope}{\nterm{scope}} 
\newcommand{\delimkey}{\nterm{key}} 
\newcommand{\optscope}{\nterm{opt\_scope}} 
\newcommand{\declnotation}{\nterm{decl\_notation}} 
\newcommand{\symbolentry}{\nterm{symbol}}
\newcommand{\modifiers}{\nterm{modifiers}}
\newcommand{\localdef}{\nterm{local\_def}}
\newcommand{\localdecls}{\nterm{local\_decls}}
\newcommand{\ident}{\nterm{ident}}
\newcommand{\accessident}{\nterm{access\_ident}}
\newcommand{\possiblybracketedident}{\nterm{possibly\_bracketed\_ident}}
\newcommand{\inductivebody}{\nterm{ind\_body}}
\newcommand{\inductive}{\nterm{inductive}}
\newcommand{\naturalnumber}{\nterm{natural}}
\newcommand{\integer}{\nterm{integer}}
\newcommand{\multpattern}{\nterm{mult\_pattern}}
\newcommand{\mutualcoinductive}{\nterm{mutual\_coinductive}}
\newcommand{\mutualinductive}{\nterm{mutual\_inductive}}
\newcommand{\nestedpattern}{\nterm{nested\_pattern}}
\newcommand{\name}{\nterm{name}}
\newcommand{\num}{\nterm{num}}
\newcommand{\coqindex}{\nterm{index}}
\newcommand{\pattern}{\nterm{pattern}} % pattern for pattern-matching
\newcommand{\orpattern}{\nterm{or\_pattern}}
\newcommand{\intropattern}{\nterm{intro\_pattern}}
\newcommand{\intropatternlist}{\nterm{intro\_pattern\_list}}
\newcommand{\disjconjintropattern}{\nterm{disj\_conj\_intro\_pattern}}
\newcommand{\namingintropattern}{\nterm{naming\_intro\_pattern}}
\newcommand{\termpattern}{\nterm{term\_pattern}} % term with holes
\newcommand{\pat}{\nterm{pat}}
\newcommand{\pgs}{\nterm{pgms}}
\newcommand{\pg}{\nterm{pgm}}
\newcommand{\abullet}{\nterm{bullet}}
%BEGIN LATEX
\newcommand{\proof}{\nterm{proof}}
%END LATEX
%HEVEA \renewcommand{\proof}{\nterm{proof}}
\newcommand{\record}{\nterm{record}}
\newcommand{\recordkw}{\nterm{record\_keyword}}
\newcommand{\rewrule}{\nterm{rewriting\_rule}}
\newcommand{\sentence}{\nterm{sentence}}
\newcommand{\simplepattern}{\nterm{simple\_pattern}}
\newcommand{\sort}{\nterm{sort}}
\newcommand{\specif}{\nterm{specif}}
\newcommand{\assertion}{\nterm{assertion}}
\newcommand{\str}{\nterm{string}}
\newcommand{\subsequentletter}{\nterm{subsequent\_letter}}
\newcommand{\switch}{\nterm{switch}}
\newcommand{\messagetoken}{\nterm{message\_token}}
\newcommand{\tac}{\nterm{tactic}}
\newcommand{\terms}{\nterm{terms}}
\newcommand{\term}{\nterm{term}}
\newcommand{\module}{\nterm{module}}
\newcommand{\modexpr}{\nterm{module\_expression}}
\newcommand{\modtype}{\nterm{module\_type}}
\newcommand{\onemodbinding}{\nterm{module\_binding}}
\newcommand{\modbindings}{\nterm{module\_bindings}}
\newcommand{\qualid}{\nterm{qualid}}
\newcommand{\qualidorstring}{\nterm{qualid\_or\_string}}
\newcommand{\class}{\nterm{class}}
\newcommand{\dirpath}{\nterm{dirpath}}
\newcommand{\typedidents}{\nterm{typed\_idents}}
\newcommand{\type}{\nterm{type}}
\newcommand{\vref}{\nterm{ref}}
\newcommand{\zarithformula}{\nterm{zarith\_formula}}
\newcommand{\zarith}{\nterm{zarith}}
\newcommand{\ltac}{\mbox{${\cal L}_{tac}$}}

%%%%%%%%%%%%%%%%%%%%%%%%%%%%%%%%%%%%%%%%%%%%%%%%%%%%%%%
% \mbox{\sf } series for roman text in maths formulas %
%%%%%%%%%%%%%%%%%%%%%%%%%%%%%%%%%%%%%%%%%%%%%%%%%%%%%%%

\newcommand{\alors}{\mbox{\textsf{then}}}
\newcommand{\alter}{\mbox{\textsf{alter}}}
\newcommand{\bool}{\mbox{\textsf{bool}}}
\newcommand{\conc}{\mbox{\textsf{conc}}}
\newcommand{\cons}{\mbox{\textsf{cons}}}
\newcommand{\consf}{\mbox{\textsf{consf}}}
\newcommand{\emptyf}{\mbox{\textsf{emptyf}}}
\newcommand{\EqSt}{\mbox{\textsf{EqSt}}}
\newcommand{\false}{\mbox{\textsf{false}}}
\newcommand{\filter}{\mbox{\textsf{filter}}}
\newcommand{\forest}{\mbox{\textsf{forest}}}
\newcommand{\from}{\mbox{\textsf{from}}}
\newcommand{\hd}{\mbox{\textsf{hd}}}
\newcommand{\haslength}{\mbox{\textsf{has\_length}}}
\newcommand{\length}{\mbox{\textsf{length}}}
\newcommand{\haslengthA}{\mbox {\textsf{has\_length~A}}}
\newcommand{\List}{\mbox{\textsf{list}}}
\newcommand{\ListA}{\mbox{\textsf{list}}~\ensuremath{A}}
\newcommand{\nilhl}{\mbox{\textsf{nil\_hl}}}
\newcommand{\conshl}{\mbox{\textsf{cons\_hl}}}
\newcommand{\nat}{\mbox{\textsf{nat}}}
\newcommand{\nO}{\mbox{\textsf{O}}}
\newcommand{\nS}{\mbox{\textsf{S}}}
\newcommand{\node}{\mbox{\textsf{node}}}
\newcommand{\Nil}{\mbox{\textsf{nil}}}
\newcommand{\Prop}{\mbox{\textsf{Prop}}}
\newcommand{\Set}{\mbox{\textsf{Set}}}
\newcommand{\si}{\mbox{\textsf{if}}}
\newcommand{\sinon}{\mbox{\textsf{else}}}
\newcommand{\Str}{\mbox{\textsf{Stream}}}
\newcommand{\tl}{\mbox{\textsf{tl}}}
\newcommand{\tree}{\mbox{\textsf{tree}}}
\newcommand{\true}{\mbox{\textsf{true}}}
\newcommand{\Type}{\mbox{\textsf{Type}}}
\newcommand{\unfold}{\mbox{\textsf{unfold}}}
\newcommand{\zeros}{\mbox{\textsf{zeros}}}
\newcommand{\even}{\mbox{\textsf{even}}}
\newcommand{\odd}{\mbox{\textsf{odd}}}
\newcommand{\evenO}{\mbox{\textsf{even\_O}}}
\newcommand{\evenS}{\mbox{\textsf{even\_S}}}
\newcommand{\oddS}{\mbox{\textsf{odd\_S}}}
\newcommand{\Prod}{\mbox{\textsf{prod}}}
\newcommand{\Pair}{\mbox{\textsf{pair}}}

%%%%%%%%%
% Misc. %
%%%%%%%%%
\newcommand{\T}{\texttt{T}}
\newcommand{\U}{\texttt{U}}
\newcommand{\real}{\textsf{Real}}
\newcommand{\Data}{\textit{Data}}
\newcommand{\In} {{\textbf{in }}}
\newcommand{\AND} {{\textbf{and}}}
\newcommand{\If}{{\textbf{if }}}
\newcommand{\Else}{{\textbf{else }}}
\newcommand{\Then} {{\textbf{then }}}
%\newcommand{\Let}{{\textbf{let }}} % looks like this is never used
\newcommand{\Where}{{\textbf{where rec }}}
\newcommand{\Function}{{\textbf{function }}}
\newcommand{\Rec}{{\textbf{rec }}}
%\newcommand{\cn}{\centering}
\newcommand{\nth}{\mbox{$^{\mbox{\scriptsize th}}$}}

%%%%%%%%%%%%%%%%%%%%%%%%%%%%%
% Math commands and symbols %
%%%%%%%%%%%%%%%%%%%%%%%%%%%%%

\newcommand{\la}{\leftarrow}
\newcommand{\ra}{\rightarrow}
\newcommand{\Ra}{\Rightarrow}
\newcommand{\rt}{\Rightarrow}
\newcommand{\lla}{\longleftarrow}
\newcommand{\lra}{\longrightarrow}
\newcommand{\Llra}{\Longleftrightarrow}
\newcommand{\mt}{\mapsto}
\newcommand{\ov}{\overrightarrow}
\newcommand{\wh}{\widehat}
\newcommand{\up}{\uparrow}
\newcommand{\dw}{\downarrow}
\newcommand{\nr}{\nearrow}
\newcommand{\se}{\searrow}
\newcommand{\sw}{\swarrow}
\newcommand{\nw}{\nwarrow}
\newcommand{\mto}{.\;}

\newcommand{\vm}[1]{\vspace{#1em}}
\newcommand{\vx}[1]{\vspace{#1ex}}
\newcommand{\hm}[1]{\hspace{#1em}}
\newcommand{\hx}[1]{\hspace{#1ex}}
\newcommand{\sm}{\mbox{ }}
\newcommand{\mx}{\mbox}

%\newcommand{\nq}{\neq}
%\newcommand{\eq}{\equiv}
\newcommand{\fa}{\forall}
%\newcommand{\ex}{\exists}
\newcommand{\impl}{\rightarrow}
%\newcommand{\Or}{\vee}
%\newcommand{\And}{\wedge}
\newcommand{\ms}{\models}
\newcommand{\bw}{\bigwedge}
\newcommand{\ts}{\times}
\newcommand{\cc}{\circ}
%\newcommand{\es}{\emptyset}
%\newcommand{\bs}{\backslash}
\newcommand{\vd}{\vdash}
%\newcommand{\lan}{{\langle }}
%\newcommand{\ran}{{\rangle }}

%\newcommand{\al}{\alpha}
\newcommand{\bt}{\beta}
%\newcommand{\io}{\iota}
\newcommand{\lb}{\lambda}
%\newcommand{\sg}{\sigma}
%\newcommand{\sa}{\Sigma}
%\newcommand{\om}{\Omega}
%\newcommand{\tu}{\tau}

%%%%%%%%%%%%%%%%%%%%%%%%%
% Custom maths commands %
%%%%%%%%%%%%%%%%%%%%%%%%%

\newcommand{\sumbool}[2]{\{#1\}+\{#2\}}
\newcommand{\myifthenelse}[3]{\kw{if} ~ #1 ~\kw{then} ~ #2 ~ \kw{else} ~ #3}
\newcommand{\fun}[2]{\item[]{\tt {#1}}. \quad\\ #2}
\newcommand{\WF}[2]{\ensuremath{{\cal W\!F}(#1)[#2]}}
\newcommand{\WFTWOLINES}[2]{\ensuremath{{\cal W\!F}\begin{array}{l}(#1)\\\mbox{}[{#2}]\end{array}}}
\newcommand{\WFE}[1]{\WF{E}{#1}}
\newcommand{\WT}[4]{\ensuremath{#1[#2] \vdash #3 : #4}}
\newcommand{\WTE}[3]{\WT{E}{#1}{#2}{#3}}
\newcommand{\WTEG}[2]{\WTE{\Gamma}{#1}{#2}}

\newcommand{\WTM}[3]{\WT{#1}{}{#2}{#3}}
\newcommand{\WFT}[2]{\ensuremath{#1[] \vdash {\cal W\!F}(#2)}}
\newcommand{\WS}[3]{\ensuremath{#1[] \vdash #2 <: #3}}
\newcommand{\WSE}[2]{\WS{E}{#1}{#2}}
\newcommand{\WEV}[3]{\mbox{$#1[] \vdash #2 \lra  #3$}}
\newcommand{\WEVT}[3]{\mbox{$#1[] \vdash #2 \lra$}\\ \mbox{$ #3$}}

\newcommand{\WTRED}[5]{\mbox{$#1[#2] \vdash #3 #4 #5$}}
\newcommand{\WTERED}[4]{\mbox{$E[#1] \vdash #2 #3 #4$}}
\newcommand{\WTELECONV}[3]{\WTERED{#1}{#2}{\leconvert}{#3}}
\newcommand{\WTEGRED}[3]{\WTERED{\Gamma}{#1}{#2}{#3}}
\newcommand{\WTECONV}[3]{\WTERED{#1}{#2}{\convert}{#3}}
\newcommand{\WTEGCONV}[2]{\WTERED{\Gamma}{#1}{\convert}{#2}}
\newcommand{\WTEGLECONV}[2]{\WTERED{\Gamma}{#1}{\leconvert}{#2}}

\newcommand{\lab}[1]{\mathit{labels}(#1)}
\newcommand{\dom}[1]{\mathit{dom}(#1)}

\newcommand{\CI}[2]{\mbox{$\{#1\}^{#2}$}}
\newcommand{\CIP}[3]{\mbox{$\{#1\}_{#2}^{#3}$}}
\newcommand{\CIPV}[1]{\CIP{#1}{I_1.. I_k}{P_1.. P_k}}
\newcommand{\CIPI}[1]{\CIP{#1}{I}{P}}
\newcommand{\CIF}[1]{\mbox{$\{#1\}_{f_1.. f_n}$}}
%BEGIN LATEX
\newcommand{\NInd}[3]{\mbox{{\sf Ind}$(\begin{array}[t]{@{}l}#2:=#3
                                              \,)\end{array}$}}
\newcommand{\Ind}[4]{\mbox{{\sf Ind}$[#2](\begin{array}[t]{@{}l@{}}#3:=#4
                                                 \,)\end{array}$}}
%END LATEX
%HEVEA \newcommand{\NInd}[3]{\mbox{{\sf Ind}$(#2\,:=\,#3)$}}
%HEVEA \newcommand{\Ind}[4]{\mbox{{\sf Ind}$[#2](#3\,:=\,#4)$}}

\newcommand{\Indp}[5]{\mbox{{\sf Ind}$_{#5}(#1)[#2](\begin{array}[t]{@{}l}#3:=#4
                                                 \,)\end{array}$}}
\newcommand{\Indpstr}[6]{\mbox{{\sf Ind}$_{#5}(#1)[#2](\begin{array}[t]{@{}l}#3:=#4
                                                 \,)/{#6}\end{array}$}}
\newcommand{\Def}[4]{\mbox{{\sf Def}$(#1)(#2:=#3:#4)$}}
\newcommand{\Assum}[3]{\mbox{{\sf Assum}$(#1)(#2:#3)$}}
\newcommand{\Match}[3]{\mbox{$<\!#1\!>\!{\mbox{\tt Match}}~#2~{\mbox{\tt with}}~#3~{\mbox{\tt end}}$}}
\newcommand{\Case}[3]{\mbox{$\kw{case}(#2,#1,#3)$}}
\newcommand{\match}[3]{\mbox{$\kw{match}~ #2 ~\kw{with}~ #3 ~\kw{end}$}}
\newcommand{\Fix}[2]{\mbox{\tt Fix}~#1\{#2\}}
\newcommand{\CoFix}[2]{\mbox{\tt CoFix}~#1\{#2\}}
\newcommand{\With}[2]{\mbox{\tt ~with~}}
\newcommand{\letin}[3]{\kw{let}~#1:=#2~\kw{in}~#3}
\newcommand{\subst}[3]{#1\{#2/#3\}}
\newcommand{\substs}[4]{#1\{(#2/#3)_{#4}\}}
\newcommand{\Sort}{\mbox{$\cal S$}}
\newcommand{\convert}{=_{\beta\delta\iota\zeta\eta}}
\newcommand{\leconvert}{\leq_{\beta\delta\iota\zeta\eta}}
\newcommand{\NN}{\mathbb{N}}
\newcommand{\inference}[1]{$${#1}$$}

\newcommand{\compat}[2]{\mbox{$[#1|#2]$}}
\newcommand{\tristackrel}[3]{\mathrel{\mathop{#2}\limits_{#3}^{#1}}}

\newcommand{\Impl}{{\it Impl}}
\newcommand{\elem}{{\it e}}
\newcommand{\Mod}[3]{{\sf Mod}({#1}:{#2}\,\zeroone{:={#3}})}
\newcommand{\ModS}[2]{{\sf Mod}({#1}:{#2})}
\newcommand{\ModType}[2]{{\sf ModType}({#1}:={#2})}
\newcommand{\ModA}[2]{{\sf ModA}({#1}=={#2})}
\newcommand{\functor}[3]{\ensuremath{{\sf Functor}(#1:#2)\;#3}}
\newcommand{\funsig}[3]{\ensuremath{{\sf Funsig}(#1:#2)\;#3}}
\newcommand{\sig}[1]{\ensuremath{{\sf Sig}~#1~{\sf End}}}
\newcommand{\struct}[1]{\ensuremath{{\sf Struct}~#1~{\sf End}}}
\newcommand{\structe}[1]{\ensuremath{
        {\sf Struct}~\elem_1;\ldots;\elem_i;#1;\elem_{i+2};\ldots
        ;\elem_n~{\sf End}}}
\newcommand{\structes}[2]{\ensuremath{
        {\sf Struct}~\elem_1;\ldots;\elem_i;#1;\elem_{i+2}\{#2\}
        ;\ldots;\elem_n\{#2\}~{\sf End}}}
\newcommand{\with}[3]{\ensuremath{#1~{\sf with}~#2 := #3}}

\newcommand{\Spec}{{\it Spec}}
\newcommand{\ModSEq}[3]{{\sf Mod}({#1}:{#2}:={#3})}


%\newbox\tempa
%\newbox\tempb
%\newdimen\tempc
%\newcommand{\mud}[1]{\hfil $\displaystyle{\mathstrut #1}$\hfil}
%\newcommand{\rig}[1]{\hfil $\displaystyle{#1}$}
% \newcommand{\irulehelp}[3]{\setbox\tempa=\hbox{$\displaystyle{\mathstrut #2}$}%
%                         \setbox\tempb=\vbox{\halign{##\cr
%         \mud{#1}\cr
%         \noalign{\vskip\the\lineskip}
%         \noalign{\hrule height 0pt}
%         \rig{\vbox to 0pt{\vss\hbox to 0pt{${\; #3}$\hss}\vss}}\cr
%         \noalign{\hrule}
%         \noalign{\vskip\the\lineskip}
%         \mud{\copy\tempa}\cr}}
%                       \tempc=\wd\tempb
%                       \advance\tempc by \wd\tempa
%                       \divide\tempc by 2 }
% \newcommand{\irule}[3]{{\irulehelp{#1}{#2}{#3}
%                      \hbox to \wd\tempa{\hss \box\tempb \hss}}}

\newcommand{\sverb}[1]{{\tt #1}}
\newcommand{\mover}[2]{{#1\over #2}}
\newcommand{\jd}[2]{#1 \vdash #2}
\newcommand{\mathline}[1]{\[#1\]}
\newcommand{\zrule}[2]{#2: #1}
\newcommand{\orule}[3]{#3: {\mover{#1}{#2}}}
\newcommand{\trule}[4]{#4: \mover{#1  \qquad #2} {#3}}
\newcommand{\thrule}[5]{#5: {\mover{#1  \qquad #2 \qquad #3}{#4}}}



% placement of figures

%BEGIN LATEX
\renewcommand{\topfraction}{.99}
\renewcommand{\bottomfraction}{.99}
\renewcommand{\textfraction}{.01}
\renewcommand{\floatpagefraction}{.9}
%END LATEX

% Macros Bruno pour description de la syntaxe

\def\bfbar{\ensuremath{|\hskip -0.22em{}|\hskip -0.24em{}|}}
\def\TERMbar{\bfbar}
\def\TERMbarbar{\bfbar\bfbar}


%% Macros pour les grammaires
\def\GR#1{\text{\large(}#1\text{\large)}}
\def\NT#1{\langle\textit{#1}\rangle}
\def\NTL#1#2{\langle\textit{#1}\rangle_{#2}}
\def\TERM#1{{\bf\textrm{\bf #1}}}
%\def\TERM#1{{\bf\textsf{#1}}}
\def\KWD#1{\TERM{#1}}
\def\ETERM#1{\TERM{#1}}
\def\CHAR#1{\TERM{#1}}

\def\STAR#1{#1*}
\def\STARGR#1{\GR{#1}*}
\def\PLUS#1{#1+}
\def\PLUSGR#1{\GR{#1}+}
\def\OPT#1{#1?}
\def\OPTGR#1{\GR{#1}?}
%% Tableaux de definition de non-terminaux
\newenvironment{cadre}
        {\begin{array}{|c|}\hline\\}
        {\\\\\hline\end{array}}
\newenvironment{rulebox}
        {$$\begin{cadre}\begin{array}{r@{~}c@{~}l@{}l@{}r}}
        {\end{array}\end{cadre}$$}
\def\DEFNT#1{\NT{#1} & ::= &}
\def\EXTNT#1{\NT{#1} & ::= & ... \\&|&}
\def\RNAME#1{(\textsc{#1})}
\def\SEPDEF{\\\\}
\def\nlsep{\\&|&}
\def\nlcont{\\&&}
\newenvironment{rules}
        {\begin{center}\begin{rulebox}}
        {\end{rulebox}\end{center}}


%%% Local Variables: 
%%% mode: latex
%%% TeX-master: "Reference-Manual"
%%% End: 


\begin{document}
 
\coverpage{The standard library}%
{\ }
{This material is distributed under the terms of the GNU Lesser
General Public License Version 2.1.}

\tableofcontents

\newpage
% \section*{The \Coq\ standard library}

This document is a short description of the \Coq\ standard library.
This library comes with the system as a complement of the core library
(the {\bf Init} library ; see the Reference Manual for a description
of this library). It provides a set of modules directly available
through the \verb!Require! command.

The standard library is composed of the following subdirectories:
\begin{description}
  \item[Logic]  Classical logic and dependent equality
  \item[Bool]   Booleans (basic functions and results)
  \item[Arith]  Basic Peano arithmetic
  \item[ZArith] Basic integer arithmetic
  \item[Reals]  Classical Real Numbers and Analysis
  \item[Lists]  Monomorphic and polymorphic lists (basic functions and
                  results), Streams (infinite sequences defined 
                  with co-inductive types)
  \item[Sets]   Sets (classical, constructive, finite, infinite, power set,
                  etc.)
  \item[Relations] Relations (definitions and basic results).
  \item[Sorting] Sorted list (basic definitions and heapsort
                 correctness). 
  \item[Wellfounded] Well-founded relations (basic results).
  \item[Program] Tactics to deal with dependently-typed programs and
    their proofs.
  \item[Classes] Standard type class instances on relations and
    Coq part of the setoid rewriting tactic.
\end{description}


Each of these subdirectories contains a set of modules, whose
specifications (\gallina{} files) have
been roughly, and automatically, pasted in the following pages. There
is also a version of this document in HTML format on the WWW, which
you can access from the \Coq\ home page at
\texttt{http://coq.inria.fr/library}.

\documentclass[11pt]{report}

\usepackage[mathletters]{ucs}
\usepackage[utf8x]{inputenc}
\usepackage[T1]{fontenc}
\usepackage{fullpage}
\usepackage{amsfonts}
\usepackage{amssymb}
\usepackage{url}
\usepackage[color]{../../tools/coqdoc/coqdoc}

\input{../common/version}
%%%%%%%%%%%%%%%%%%%%%%%%%%%%%%%%
% File title.tex
% Page formatting commands
% Macro \coverpage
%%%%%%%%%%%%%%%%%%%%%%%%%%%%%%%%

%\setlength{\marginparwidth}{0pt}
%\setlength{\oddsidemargin}{0pt}
%\setlength{\evensidemargin}{0pt}
%\setlength{\marginparsep}{0pt}
%\setlength{\topmargin}{0pt}
%\setlength{\textwidth}{16.9cm}
%\setlength{\textheight}{22cm}
%\usepackage{fullpage}

%\newcommand{\printingdate}{\today}
%\newcommand{\isdraft}{\Large\bf\today\\[20pt]}
%\newcommand{\isdraft}{\vspace{20pt}}

\newcommand{\coverpage}[3]{
\thispagestyle{empty}
\begin{center}
\bfseries % for the rest of this page, until \end{center}
\Huge
The Coq Proof Assistant\\[12pt]
#1\\[20pt]
\Large\today\\[20pt]
Version \coqversion\footnote[1]{This research was partly supported by IST working group ``Types''}

\vspace{0pt plus .5fill}
#2
\par\vfill
$\pi r^2$ Project (formerly LogiCal, then TypiCal)

\vspace*{15pt}
\end{center}
\newpage

\thispagestyle{empty}
\hbox{}\vfill % without \hbox \vfill does not work at the top of the page
\begin{flushleft}
%BEGIN LATEX
V\coqversion, \today
\par\vspace{20pt}
%END LATEX
\copyright INRIA 1999-2004 ({\Coq} versions 7.x)

\copyright INRIA 2004-2017 ({\Coq} versions 8.x)

#3
\end{flushleft}
} % end of \coverpage definition


% \newcommand{\shorttitle}[1]{
% \begin{center}
% \begin{huge}
% \begin{bf}
% The Coq Proof Assistant\\
% \vspace{10pt}
%     #1\\
% \end{bf}
% \end{huge}
% \end{center}
% \vspace{5pt}
% }

% Local Variables: 
% mode: LaTeX
% TeX-master: ""
% End: 


%%%%%%%%%%%%%%%%%%%%%%%%%%%%%%%%%%%%%%%%%%
% MACROS FOR THE REFERENCE MANUAL OF COQ %
%%%%%%%%%%%%%%%%%%%%%%%%%%%%%%%%%%%%%%%%%%

% For commentaries (define \com as {} for the release manual)
%\newcommand{\com}[1]{{\it(* #1 *)}}
%\newcommand{\com}[1]{}

%%OPTIONS for HACHA
%\renewcommand{\cuttingunit}{section}


%BEGIN LATEX
\newenvironment{centerframe}%
{\bgroup
\dimen0=\textwidth
\advance\dimen0 by -2\fboxrule
\advance\dimen0 by -2\fboxsep
\setbox0=\hbox\bgroup
\begin{minipage}{\dimen0}%
\begin{center}}%
{\end{center}%
\end{minipage}\egroup
\centerline{\fbox{\box0}}\egroup
}
%END LATEX
%HEVEA \newenvironment{centerframe}{\begin{center}}{\end{center}}

%HEVEA \renewcommand{\vec}[1]{\mathbf{#1}}
%\renewcommand{\ominus}{-} % Hevea does a good job translating these commands
%\renewcommand{\oplus}{+}
%\renewcommand{\otimes}{\times}
%\newcommand{\land}{\wedge}
%\newcommand{\lor}{\vee}
%HEVEA \renewcommand{\k}[1]{#1} % \k{a} is supposed to produce a with a little stroke
%HEVEA \newcommand{\phantom}[1]{\qquad}

%%%%%%%%%%%%%%%%%%%%%%%
% Formatting commands %
%%%%%%%%%%%%%%%%%%%%%%%

\newcommand{\ErrMsg}{\medskip \noindent {\bf Error message: }}
\newcommand{\ErrMsgx}{\medskip \noindent {\bf Error messages: }}
\newcommand{\variant}{\medskip \noindent {\bf Variant: }}
\newcommand{\variants}{\medskip \noindent {\bf Variants: }}
\newcommand{\SeeAlso}{\medskip \noindent {\bf See also: }}
\newcommand{\Rem}{\medskip \noindent {\bf Remark: }}
\newcommand{\Rems}{\medskip \noindent {\bf Remarks: }}
\newcommand{\Example}{\medskip \noindent {\bf Example: }}
\newcommand{\examples}{\medskip \noindent {\bf Examples: }}
\newcommand{\Warning}{\medskip \noindent {\bf Warning: }}
\newcommand{\Warns}{\medskip \noindent {\bf Warnings: }}
\newcounter{ex}
\newcommand{\firstexample}{\setcounter{ex}{1}}
\newcommand{\example}[1]{
\medskip \noindent \textbf{Example \arabic{ex}: }\textit{#1}
\addtocounter{ex}{1}}

\newenvironment{Variant}{\variant\begin{enumerate}}{\end{enumerate}}
\newenvironment{Variants}{\variants\begin{enumerate}}{\end{enumerate}}
\newenvironment{ErrMsgs}{\ErrMsgx\begin{enumerate}}{\end{enumerate}}
\newenvironment{Remarks}{\Rems\begin{enumerate}}{\end{enumerate}}
\newenvironment{Warnings}{\Warns\begin{enumerate}}{\end{enumerate}}
\newenvironment{Examples}{\medskip\noindent{\bf Examples:}
\begin{enumerate}}{\end{enumerate}}

%\newcommand{\bd}{\noindent\bf}
%\newcommand{\sbd}{\vspace{8pt}\noindent\bf}
%\newcommand{\sdoll}[1]{\begin{small}$ #1~ $\end{small}}
%\newcommand{\sdollnb}[1]{\begin{small}$ #1 $\end{small}}
\newcommand{\kw}[1]{\textsf{#1}}
%\newcommand{\spec}[1]{\{\,#1\,\}}

% Building regular expressions
\newcommand{\zeroone}[1]{\mbox{\sl [}#1\mbox{\sl ]}}
%\newcommand{\zeroonemany}[1]{$\{$#1$\}$*}
%\newcommand{\onemany}[1]{$\{$#1$\}$+}
\newcommand{\nelistnosep}[1]{{#1} \mbox{\dots} {#1}}
\newcommand{\nelist}[2]{{#1} {\tt #2} \mbox{\dots} {\tt #2} {#1}}
\newcommand{\sequence}[2]{{\sl [}{#1} {\tt #2} \mbox{\dots} {\tt #2} {#1}{\sl ]}}
\newcommand{\nelistwithoutblank}[2]{#1{\tt #2}\mbox{\dots}{\tt #2}#1}
\newcommand{\sequencewithoutblank}[2]{$[$#1{\tt #2}\mbox{\dots}{\tt #2}#1$]$}

% Used for RefMan-gal
%\newcommand{\ml}[1]{\hbox{\tt{#1}}}
%\newcommand{\op}{\,|\,}

%%%%%%%%%%%%%%%%%%%%%%%%
% Trademarks and so on %
%%%%%%%%%%%%%%%%%%%%%%%%

\newcommand{\Coq}{\textsc{Coq}}
\newcommand{\gallina}{\textsc{Gallina}}
\newcommand{\Gallina}{\textsc{Gallina}}
\newcommand{\CoqIDE}{\textsc{CoqIDE}}
\newcommand{\ocaml}{\textsc{Objective Caml}}
\newcommand{\camlpppp}{\textsc{Camlp4}}
\newcommand{\emacs}{\textsc{GNU Emacs}}
\newcommand{\ProofGeneral}{\textsc{Proof General}}
\newcommand{\CIC}{\textsc{Cic}}
\newcommand{\iCIC}{\textsc{Cic}}
\newcommand{\FW}{\ensuremath{F_{\omega}}}
\newcommand{\Program}{\textsc{Program}}
\newcommand{\Russell}{\textsc{Russell}}
\newcommand{\PVS}{\textsc{PVS}}
%\newcommand{\bn}{{\sf BNF}}

%%%%%%%%%%%%%%%%%%%
% Name of tactics %
%%%%%%%%%%%%%%%%%%%

%\newcommand{\Natural}{\mbox{\tt Natural}}

%%%%%%%%%%%%%%%%%
% \rm\sl series %
%%%%%%%%%%%%%%%%%

\newcommand{\nterm}[1]{\textrm{\textsl{#1}}}

\newcommand{\qstring}{\nterm{string}}

%% New syntax specific entries
\newcommand{\annotation}{\nterm{annotation}}
\newcommand{\assums}{\nterm{assums}} % vernac
\newcommand{\simpleassums}{\nterm{simple\_assums}} % assumptions
\newcommand{\binder}{\nterm{binder}}
\newcommand{\binders}{\nterm{binders}}
\newcommand{\caseitems}{\nterm{match\_items}}
\newcommand{\caseitem}{\nterm{match\_item}}
\newcommand{\eqn}{\nterm{equation}}
\newcommand{\ifitem}{\nterm{dep\_ret\_type}}
\newcommand{\hyplocation}{\nterm{hyp\_location}}
\newcommand{\convclause}{\nterm{conversion\_clause}}
\newcommand{\occclause}{\nterm{occurrence\_clause}}
\newcommand{\occgoalset}{\nterm{goal\_occurrences}}
\newcommand{\atoccurrences}{\nterm{at\_occurrences}}
\newcommand{\occlist}{\nterm{occurrences}}
\newcommand{\params}{\nterm{params}} % vernac
\newcommand{\returntype}{\nterm{return\_type}}
\newcommand{\idparams}{\nterm{ident\_with\_params}}
\newcommand{\statkwd}{\nterm{assertion\_keyword}} % vernac
\newcommand{\termarg}{\nterm{arg}}
\newcommand{\hintdef}{\nterm{hint\_definition}}

\newcommand{\typecstr}{\zeroone{{\tt :}~{\term}}}
\newcommand{\typecstrwithoutblank}{\zeroone{{\tt :}{\term}}}


\newcommand{\Fwterm}{\nterm{Fwterm}}
\newcommand{\Index}{\nterm{index}}
\newcommand{\abbrev}{\nterm{abbreviation}}
\newcommand{\atomictac}{\nterm{atomic\_tactic}}
\newcommand{\bindinglist}{\nterm{bindings\_list}}
\newcommand{\cast}{\nterm{cast}}
\newcommand{\cofixpointbodies}{\nterm{cofix\_bodies}}
\newcommand{\cofixpointbody}{\nterm{cofix\_body}}
\newcommand{\commandtac}{\nterm{tactic\_invocation}}
\newcommand{\constructor}{\nterm{constructor}}
\newcommand{\convtactic}{\nterm{conv\_tactic}}
\newcommand{\assumptionkeyword}{\nterm{assumption\_keyword}}
\newcommand{\assumption}{\nterm{assumption}}
\newcommand{\definition}{\nterm{definition}}
\newcommand{\digit}{\nterm{digit}}
\newcommand{\exteqn}{\nterm{ext\_eqn}}
\newcommand{\field}{\nterm{field}}
\newcommand{\firstletter}{\nterm{first\_letter}}
\newcommand{\fixpg}{\nterm{fix\_pgm}}
\newcommand{\fixpointbodies}{\nterm{fix\_bodies}}
\newcommand{\fixpointbody}{\nterm{fix\_body}}
\newcommand{\fixpoint}{\nterm{fixpoint}}
\newcommand{\flag}{\nterm{flag}}
\newcommand{\form}{\nterm{form}}
\newcommand{\entry}{\nterm{entry}} 
\newcommand{\proditem}{\nterm{prod\_item}}
\newcommand{\taclevel}{\nterm{tactic\_level}}
\newcommand{\tacargtype}{\nterm{tactic\_argument\_type}} 
\newcommand{\scope}{\nterm{scope}} 
\newcommand{\delimkey}{\nterm{key}} 
\newcommand{\optscope}{\nterm{opt\_scope}} 
\newcommand{\declnotation}{\nterm{decl\_notation}} 
\newcommand{\symbolentry}{\nterm{symbol}}
\newcommand{\modifiers}{\nterm{modifiers}}
\newcommand{\localdef}{\nterm{local\_def}}
\newcommand{\localdecls}{\nterm{local\_decls}}
\newcommand{\ident}{\nterm{ident}}
\newcommand{\accessident}{\nterm{access\_ident}}
\newcommand{\possiblybracketedident}{\nterm{possibly\_bracketed\_ident}}
\newcommand{\inductivebody}{\nterm{ind\_body}}
\newcommand{\inductive}{\nterm{inductive}}
\newcommand{\naturalnumber}{\nterm{natural}}
\newcommand{\integer}{\nterm{integer}}
\newcommand{\multpattern}{\nterm{mult\_pattern}}
\newcommand{\mutualcoinductive}{\nterm{mutual\_coinductive}}
\newcommand{\mutualinductive}{\nterm{mutual\_inductive}}
\newcommand{\nestedpattern}{\nterm{nested\_pattern}}
\newcommand{\name}{\nterm{name}}
\newcommand{\num}{\nterm{num}}
\newcommand{\coqindex}{\nterm{index}}
\newcommand{\pattern}{\nterm{pattern}} % pattern for pattern-matching
\newcommand{\orpattern}{\nterm{or\_pattern}}
\newcommand{\intropattern}{\nterm{intro\_pattern}}
\newcommand{\intropatternlist}{\nterm{intro\_pattern\_list}}
\newcommand{\disjconjintropattern}{\nterm{disj\_conj\_intro\_pattern}}
\newcommand{\namingintropattern}{\nterm{naming\_intro\_pattern}}
\newcommand{\termpattern}{\nterm{term\_pattern}} % term with holes
\newcommand{\pat}{\nterm{pat}}
\newcommand{\pgs}{\nterm{pgms}}
\newcommand{\pg}{\nterm{pgm}}
\newcommand{\abullet}{\nterm{bullet}}
%BEGIN LATEX
\newcommand{\proof}{\nterm{proof}}
%END LATEX
%HEVEA \renewcommand{\proof}{\nterm{proof}}
\newcommand{\record}{\nterm{record}}
\newcommand{\recordkw}{\nterm{record\_keyword}}
\newcommand{\rewrule}{\nterm{rewriting\_rule}}
\newcommand{\sentence}{\nterm{sentence}}
\newcommand{\simplepattern}{\nterm{simple\_pattern}}
\newcommand{\sort}{\nterm{sort}}
\newcommand{\specif}{\nterm{specif}}
\newcommand{\assertion}{\nterm{assertion}}
\newcommand{\str}{\nterm{string}}
\newcommand{\subsequentletter}{\nterm{subsequent\_letter}}
\newcommand{\switch}{\nterm{switch}}
\newcommand{\messagetoken}{\nterm{message\_token}}
\newcommand{\tac}{\nterm{tactic}}
\newcommand{\terms}{\nterm{terms}}
\newcommand{\term}{\nterm{term}}
\newcommand{\module}{\nterm{module}}
\newcommand{\modexpr}{\nterm{module\_expression}}
\newcommand{\modtype}{\nterm{module\_type}}
\newcommand{\onemodbinding}{\nterm{module\_binding}}
\newcommand{\modbindings}{\nterm{module\_bindings}}
\newcommand{\qualid}{\nterm{qualid}}
\newcommand{\qualidorstring}{\nterm{qualid\_or\_string}}
\newcommand{\class}{\nterm{class}}
\newcommand{\dirpath}{\nterm{dirpath}}
\newcommand{\typedidents}{\nterm{typed\_idents}}
\newcommand{\type}{\nterm{type}}
\newcommand{\vref}{\nterm{ref}}
\newcommand{\zarithformula}{\nterm{zarith\_formula}}
\newcommand{\zarith}{\nterm{zarith}}
\newcommand{\ltac}{\mbox{${\cal L}_{tac}$}}

%%%%%%%%%%%%%%%%%%%%%%%%%%%%%%%%%%%%%%%%%%%%%%%%%%%%%%%
% \mbox{\sf } series for roman text in maths formulas %
%%%%%%%%%%%%%%%%%%%%%%%%%%%%%%%%%%%%%%%%%%%%%%%%%%%%%%%

\newcommand{\alors}{\mbox{\textsf{then}}}
\newcommand{\alter}{\mbox{\textsf{alter}}}
\newcommand{\bool}{\mbox{\textsf{bool}}}
\newcommand{\conc}{\mbox{\textsf{conc}}}
\newcommand{\cons}{\mbox{\textsf{cons}}}
\newcommand{\consf}{\mbox{\textsf{consf}}}
\newcommand{\emptyf}{\mbox{\textsf{emptyf}}}
\newcommand{\EqSt}{\mbox{\textsf{EqSt}}}
\newcommand{\false}{\mbox{\textsf{false}}}
\newcommand{\filter}{\mbox{\textsf{filter}}}
\newcommand{\forest}{\mbox{\textsf{forest}}}
\newcommand{\from}{\mbox{\textsf{from}}}
\newcommand{\hd}{\mbox{\textsf{hd}}}
\newcommand{\haslength}{\mbox{\textsf{has\_length}}}
\newcommand{\length}{\mbox{\textsf{length}}}
\newcommand{\haslengthA}{\mbox {\textsf{has\_length~A}}}
\newcommand{\List}{\mbox{\textsf{list}}}
\newcommand{\ListA}{\mbox{\textsf{list}}~\ensuremath{A}}
\newcommand{\nilhl}{\mbox{\textsf{nil\_hl}}}
\newcommand{\conshl}{\mbox{\textsf{cons\_hl}}}
\newcommand{\nat}{\mbox{\textsf{nat}}}
\newcommand{\nO}{\mbox{\textsf{O}}}
\newcommand{\nS}{\mbox{\textsf{S}}}
\newcommand{\node}{\mbox{\textsf{node}}}
\newcommand{\Nil}{\mbox{\textsf{nil}}}
\newcommand{\Prop}{\mbox{\textsf{Prop}}}
\newcommand{\Set}{\mbox{\textsf{Set}}}
\newcommand{\si}{\mbox{\textsf{if}}}
\newcommand{\sinon}{\mbox{\textsf{else}}}
\newcommand{\Str}{\mbox{\textsf{Stream}}}
\newcommand{\tl}{\mbox{\textsf{tl}}}
\newcommand{\tree}{\mbox{\textsf{tree}}}
\newcommand{\true}{\mbox{\textsf{true}}}
\newcommand{\Type}{\mbox{\textsf{Type}}}
\newcommand{\unfold}{\mbox{\textsf{unfold}}}
\newcommand{\zeros}{\mbox{\textsf{zeros}}}
\newcommand{\even}{\mbox{\textsf{even}}}
\newcommand{\odd}{\mbox{\textsf{odd}}}
\newcommand{\evenO}{\mbox{\textsf{even\_O}}}
\newcommand{\evenS}{\mbox{\textsf{even\_S}}}
\newcommand{\oddS}{\mbox{\textsf{odd\_S}}}
\newcommand{\Prod}{\mbox{\textsf{prod}}}
\newcommand{\Pair}{\mbox{\textsf{pair}}}

%%%%%%%%%
% Misc. %
%%%%%%%%%
\newcommand{\T}{\texttt{T}}
\newcommand{\U}{\texttt{U}}
\newcommand{\real}{\textsf{Real}}
\newcommand{\Data}{\textit{Data}}
\newcommand{\In} {{\textbf{in }}}
\newcommand{\AND} {{\textbf{and}}}
\newcommand{\If}{{\textbf{if }}}
\newcommand{\Else}{{\textbf{else }}}
\newcommand{\Then} {{\textbf{then }}}
%\newcommand{\Let}{{\textbf{let }}} % looks like this is never used
\newcommand{\Where}{{\textbf{where rec }}}
\newcommand{\Function}{{\textbf{function }}}
\newcommand{\Rec}{{\textbf{rec }}}
%\newcommand{\cn}{\centering}
\newcommand{\nth}{\mbox{$^{\mbox{\scriptsize th}}$}}

%%%%%%%%%%%%%%%%%%%%%%%%%%%%%
% Math commands and symbols %
%%%%%%%%%%%%%%%%%%%%%%%%%%%%%

\newcommand{\la}{\leftarrow}
\newcommand{\ra}{\rightarrow}
\newcommand{\Ra}{\Rightarrow}
\newcommand{\rt}{\Rightarrow}
\newcommand{\lla}{\longleftarrow}
\newcommand{\lra}{\longrightarrow}
\newcommand{\Llra}{\Longleftrightarrow}
\newcommand{\mt}{\mapsto}
\newcommand{\ov}{\overrightarrow}
\newcommand{\wh}{\widehat}
\newcommand{\up}{\uparrow}
\newcommand{\dw}{\downarrow}
\newcommand{\nr}{\nearrow}
\newcommand{\se}{\searrow}
\newcommand{\sw}{\swarrow}
\newcommand{\nw}{\nwarrow}
\newcommand{\mto}{.\;}

\newcommand{\vm}[1]{\vspace{#1em}}
\newcommand{\vx}[1]{\vspace{#1ex}}
\newcommand{\hm}[1]{\hspace{#1em}}
\newcommand{\hx}[1]{\hspace{#1ex}}
\newcommand{\sm}{\mbox{ }}
\newcommand{\mx}{\mbox}

%\newcommand{\nq}{\neq}
%\newcommand{\eq}{\equiv}
\newcommand{\fa}{\forall}
%\newcommand{\ex}{\exists}
\newcommand{\impl}{\rightarrow}
%\newcommand{\Or}{\vee}
%\newcommand{\And}{\wedge}
\newcommand{\ms}{\models}
\newcommand{\bw}{\bigwedge}
\newcommand{\ts}{\times}
\newcommand{\cc}{\circ}
%\newcommand{\es}{\emptyset}
%\newcommand{\bs}{\backslash}
\newcommand{\vd}{\vdash}
%\newcommand{\lan}{{\langle }}
%\newcommand{\ran}{{\rangle }}

%\newcommand{\al}{\alpha}
\newcommand{\bt}{\beta}
%\newcommand{\io}{\iota}
\newcommand{\lb}{\lambda}
%\newcommand{\sg}{\sigma}
%\newcommand{\sa}{\Sigma}
%\newcommand{\om}{\Omega}
%\newcommand{\tu}{\tau}

%%%%%%%%%%%%%%%%%%%%%%%%%
% Custom maths commands %
%%%%%%%%%%%%%%%%%%%%%%%%%

\newcommand{\sumbool}[2]{\{#1\}+\{#2\}}
\newcommand{\myifthenelse}[3]{\kw{if} ~ #1 ~\kw{then} ~ #2 ~ \kw{else} ~ #3}
\newcommand{\fun}[2]{\item[]{\tt {#1}}. \quad\\ #2}
\newcommand{\WF}[2]{\ensuremath{{\cal W\!F}(#1)[#2]}}
\newcommand{\WFTWOLINES}[2]{\ensuremath{{\cal W\!F}\begin{array}{l}(#1)\\\mbox{}[{#2}]\end{array}}}
\newcommand{\WFE}[1]{\WF{E}{#1}}
\newcommand{\WT}[4]{\ensuremath{#1[#2] \vdash #3 : #4}}
\newcommand{\WTE}[3]{\WT{E}{#1}{#2}{#3}}
\newcommand{\WTEG}[2]{\WTE{\Gamma}{#1}{#2}}

\newcommand{\WTM}[3]{\WT{#1}{}{#2}{#3}}
\newcommand{\WFT}[2]{\ensuremath{#1[] \vdash {\cal W\!F}(#2)}}
\newcommand{\WS}[3]{\ensuremath{#1[] \vdash #2 <: #3}}
\newcommand{\WSE}[2]{\WS{E}{#1}{#2}}
\newcommand{\WEV}[3]{\mbox{$#1[] \vdash #2 \lra  #3$}}
\newcommand{\WEVT}[3]{\mbox{$#1[] \vdash #2 \lra$}\\ \mbox{$ #3$}}

\newcommand{\WTRED}[5]{\mbox{$#1[#2] \vdash #3 #4 #5$}}
\newcommand{\WTERED}[4]{\mbox{$E[#1] \vdash #2 #3 #4$}}
\newcommand{\WTELECONV}[3]{\WTERED{#1}{#2}{\leconvert}{#3}}
\newcommand{\WTEGRED}[3]{\WTERED{\Gamma}{#1}{#2}{#3}}
\newcommand{\WTECONV}[3]{\WTERED{#1}{#2}{\convert}{#3}}
\newcommand{\WTEGCONV}[2]{\WTERED{\Gamma}{#1}{\convert}{#2}}
\newcommand{\WTEGLECONV}[2]{\WTERED{\Gamma}{#1}{\leconvert}{#2}}

\newcommand{\lab}[1]{\mathit{labels}(#1)}
\newcommand{\dom}[1]{\mathit{dom}(#1)}

\newcommand{\CI}[2]{\mbox{$\{#1\}^{#2}$}}
\newcommand{\CIP}[3]{\mbox{$\{#1\}_{#2}^{#3}$}}
\newcommand{\CIPV}[1]{\CIP{#1}{I_1.. I_k}{P_1.. P_k}}
\newcommand{\CIPI}[1]{\CIP{#1}{I}{P}}
\newcommand{\CIF}[1]{\mbox{$\{#1\}_{f_1.. f_n}$}}
%BEGIN LATEX
\newcommand{\NInd}[3]{\mbox{{\sf Ind}$(\begin{array}[t]{@{}l}#2:=#3
                                              \,)\end{array}$}}
\newcommand{\Ind}[4]{\mbox{{\sf Ind}$[#2](\begin{array}[t]{@{}l@{}}#3:=#4
                                                 \,)\end{array}$}}
%END LATEX
%HEVEA \newcommand{\NInd}[3]{\mbox{{\sf Ind}$(#2\,:=\,#3)$}}
%HEVEA \newcommand{\Ind}[4]{\mbox{{\sf Ind}$[#2](#3\,:=\,#4)$}}

\newcommand{\Indp}[5]{\mbox{{\sf Ind}$_{#5}(#1)[#2](\begin{array}[t]{@{}l}#3:=#4
                                                 \,)\end{array}$}}
\newcommand{\Indpstr}[6]{\mbox{{\sf Ind}$_{#5}(#1)[#2](\begin{array}[t]{@{}l}#3:=#4
                                                 \,)/{#6}\end{array}$}}
\newcommand{\Def}[4]{\mbox{{\sf Def}$(#1)(#2:=#3:#4)$}}
\newcommand{\Assum}[3]{\mbox{{\sf Assum}$(#1)(#2:#3)$}}
\newcommand{\Match}[3]{\mbox{$<\!#1\!>\!{\mbox{\tt Match}}~#2~{\mbox{\tt with}}~#3~{\mbox{\tt end}}$}}
\newcommand{\Case}[3]{\mbox{$\kw{case}(#2,#1,#3)$}}
\newcommand{\match}[3]{\mbox{$\kw{match}~ #2 ~\kw{with}~ #3 ~\kw{end}$}}
\newcommand{\Fix}[2]{\mbox{\tt Fix}~#1\{#2\}}
\newcommand{\CoFix}[2]{\mbox{\tt CoFix}~#1\{#2\}}
\newcommand{\With}[2]{\mbox{\tt ~with~}}
\newcommand{\letin}[3]{\kw{let}~#1:=#2~\kw{in}~#3}
\newcommand{\subst}[3]{#1\{#2/#3\}}
\newcommand{\substs}[4]{#1\{(#2/#3)_{#4}\}}
\newcommand{\Sort}{\mbox{$\cal S$}}
\newcommand{\convert}{=_{\beta\delta\iota\zeta\eta}}
\newcommand{\leconvert}{\leq_{\beta\delta\iota\zeta\eta}}
\newcommand{\NN}{\mathbb{N}}
\newcommand{\inference}[1]{$${#1}$$}

\newcommand{\compat}[2]{\mbox{$[#1|#2]$}}
\newcommand{\tristackrel}[3]{\mathrel{\mathop{#2}\limits_{#3}^{#1}}}

\newcommand{\Impl}{{\it Impl}}
\newcommand{\elem}{{\it e}}
\newcommand{\Mod}[3]{{\sf Mod}({#1}:{#2}\,\zeroone{:={#3}})}
\newcommand{\ModS}[2]{{\sf Mod}({#1}:{#2})}
\newcommand{\ModType}[2]{{\sf ModType}({#1}:={#2})}
\newcommand{\ModA}[2]{{\sf ModA}({#1}=={#2})}
\newcommand{\functor}[3]{\ensuremath{{\sf Functor}(#1:#2)\;#3}}
\newcommand{\funsig}[3]{\ensuremath{{\sf Funsig}(#1:#2)\;#3}}
\newcommand{\sig}[1]{\ensuremath{{\sf Sig}~#1~{\sf End}}}
\newcommand{\struct}[1]{\ensuremath{{\sf Struct}~#1~{\sf End}}}
\newcommand{\structe}[1]{\ensuremath{
        {\sf Struct}~\elem_1;\ldots;\elem_i;#1;\elem_{i+2};\ldots
        ;\elem_n~{\sf End}}}
\newcommand{\structes}[2]{\ensuremath{
        {\sf Struct}~\elem_1;\ldots;\elem_i;#1;\elem_{i+2}\{#2\}
        ;\ldots;\elem_n\{#2\}~{\sf End}}}
\newcommand{\with}[3]{\ensuremath{#1~{\sf with}~#2 := #3}}

\newcommand{\Spec}{{\it Spec}}
\newcommand{\ModSEq}[3]{{\sf Mod}({#1}:{#2}:={#3})}


%\newbox\tempa
%\newbox\tempb
%\newdimen\tempc
%\newcommand{\mud}[1]{\hfil $\displaystyle{\mathstrut #1}$\hfil}
%\newcommand{\rig}[1]{\hfil $\displaystyle{#1}$}
% \newcommand{\irulehelp}[3]{\setbox\tempa=\hbox{$\displaystyle{\mathstrut #2}$}%
%                         \setbox\tempb=\vbox{\halign{##\cr
%         \mud{#1}\cr
%         \noalign{\vskip\the\lineskip}
%         \noalign{\hrule height 0pt}
%         \rig{\vbox to 0pt{\vss\hbox to 0pt{${\; #3}$\hss}\vss}}\cr
%         \noalign{\hrule}
%         \noalign{\vskip\the\lineskip}
%         \mud{\copy\tempa}\cr}}
%                       \tempc=\wd\tempb
%                       \advance\tempc by \wd\tempa
%                       \divide\tempc by 2 }
% \newcommand{\irule}[3]{{\irulehelp{#1}{#2}{#3}
%                      \hbox to \wd\tempa{\hss \box\tempb \hss}}}

\newcommand{\sverb}[1]{{\tt #1}}
\newcommand{\mover}[2]{{#1\over #2}}
\newcommand{\jd}[2]{#1 \vdash #2}
\newcommand{\mathline}[1]{\[#1\]}
\newcommand{\zrule}[2]{#2: #1}
\newcommand{\orule}[3]{#3: {\mover{#1}{#2}}}
\newcommand{\trule}[4]{#4: \mover{#1  \qquad #2} {#3}}
\newcommand{\thrule}[5]{#5: {\mover{#1  \qquad #2 \qquad #3}{#4}}}



% placement of figures

%BEGIN LATEX
\renewcommand{\topfraction}{.99}
\renewcommand{\bottomfraction}{.99}
\renewcommand{\textfraction}{.01}
\renewcommand{\floatpagefraction}{.9}
%END LATEX

% Macros Bruno pour description de la syntaxe

\def\bfbar{\ensuremath{|\hskip -0.22em{}|\hskip -0.24em{}|}}
\def\TERMbar{\bfbar}
\def\TERMbarbar{\bfbar\bfbar}


%% Macros pour les grammaires
\def\GR#1{\text{\large(}#1\text{\large)}}
\def\NT#1{\langle\textit{#1}\rangle}
\def\NTL#1#2{\langle\textit{#1}\rangle_{#2}}
\def\TERM#1{{\bf\textrm{\bf #1}}}
%\def\TERM#1{{\bf\textsf{#1}}}
\def\KWD#1{\TERM{#1}}
\def\ETERM#1{\TERM{#1}}
\def\CHAR#1{\TERM{#1}}

\def\STAR#1{#1*}
\def\STARGR#1{\GR{#1}*}
\def\PLUS#1{#1+}
\def\PLUSGR#1{\GR{#1}+}
\def\OPT#1{#1?}
\def\OPTGR#1{\GR{#1}?}
%% Tableaux de definition de non-terminaux
\newenvironment{cadre}
        {\begin{array}{|c|}\hline\\}
        {\\\\\hline\end{array}}
\newenvironment{rulebox}
        {$$\begin{cadre}\begin{array}{r@{~}c@{~}l@{}l@{}r}}
        {\end{array}\end{cadre}$$}
\def\DEFNT#1{\NT{#1} & ::= &}
\def\EXTNT#1{\NT{#1} & ::= & ... \\&|&}
\def\RNAME#1{(\textsc{#1})}
\def\SEPDEF{\\\\}
\def\nlsep{\\&|&}
\def\nlcont{\\&&}
\newenvironment{rules}
        {\begin{center}\begin{rulebox}}
        {\end{rulebox}\end{center}}


%%% Local Variables: 
%%% mode: latex
%%% TeX-master: "Reference-Manual"
%%% End: 


\begin{document}
 
\coverpage{The standard library}%
{\ }
{This material is distributed under the terms of the GNU Lesser
General Public License Version 2.1.}

\tableofcontents

\newpage
% \section*{The \Coq\ standard library}

This document is a short description of the \Coq\ standard library.
This library comes with the system as a complement of the core library
(the {\bf Init} library ; see the Reference Manual for a description
of this library). It provides a set of modules directly available
through the \verb!Require! command.

The standard library is composed of the following subdirectories:
\begin{description}
  \item[Logic]  Classical logic and dependent equality
  \item[Bool]   Booleans (basic functions and results)
  \item[Arith]  Basic Peano arithmetic
  \item[ZArith] Basic integer arithmetic
  \item[Reals]  Classical Real Numbers and Analysis
  \item[Lists]  Monomorphic and polymorphic lists (basic functions and
                  results), Streams (infinite sequences defined 
                  with co-inductive types)
  \item[Sets]   Sets (classical, constructive, finite, infinite, power set,
                  etc.)
  \item[Relations] Relations (definitions and basic results).
  \item[Sorting] Sorted list (basic definitions and heapsort
                 correctness). 
  \item[Wellfounded] Well-founded relations (basic results).
  \item[Program] Tactics to deal with dependently-typed programs and
    their proofs.
  \item[Classes] Standard type class instances on relations and
    Coq part of the setoid rewriting tactic.
\end{description}


Each of these subdirectories contains a set of modules, whose
specifications (\gallina{} files) have
been roughly, and automatically, pasted in the following pages. There
is also a version of this document in HTML format on the WWW, which
you can access from the \Coq\ home page at
\texttt{http://coq.inria.fr/library}.

\documentclass[11pt]{report}

\usepackage[mathletters]{ucs}
\usepackage[utf8x]{inputenc}
\usepackage[T1]{fontenc}
\usepackage{fullpage}
\usepackage{amsfonts}
\usepackage{amssymb}
\usepackage{url}
\usepackage[color]{../../tools/coqdoc/coqdoc}

\input{../common/version}
%%%%%%%%%%%%%%%%%%%%%%%%%%%%%%%%
% File title.tex
% Page formatting commands
% Macro \coverpage
%%%%%%%%%%%%%%%%%%%%%%%%%%%%%%%%

%\setlength{\marginparwidth}{0pt}
%\setlength{\oddsidemargin}{0pt}
%\setlength{\evensidemargin}{0pt}
%\setlength{\marginparsep}{0pt}
%\setlength{\topmargin}{0pt}
%\setlength{\textwidth}{16.9cm}
%\setlength{\textheight}{22cm}
%\usepackage{fullpage}

%\newcommand{\printingdate}{\today}
%\newcommand{\isdraft}{\Large\bf\today\\[20pt]}
%\newcommand{\isdraft}{\vspace{20pt}}

\newcommand{\coverpage}[3]{
\thispagestyle{empty}
\begin{center}
\bfseries % for the rest of this page, until \end{center}
\Huge
The Coq Proof Assistant\\[12pt]
#1\\[20pt]
\Large\today\\[20pt]
Version \coqversion\footnote[1]{This research was partly supported by IST working group ``Types''}

\vspace{0pt plus .5fill}
#2
\par\vfill
$\pi r^2$ Project (formerly LogiCal, then TypiCal)

\vspace*{15pt}
\end{center}
\newpage

\thispagestyle{empty}
\hbox{}\vfill % without \hbox \vfill does not work at the top of the page
\begin{flushleft}
%BEGIN LATEX
V\coqversion, \today
\par\vspace{20pt}
%END LATEX
\copyright INRIA 1999-2004 ({\Coq} versions 7.x)

\copyright INRIA 2004-2017 ({\Coq} versions 8.x)

#3
\end{flushleft}
} % end of \coverpage definition


% \newcommand{\shorttitle}[1]{
% \begin{center}
% \begin{huge}
% \begin{bf}
% The Coq Proof Assistant\\
% \vspace{10pt}
%     #1\\
% \end{bf}
% \end{huge}
% \end{center}
% \vspace{5pt}
% }

% Local Variables: 
% mode: LaTeX
% TeX-master: ""
% End: 


%%%%%%%%%%%%%%%%%%%%%%%%%%%%%%%%%%%%%%%%%%
% MACROS FOR THE REFERENCE MANUAL OF COQ %
%%%%%%%%%%%%%%%%%%%%%%%%%%%%%%%%%%%%%%%%%%

% For commentaries (define \com as {} for the release manual)
%\newcommand{\com}[1]{{\it(* #1 *)}}
%\newcommand{\com}[1]{}

%%OPTIONS for HACHA
%\renewcommand{\cuttingunit}{section}


%BEGIN LATEX
\newenvironment{centerframe}%
{\bgroup
\dimen0=\textwidth
\advance\dimen0 by -2\fboxrule
\advance\dimen0 by -2\fboxsep
\setbox0=\hbox\bgroup
\begin{minipage}{\dimen0}%
\begin{center}}%
{\end{center}%
\end{minipage}\egroup
\centerline{\fbox{\box0}}\egroup
}
%END LATEX
%HEVEA \newenvironment{centerframe}{\begin{center}}{\end{center}}

%HEVEA \renewcommand{\vec}[1]{\mathbf{#1}}
%\renewcommand{\ominus}{-} % Hevea does a good job translating these commands
%\renewcommand{\oplus}{+}
%\renewcommand{\otimes}{\times}
%\newcommand{\land}{\wedge}
%\newcommand{\lor}{\vee}
%HEVEA \renewcommand{\k}[1]{#1} % \k{a} is supposed to produce a with a little stroke
%HEVEA \newcommand{\phantom}[1]{\qquad}

%%%%%%%%%%%%%%%%%%%%%%%
% Formatting commands %
%%%%%%%%%%%%%%%%%%%%%%%

\newcommand{\ErrMsg}{\medskip \noindent {\bf Error message: }}
\newcommand{\ErrMsgx}{\medskip \noindent {\bf Error messages: }}
\newcommand{\variant}{\medskip \noindent {\bf Variant: }}
\newcommand{\variants}{\medskip \noindent {\bf Variants: }}
\newcommand{\SeeAlso}{\medskip \noindent {\bf See also: }}
\newcommand{\Rem}{\medskip \noindent {\bf Remark: }}
\newcommand{\Rems}{\medskip \noindent {\bf Remarks: }}
\newcommand{\Example}{\medskip \noindent {\bf Example: }}
\newcommand{\examples}{\medskip \noindent {\bf Examples: }}
\newcommand{\Warning}{\medskip \noindent {\bf Warning: }}
\newcommand{\Warns}{\medskip \noindent {\bf Warnings: }}
\newcounter{ex}
\newcommand{\firstexample}{\setcounter{ex}{1}}
\newcommand{\example}[1]{
\medskip \noindent \textbf{Example \arabic{ex}: }\textit{#1}
\addtocounter{ex}{1}}

\newenvironment{Variant}{\variant\begin{enumerate}}{\end{enumerate}}
\newenvironment{Variants}{\variants\begin{enumerate}}{\end{enumerate}}
\newenvironment{ErrMsgs}{\ErrMsgx\begin{enumerate}}{\end{enumerate}}
\newenvironment{Remarks}{\Rems\begin{enumerate}}{\end{enumerate}}
\newenvironment{Warnings}{\Warns\begin{enumerate}}{\end{enumerate}}
\newenvironment{Examples}{\medskip\noindent{\bf Examples:}
\begin{enumerate}}{\end{enumerate}}

%\newcommand{\bd}{\noindent\bf}
%\newcommand{\sbd}{\vspace{8pt}\noindent\bf}
%\newcommand{\sdoll}[1]{\begin{small}$ #1~ $\end{small}}
%\newcommand{\sdollnb}[1]{\begin{small}$ #1 $\end{small}}
\newcommand{\kw}[1]{\textsf{#1}}
%\newcommand{\spec}[1]{\{\,#1\,\}}

% Building regular expressions
\newcommand{\zeroone}[1]{\mbox{\sl [}#1\mbox{\sl ]}}
%\newcommand{\zeroonemany}[1]{$\{$#1$\}$*}
%\newcommand{\onemany}[1]{$\{$#1$\}$+}
\newcommand{\nelistnosep}[1]{{#1} \mbox{\dots} {#1}}
\newcommand{\nelist}[2]{{#1} {\tt #2} \mbox{\dots} {\tt #2} {#1}}
\newcommand{\sequence}[2]{{\sl [}{#1} {\tt #2} \mbox{\dots} {\tt #2} {#1}{\sl ]}}
\newcommand{\nelistwithoutblank}[2]{#1{\tt #2}\mbox{\dots}{\tt #2}#1}
\newcommand{\sequencewithoutblank}[2]{$[$#1{\tt #2}\mbox{\dots}{\tt #2}#1$]$}

% Used for RefMan-gal
%\newcommand{\ml}[1]{\hbox{\tt{#1}}}
%\newcommand{\op}{\,|\,}

%%%%%%%%%%%%%%%%%%%%%%%%
% Trademarks and so on %
%%%%%%%%%%%%%%%%%%%%%%%%

\newcommand{\Coq}{\textsc{Coq}}
\newcommand{\gallina}{\textsc{Gallina}}
\newcommand{\Gallina}{\textsc{Gallina}}
\newcommand{\CoqIDE}{\textsc{CoqIDE}}
\newcommand{\ocaml}{\textsc{Objective Caml}}
\newcommand{\camlpppp}{\textsc{Camlp4}}
\newcommand{\emacs}{\textsc{GNU Emacs}}
\newcommand{\ProofGeneral}{\textsc{Proof General}}
\newcommand{\CIC}{\textsc{Cic}}
\newcommand{\iCIC}{\textsc{Cic}}
\newcommand{\FW}{\ensuremath{F_{\omega}}}
\newcommand{\Program}{\textsc{Program}}
\newcommand{\Russell}{\textsc{Russell}}
\newcommand{\PVS}{\textsc{PVS}}
%\newcommand{\bn}{{\sf BNF}}

%%%%%%%%%%%%%%%%%%%
% Name of tactics %
%%%%%%%%%%%%%%%%%%%

%\newcommand{\Natural}{\mbox{\tt Natural}}

%%%%%%%%%%%%%%%%%
% \rm\sl series %
%%%%%%%%%%%%%%%%%

\newcommand{\nterm}[1]{\textrm{\textsl{#1}}}

\newcommand{\qstring}{\nterm{string}}

%% New syntax specific entries
\newcommand{\annotation}{\nterm{annotation}}
\newcommand{\assums}{\nterm{assums}} % vernac
\newcommand{\simpleassums}{\nterm{simple\_assums}} % assumptions
\newcommand{\binder}{\nterm{binder}}
\newcommand{\binders}{\nterm{binders}}
\newcommand{\caseitems}{\nterm{match\_items}}
\newcommand{\caseitem}{\nterm{match\_item}}
\newcommand{\eqn}{\nterm{equation}}
\newcommand{\ifitem}{\nterm{dep\_ret\_type}}
\newcommand{\hyplocation}{\nterm{hyp\_location}}
\newcommand{\convclause}{\nterm{conversion\_clause}}
\newcommand{\occclause}{\nterm{occurrence\_clause}}
\newcommand{\occgoalset}{\nterm{goal\_occurrences}}
\newcommand{\atoccurrences}{\nterm{at\_occurrences}}
\newcommand{\occlist}{\nterm{occurrences}}
\newcommand{\params}{\nterm{params}} % vernac
\newcommand{\returntype}{\nterm{return\_type}}
\newcommand{\idparams}{\nterm{ident\_with\_params}}
\newcommand{\statkwd}{\nterm{assertion\_keyword}} % vernac
\newcommand{\termarg}{\nterm{arg}}
\newcommand{\hintdef}{\nterm{hint\_definition}}

\newcommand{\typecstr}{\zeroone{{\tt :}~{\term}}}
\newcommand{\typecstrwithoutblank}{\zeroone{{\tt :}{\term}}}


\newcommand{\Fwterm}{\nterm{Fwterm}}
\newcommand{\Index}{\nterm{index}}
\newcommand{\abbrev}{\nterm{abbreviation}}
\newcommand{\atomictac}{\nterm{atomic\_tactic}}
\newcommand{\bindinglist}{\nterm{bindings\_list}}
\newcommand{\cast}{\nterm{cast}}
\newcommand{\cofixpointbodies}{\nterm{cofix\_bodies}}
\newcommand{\cofixpointbody}{\nterm{cofix\_body}}
\newcommand{\commandtac}{\nterm{tactic\_invocation}}
\newcommand{\constructor}{\nterm{constructor}}
\newcommand{\convtactic}{\nterm{conv\_tactic}}
\newcommand{\assumptionkeyword}{\nterm{assumption\_keyword}}
\newcommand{\assumption}{\nterm{assumption}}
\newcommand{\definition}{\nterm{definition}}
\newcommand{\digit}{\nterm{digit}}
\newcommand{\exteqn}{\nterm{ext\_eqn}}
\newcommand{\field}{\nterm{field}}
\newcommand{\firstletter}{\nterm{first\_letter}}
\newcommand{\fixpg}{\nterm{fix\_pgm}}
\newcommand{\fixpointbodies}{\nterm{fix\_bodies}}
\newcommand{\fixpointbody}{\nterm{fix\_body}}
\newcommand{\fixpoint}{\nterm{fixpoint}}
\newcommand{\flag}{\nterm{flag}}
\newcommand{\form}{\nterm{form}}
\newcommand{\entry}{\nterm{entry}} 
\newcommand{\proditem}{\nterm{prod\_item}}
\newcommand{\taclevel}{\nterm{tactic\_level}}
\newcommand{\tacargtype}{\nterm{tactic\_argument\_type}} 
\newcommand{\scope}{\nterm{scope}} 
\newcommand{\delimkey}{\nterm{key}} 
\newcommand{\optscope}{\nterm{opt\_scope}} 
\newcommand{\declnotation}{\nterm{decl\_notation}} 
\newcommand{\symbolentry}{\nterm{symbol}}
\newcommand{\modifiers}{\nterm{modifiers}}
\newcommand{\localdef}{\nterm{local\_def}}
\newcommand{\localdecls}{\nterm{local\_decls}}
\newcommand{\ident}{\nterm{ident}}
\newcommand{\accessident}{\nterm{access\_ident}}
\newcommand{\possiblybracketedident}{\nterm{possibly\_bracketed\_ident}}
\newcommand{\inductivebody}{\nterm{ind\_body}}
\newcommand{\inductive}{\nterm{inductive}}
\newcommand{\naturalnumber}{\nterm{natural}}
\newcommand{\integer}{\nterm{integer}}
\newcommand{\multpattern}{\nterm{mult\_pattern}}
\newcommand{\mutualcoinductive}{\nterm{mutual\_coinductive}}
\newcommand{\mutualinductive}{\nterm{mutual\_inductive}}
\newcommand{\nestedpattern}{\nterm{nested\_pattern}}
\newcommand{\name}{\nterm{name}}
\newcommand{\num}{\nterm{num}}
\newcommand{\coqindex}{\nterm{index}}
\newcommand{\pattern}{\nterm{pattern}} % pattern for pattern-matching
\newcommand{\orpattern}{\nterm{or\_pattern}}
\newcommand{\intropattern}{\nterm{intro\_pattern}}
\newcommand{\intropatternlist}{\nterm{intro\_pattern\_list}}
\newcommand{\disjconjintropattern}{\nterm{disj\_conj\_intro\_pattern}}
\newcommand{\namingintropattern}{\nterm{naming\_intro\_pattern}}
\newcommand{\termpattern}{\nterm{term\_pattern}} % term with holes
\newcommand{\pat}{\nterm{pat}}
\newcommand{\pgs}{\nterm{pgms}}
\newcommand{\pg}{\nterm{pgm}}
\newcommand{\abullet}{\nterm{bullet}}
%BEGIN LATEX
\newcommand{\proof}{\nterm{proof}}
%END LATEX
%HEVEA \renewcommand{\proof}{\nterm{proof}}
\newcommand{\record}{\nterm{record}}
\newcommand{\recordkw}{\nterm{record\_keyword}}
\newcommand{\rewrule}{\nterm{rewriting\_rule}}
\newcommand{\sentence}{\nterm{sentence}}
\newcommand{\simplepattern}{\nterm{simple\_pattern}}
\newcommand{\sort}{\nterm{sort}}
\newcommand{\specif}{\nterm{specif}}
\newcommand{\assertion}{\nterm{assertion}}
\newcommand{\str}{\nterm{string}}
\newcommand{\subsequentletter}{\nterm{subsequent\_letter}}
\newcommand{\switch}{\nterm{switch}}
\newcommand{\messagetoken}{\nterm{message\_token}}
\newcommand{\tac}{\nterm{tactic}}
\newcommand{\terms}{\nterm{terms}}
\newcommand{\term}{\nterm{term}}
\newcommand{\module}{\nterm{module}}
\newcommand{\modexpr}{\nterm{module\_expression}}
\newcommand{\modtype}{\nterm{module\_type}}
\newcommand{\onemodbinding}{\nterm{module\_binding}}
\newcommand{\modbindings}{\nterm{module\_bindings}}
\newcommand{\qualid}{\nterm{qualid}}
\newcommand{\qualidorstring}{\nterm{qualid\_or\_string}}
\newcommand{\class}{\nterm{class}}
\newcommand{\dirpath}{\nterm{dirpath}}
\newcommand{\typedidents}{\nterm{typed\_idents}}
\newcommand{\type}{\nterm{type}}
\newcommand{\vref}{\nterm{ref}}
\newcommand{\zarithformula}{\nterm{zarith\_formula}}
\newcommand{\zarith}{\nterm{zarith}}
\newcommand{\ltac}{\mbox{${\cal L}_{tac}$}}

%%%%%%%%%%%%%%%%%%%%%%%%%%%%%%%%%%%%%%%%%%%%%%%%%%%%%%%
% \mbox{\sf } series for roman text in maths formulas %
%%%%%%%%%%%%%%%%%%%%%%%%%%%%%%%%%%%%%%%%%%%%%%%%%%%%%%%

\newcommand{\alors}{\mbox{\textsf{then}}}
\newcommand{\alter}{\mbox{\textsf{alter}}}
\newcommand{\bool}{\mbox{\textsf{bool}}}
\newcommand{\conc}{\mbox{\textsf{conc}}}
\newcommand{\cons}{\mbox{\textsf{cons}}}
\newcommand{\consf}{\mbox{\textsf{consf}}}
\newcommand{\emptyf}{\mbox{\textsf{emptyf}}}
\newcommand{\EqSt}{\mbox{\textsf{EqSt}}}
\newcommand{\false}{\mbox{\textsf{false}}}
\newcommand{\filter}{\mbox{\textsf{filter}}}
\newcommand{\forest}{\mbox{\textsf{forest}}}
\newcommand{\from}{\mbox{\textsf{from}}}
\newcommand{\hd}{\mbox{\textsf{hd}}}
\newcommand{\haslength}{\mbox{\textsf{has\_length}}}
\newcommand{\length}{\mbox{\textsf{length}}}
\newcommand{\haslengthA}{\mbox {\textsf{has\_length~A}}}
\newcommand{\List}{\mbox{\textsf{list}}}
\newcommand{\ListA}{\mbox{\textsf{list}}~\ensuremath{A}}
\newcommand{\nilhl}{\mbox{\textsf{nil\_hl}}}
\newcommand{\conshl}{\mbox{\textsf{cons\_hl}}}
\newcommand{\nat}{\mbox{\textsf{nat}}}
\newcommand{\nO}{\mbox{\textsf{O}}}
\newcommand{\nS}{\mbox{\textsf{S}}}
\newcommand{\node}{\mbox{\textsf{node}}}
\newcommand{\Nil}{\mbox{\textsf{nil}}}
\newcommand{\Prop}{\mbox{\textsf{Prop}}}
\newcommand{\Set}{\mbox{\textsf{Set}}}
\newcommand{\si}{\mbox{\textsf{if}}}
\newcommand{\sinon}{\mbox{\textsf{else}}}
\newcommand{\Str}{\mbox{\textsf{Stream}}}
\newcommand{\tl}{\mbox{\textsf{tl}}}
\newcommand{\tree}{\mbox{\textsf{tree}}}
\newcommand{\true}{\mbox{\textsf{true}}}
\newcommand{\Type}{\mbox{\textsf{Type}}}
\newcommand{\unfold}{\mbox{\textsf{unfold}}}
\newcommand{\zeros}{\mbox{\textsf{zeros}}}
\newcommand{\even}{\mbox{\textsf{even}}}
\newcommand{\odd}{\mbox{\textsf{odd}}}
\newcommand{\evenO}{\mbox{\textsf{even\_O}}}
\newcommand{\evenS}{\mbox{\textsf{even\_S}}}
\newcommand{\oddS}{\mbox{\textsf{odd\_S}}}
\newcommand{\Prod}{\mbox{\textsf{prod}}}
\newcommand{\Pair}{\mbox{\textsf{pair}}}

%%%%%%%%%
% Misc. %
%%%%%%%%%
\newcommand{\T}{\texttt{T}}
\newcommand{\U}{\texttt{U}}
\newcommand{\real}{\textsf{Real}}
\newcommand{\Data}{\textit{Data}}
\newcommand{\In} {{\textbf{in }}}
\newcommand{\AND} {{\textbf{and}}}
\newcommand{\If}{{\textbf{if }}}
\newcommand{\Else}{{\textbf{else }}}
\newcommand{\Then} {{\textbf{then }}}
%\newcommand{\Let}{{\textbf{let }}} % looks like this is never used
\newcommand{\Where}{{\textbf{where rec }}}
\newcommand{\Function}{{\textbf{function }}}
\newcommand{\Rec}{{\textbf{rec }}}
%\newcommand{\cn}{\centering}
\newcommand{\nth}{\mbox{$^{\mbox{\scriptsize th}}$}}

%%%%%%%%%%%%%%%%%%%%%%%%%%%%%
% Math commands and symbols %
%%%%%%%%%%%%%%%%%%%%%%%%%%%%%

\newcommand{\la}{\leftarrow}
\newcommand{\ra}{\rightarrow}
\newcommand{\Ra}{\Rightarrow}
\newcommand{\rt}{\Rightarrow}
\newcommand{\lla}{\longleftarrow}
\newcommand{\lra}{\longrightarrow}
\newcommand{\Llra}{\Longleftrightarrow}
\newcommand{\mt}{\mapsto}
\newcommand{\ov}{\overrightarrow}
\newcommand{\wh}{\widehat}
\newcommand{\up}{\uparrow}
\newcommand{\dw}{\downarrow}
\newcommand{\nr}{\nearrow}
\newcommand{\se}{\searrow}
\newcommand{\sw}{\swarrow}
\newcommand{\nw}{\nwarrow}
\newcommand{\mto}{.\;}

\newcommand{\vm}[1]{\vspace{#1em}}
\newcommand{\vx}[1]{\vspace{#1ex}}
\newcommand{\hm}[1]{\hspace{#1em}}
\newcommand{\hx}[1]{\hspace{#1ex}}
\newcommand{\sm}{\mbox{ }}
\newcommand{\mx}{\mbox}

%\newcommand{\nq}{\neq}
%\newcommand{\eq}{\equiv}
\newcommand{\fa}{\forall}
%\newcommand{\ex}{\exists}
\newcommand{\impl}{\rightarrow}
%\newcommand{\Or}{\vee}
%\newcommand{\And}{\wedge}
\newcommand{\ms}{\models}
\newcommand{\bw}{\bigwedge}
\newcommand{\ts}{\times}
\newcommand{\cc}{\circ}
%\newcommand{\es}{\emptyset}
%\newcommand{\bs}{\backslash}
\newcommand{\vd}{\vdash}
%\newcommand{\lan}{{\langle }}
%\newcommand{\ran}{{\rangle }}

%\newcommand{\al}{\alpha}
\newcommand{\bt}{\beta}
%\newcommand{\io}{\iota}
\newcommand{\lb}{\lambda}
%\newcommand{\sg}{\sigma}
%\newcommand{\sa}{\Sigma}
%\newcommand{\om}{\Omega}
%\newcommand{\tu}{\tau}

%%%%%%%%%%%%%%%%%%%%%%%%%
% Custom maths commands %
%%%%%%%%%%%%%%%%%%%%%%%%%

\newcommand{\sumbool}[2]{\{#1\}+\{#2\}}
\newcommand{\myifthenelse}[3]{\kw{if} ~ #1 ~\kw{then} ~ #2 ~ \kw{else} ~ #3}
\newcommand{\fun}[2]{\item[]{\tt {#1}}. \quad\\ #2}
\newcommand{\WF}[2]{\ensuremath{{\cal W\!F}(#1)[#2]}}
\newcommand{\WFTWOLINES}[2]{\ensuremath{{\cal W\!F}\begin{array}{l}(#1)\\\mbox{}[{#2}]\end{array}}}
\newcommand{\WFE}[1]{\WF{E}{#1}}
\newcommand{\WT}[4]{\ensuremath{#1[#2] \vdash #3 : #4}}
\newcommand{\WTE}[3]{\WT{E}{#1}{#2}{#3}}
\newcommand{\WTEG}[2]{\WTE{\Gamma}{#1}{#2}}

\newcommand{\WTM}[3]{\WT{#1}{}{#2}{#3}}
\newcommand{\WFT}[2]{\ensuremath{#1[] \vdash {\cal W\!F}(#2)}}
\newcommand{\WS}[3]{\ensuremath{#1[] \vdash #2 <: #3}}
\newcommand{\WSE}[2]{\WS{E}{#1}{#2}}
\newcommand{\WEV}[3]{\mbox{$#1[] \vdash #2 \lra  #3$}}
\newcommand{\WEVT}[3]{\mbox{$#1[] \vdash #2 \lra$}\\ \mbox{$ #3$}}

\newcommand{\WTRED}[5]{\mbox{$#1[#2] \vdash #3 #4 #5$}}
\newcommand{\WTERED}[4]{\mbox{$E[#1] \vdash #2 #3 #4$}}
\newcommand{\WTELECONV}[3]{\WTERED{#1}{#2}{\leconvert}{#3}}
\newcommand{\WTEGRED}[3]{\WTERED{\Gamma}{#1}{#2}{#3}}
\newcommand{\WTECONV}[3]{\WTERED{#1}{#2}{\convert}{#3}}
\newcommand{\WTEGCONV}[2]{\WTERED{\Gamma}{#1}{\convert}{#2}}
\newcommand{\WTEGLECONV}[2]{\WTERED{\Gamma}{#1}{\leconvert}{#2}}

\newcommand{\lab}[1]{\mathit{labels}(#1)}
\newcommand{\dom}[1]{\mathit{dom}(#1)}

\newcommand{\CI}[2]{\mbox{$\{#1\}^{#2}$}}
\newcommand{\CIP}[3]{\mbox{$\{#1\}_{#2}^{#3}$}}
\newcommand{\CIPV}[1]{\CIP{#1}{I_1.. I_k}{P_1.. P_k}}
\newcommand{\CIPI}[1]{\CIP{#1}{I}{P}}
\newcommand{\CIF}[1]{\mbox{$\{#1\}_{f_1.. f_n}$}}
%BEGIN LATEX
\newcommand{\NInd}[3]{\mbox{{\sf Ind}$(\begin{array}[t]{@{}l}#2:=#3
                                              \,)\end{array}$}}
\newcommand{\Ind}[4]{\mbox{{\sf Ind}$[#2](\begin{array}[t]{@{}l@{}}#3:=#4
                                                 \,)\end{array}$}}
%END LATEX
%HEVEA \newcommand{\NInd}[3]{\mbox{{\sf Ind}$(#2\,:=\,#3)$}}
%HEVEA \newcommand{\Ind}[4]{\mbox{{\sf Ind}$[#2](#3\,:=\,#4)$}}

\newcommand{\Indp}[5]{\mbox{{\sf Ind}$_{#5}(#1)[#2](\begin{array}[t]{@{}l}#3:=#4
                                                 \,)\end{array}$}}
\newcommand{\Indpstr}[6]{\mbox{{\sf Ind}$_{#5}(#1)[#2](\begin{array}[t]{@{}l}#3:=#4
                                                 \,)/{#6}\end{array}$}}
\newcommand{\Def}[4]{\mbox{{\sf Def}$(#1)(#2:=#3:#4)$}}
\newcommand{\Assum}[3]{\mbox{{\sf Assum}$(#1)(#2:#3)$}}
\newcommand{\Match}[3]{\mbox{$<\!#1\!>\!{\mbox{\tt Match}}~#2~{\mbox{\tt with}}~#3~{\mbox{\tt end}}$}}
\newcommand{\Case}[3]{\mbox{$\kw{case}(#2,#1,#3)$}}
\newcommand{\match}[3]{\mbox{$\kw{match}~ #2 ~\kw{with}~ #3 ~\kw{end}$}}
\newcommand{\Fix}[2]{\mbox{\tt Fix}~#1\{#2\}}
\newcommand{\CoFix}[2]{\mbox{\tt CoFix}~#1\{#2\}}
\newcommand{\With}[2]{\mbox{\tt ~with~}}
\newcommand{\letin}[3]{\kw{let}~#1:=#2~\kw{in}~#3}
\newcommand{\subst}[3]{#1\{#2/#3\}}
\newcommand{\substs}[4]{#1\{(#2/#3)_{#4}\}}
\newcommand{\Sort}{\mbox{$\cal S$}}
\newcommand{\convert}{=_{\beta\delta\iota\zeta\eta}}
\newcommand{\leconvert}{\leq_{\beta\delta\iota\zeta\eta}}
\newcommand{\NN}{\mathbb{N}}
\newcommand{\inference}[1]{$${#1}$$}

\newcommand{\compat}[2]{\mbox{$[#1|#2]$}}
\newcommand{\tristackrel}[3]{\mathrel{\mathop{#2}\limits_{#3}^{#1}}}

\newcommand{\Impl}{{\it Impl}}
\newcommand{\elem}{{\it e}}
\newcommand{\Mod}[3]{{\sf Mod}({#1}:{#2}\,\zeroone{:={#3}})}
\newcommand{\ModS}[2]{{\sf Mod}({#1}:{#2})}
\newcommand{\ModType}[2]{{\sf ModType}({#1}:={#2})}
\newcommand{\ModA}[2]{{\sf ModA}({#1}=={#2})}
\newcommand{\functor}[3]{\ensuremath{{\sf Functor}(#1:#2)\;#3}}
\newcommand{\funsig}[3]{\ensuremath{{\sf Funsig}(#1:#2)\;#3}}
\newcommand{\sig}[1]{\ensuremath{{\sf Sig}~#1~{\sf End}}}
\newcommand{\struct}[1]{\ensuremath{{\sf Struct}~#1~{\sf End}}}
\newcommand{\structe}[1]{\ensuremath{
        {\sf Struct}~\elem_1;\ldots;\elem_i;#1;\elem_{i+2};\ldots
        ;\elem_n~{\sf End}}}
\newcommand{\structes}[2]{\ensuremath{
        {\sf Struct}~\elem_1;\ldots;\elem_i;#1;\elem_{i+2}\{#2\}
        ;\ldots;\elem_n\{#2\}~{\sf End}}}
\newcommand{\with}[3]{\ensuremath{#1~{\sf with}~#2 := #3}}

\newcommand{\Spec}{{\it Spec}}
\newcommand{\ModSEq}[3]{{\sf Mod}({#1}:{#2}:={#3})}


%\newbox\tempa
%\newbox\tempb
%\newdimen\tempc
%\newcommand{\mud}[1]{\hfil $\displaystyle{\mathstrut #1}$\hfil}
%\newcommand{\rig}[1]{\hfil $\displaystyle{#1}$}
% \newcommand{\irulehelp}[3]{\setbox\tempa=\hbox{$\displaystyle{\mathstrut #2}$}%
%                         \setbox\tempb=\vbox{\halign{##\cr
%         \mud{#1}\cr
%         \noalign{\vskip\the\lineskip}
%         \noalign{\hrule height 0pt}
%         \rig{\vbox to 0pt{\vss\hbox to 0pt{${\; #3}$\hss}\vss}}\cr
%         \noalign{\hrule}
%         \noalign{\vskip\the\lineskip}
%         \mud{\copy\tempa}\cr}}
%                       \tempc=\wd\tempb
%                       \advance\tempc by \wd\tempa
%                       \divide\tempc by 2 }
% \newcommand{\irule}[3]{{\irulehelp{#1}{#2}{#3}
%                      \hbox to \wd\tempa{\hss \box\tempb \hss}}}

\newcommand{\sverb}[1]{{\tt #1}}
\newcommand{\mover}[2]{{#1\over #2}}
\newcommand{\jd}[2]{#1 \vdash #2}
\newcommand{\mathline}[1]{\[#1\]}
\newcommand{\zrule}[2]{#2: #1}
\newcommand{\orule}[3]{#3: {\mover{#1}{#2}}}
\newcommand{\trule}[4]{#4: \mover{#1  \qquad #2} {#3}}
\newcommand{\thrule}[5]{#5: {\mover{#1  \qquad #2 \qquad #3}{#4}}}



% placement of figures

%BEGIN LATEX
\renewcommand{\topfraction}{.99}
\renewcommand{\bottomfraction}{.99}
\renewcommand{\textfraction}{.01}
\renewcommand{\floatpagefraction}{.9}
%END LATEX

% Macros Bruno pour description de la syntaxe

\def\bfbar{\ensuremath{|\hskip -0.22em{}|\hskip -0.24em{}|}}
\def\TERMbar{\bfbar}
\def\TERMbarbar{\bfbar\bfbar}


%% Macros pour les grammaires
\def\GR#1{\text{\large(}#1\text{\large)}}
\def\NT#1{\langle\textit{#1}\rangle}
\def\NTL#1#2{\langle\textit{#1}\rangle_{#2}}
\def\TERM#1{{\bf\textrm{\bf #1}}}
%\def\TERM#1{{\bf\textsf{#1}}}
\def\KWD#1{\TERM{#1}}
\def\ETERM#1{\TERM{#1}}
\def\CHAR#1{\TERM{#1}}

\def\STAR#1{#1*}
\def\STARGR#1{\GR{#1}*}
\def\PLUS#1{#1+}
\def\PLUSGR#1{\GR{#1}+}
\def\OPT#1{#1?}
\def\OPTGR#1{\GR{#1}?}
%% Tableaux de definition de non-terminaux
\newenvironment{cadre}
        {\begin{array}{|c|}\hline\\}
        {\\\\\hline\end{array}}
\newenvironment{rulebox}
        {$$\begin{cadre}\begin{array}{r@{~}c@{~}l@{}l@{}r}}
        {\end{array}\end{cadre}$$}
\def\DEFNT#1{\NT{#1} & ::= &}
\def\EXTNT#1{\NT{#1} & ::= & ... \\&|&}
\def\RNAME#1{(\textsc{#1})}
\def\SEPDEF{\\\\}
\def\nlsep{\\&|&}
\def\nlcont{\\&&}
\newenvironment{rules}
        {\begin{center}\begin{rulebox}}
        {\end{rulebox}\end{center}}


%%% Local Variables: 
%%% mode: latex
%%% TeX-master: "Reference-Manual"
%%% End: 


\begin{document}
 
\coverpage{The standard library}%
{\ }
{This material is distributed under the terms of the GNU Lesser
General Public License Version 2.1.}

\tableofcontents

\newpage
% \section*{The \Coq\ standard library}

This document is a short description of the \Coq\ standard library.
This library comes with the system as a complement of the core library
(the {\bf Init} library ; see the Reference Manual for a description
of this library). It provides a set of modules directly available
through the \verb!Require! command.

The standard library is composed of the following subdirectories:
\begin{description}
  \item[Logic]  Classical logic and dependent equality
  \item[Bool]   Booleans (basic functions and results)
  \item[Arith]  Basic Peano arithmetic
  \item[ZArith] Basic integer arithmetic
  \item[Reals]  Classical Real Numbers and Analysis
  \item[Lists]  Monomorphic and polymorphic lists (basic functions and
                  results), Streams (infinite sequences defined 
                  with co-inductive types)
  \item[Sets]   Sets (classical, constructive, finite, infinite, power set,
                  etc.)
  \item[Relations] Relations (definitions and basic results).
  \item[Sorting] Sorted list (basic definitions and heapsort
                 correctness). 
  \item[Wellfounded] Well-founded relations (basic results).
  \item[Program] Tactics to deal with dependently-typed programs and
    their proofs.
  \item[Classes] Standard type class instances on relations and
    Coq part of the setoid rewriting tactic.
\end{description}


Each of these subdirectories contains a set of modules, whose
specifications (\gallina{} files) have
been roughly, and automatically, pasted in the following pages. There
is also a version of this document in HTML format on the WWW, which
you can access from the \Coq\ home page at
\texttt{http://coq.inria.fr/library}.

\documentclass[11pt]{report}

\usepackage[mathletters]{ucs}
\usepackage[utf8x]{inputenc}
\usepackage[T1]{fontenc}
\usepackage{fullpage}
\usepackage{amsfonts}
\usepackage{amssymb}
\usepackage{url}
\usepackage[color]{../../tools/coqdoc/coqdoc}

\input{../common/version}
\input{../common/title}
\input{../common/macros}

\begin{document}
 
\coverpage{The standard library}%
{\ }
{This material is distributed under the terms of the GNU Lesser
General Public License Version 2.1.}

\tableofcontents

\newpage
% \section*{The \Coq\ standard library}

This document is a short description of the \Coq\ standard library.
This library comes with the system as a complement of the core library
(the {\bf Init} library ; see the Reference Manual for a description
of this library). It provides a set of modules directly available
through the \verb!Require! command.

The standard library is composed of the following subdirectories:
\begin{description}
  \item[Logic]  Classical logic and dependent equality
  \item[Bool]   Booleans (basic functions and results)
  \item[Arith]  Basic Peano arithmetic
  \item[ZArith] Basic integer arithmetic
  \item[Reals]  Classical Real Numbers and Analysis
  \item[Lists]  Monomorphic and polymorphic lists (basic functions and
                  results), Streams (infinite sequences defined 
                  with co-inductive types)
  \item[Sets]   Sets (classical, constructive, finite, infinite, power set,
                  etc.)
  \item[Relations] Relations (definitions and basic results).
  \item[Sorting] Sorted list (basic definitions and heapsort
                 correctness). 
  \item[Wellfounded] Well-founded relations (basic results).
  \item[Program] Tactics to deal with dependently-typed programs and
    their proofs.
  \item[Classes] Standard type class instances on relations and
    Coq part of the setoid rewriting tactic.
\end{description}


Each of these subdirectories contains a set of modules, whose
specifications (\gallina{} files) have
been roughly, and automatically, pasted in the following pages. There
is also a version of this document in HTML format on the WWW, which
you can access from the \Coq\ home page at
\texttt{http://coq.inria.fr/library}.

\input{Library.coqdoc}

\end{document}



\end{document}



\end{document}



\end{document}

