\chapter{The \gallina{} specification language
\label{Gallina}\index{Gallina}}
\label{BNF-syntax} % Used referred to as a chapter label

This chapter describes \gallina, the specification language of {\Coq}.
It allows developing mathematical theories and to prove specifications
of programs.  The theories are built from axioms, hypotheses,
parameters, lemmas, theorems and definitions of constants, functions,
predicates and sets. The syntax of logical objects involved in
theories is described in Section~\ref{term}. The language of
commands, called {\em The Vernacular} is described in section
\ref{Vernacular}.

In {\Coq}, logical objects are typed to ensure their logical
correctness. The rules implemented by the typing algorithm are described in
Chapter \ref{Cic}.

\subsection*{About the grammars in the manual
\index{BNF metasyntax}}

Grammars are presented in Backus-Naur form (BNF). Terminal symbols are
set in {\tt typewriter font}.  In addition, there are special
notations for regular expressions.

An expression enclosed in square brackets \zeroone{\ldots} means at
most one occurrence of this expression (this corresponds to an
optional component).

The notation ``\nelist{\entry}{sep}'' stands for a non empty
sequence of expressions parsed by {\entry} and
separated by the literal ``{\tt sep}''\footnote{This is similar to the
expression ``{\entry} $\{$ {\tt sep} {\entry} $\}$'' in
standard BNF, or ``{\entry}~{$($} {\tt sep} {\entry} {$)$*}'' in
the syntax of regular expressions.}.

Similarly, the notation ``\nelist{\entry}{}'' stands for a non
empty sequence of expressions parsed by the ``{\entry}'' entry,
without any separator between.

At the end, the notation ``\sequence{\entry}{\tt sep}'' stands for a
possibly empty sequence of expressions parsed by the ``{\entry}'' entry,
separated by the literal ``{\tt sep}''.

\section{Lexical conventions
\label{lexical}\index{Lexical conventions}}

\paragraph{Blanks}
Space, newline and horizontal tabulation are considered as blanks.
Blanks are ignored but they separate tokens.

\paragraph{Comments}

Comments in {\Coq} are enclosed between {\tt (*} and {\tt
  *)}\index{Comments}, and can be nested. They can contain any
character. However, string literals must be correctly closed. Comments
are treated as blanks.

\paragraph{Identifiers and access identifiers}

Identifiers, written {\ident}, are sequences of letters, digits,
\verb!_! and \verb!'!, that do not start with a digit or \verb!'!.
That is, they are recognized by the following lexical class:

\index{ident@\ident}
\begin{center}
\begin{tabular}{rcl} 
{\firstletter} & ::= & {\tt a..z} $\mid$ {\tt A..Z} $\mid$ {\tt \_}
$\mid$ {\tt unicode-letter}  
\\
{\subsequentletter} & ::= & {\tt a..z} $\mid$ {\tt A..Z} $\mid$ {\tt 0..9}
$\mid$ {\tt \_} % $\mid$ {\tt \$}
$\mid$ {\tt '} 
$\mid$ {\tt unicode-letter}  
$\mid$ {\tt unicode-id-part} \\
{\ident} & ::= & {\firstletter} \sequencewithoutblank{\subsequentletter}{}
\end{tabular}
\end{center}
All characters are meaningful. In particular, identifiers are
case-sensitive.  The entry {\tt unicode-letter} non-exhaustively
includes Latin, Greek, Gothic, Cyrillic, Arabic, Hebrew, Georgian,
Hangul, Hiragana and Katakana characters, CJK ideographs, mathematical
letter-like symbols, hyphens, non-breaking space, {\ldots} The entry
{\tt unicode-id-part} non-exhaustively includes symbols for prime
letters and subscripts.

Access identifiers, written {\accessident}, are identifiers prefixed
by \verb!.! (dot) without blank. They are used in the syntax of qualified
identifiers.

\paragraph{Natural numbers and integers}
Numerals are sequences of digits. Integers are numerals optionally preceded by a minus sign.

\index{num@{\num}}
\index{integer@{\integer}}
\begin{center}
\begin{tabular}{r@{\quad::=\quad}l}
{\digit} & {\tt 0..9} \\
{\num} & \nelistwithoutblank{\digit}{} \\
{\integer} & \zeroone{\tt -}{\num} \\
\end{tabular}
\end{center}

\paragraph[Strings]{Strings\label{strings}
\index{string@{\qstring}}}
Strings are delimited by \verb!"! (double quote), and enclose a
sequence of any characters different from \verb!"! or the sequence
\verb!""! to denote the double quote character. In grammars, the
entry for quoted strings is {\qstring}.

\paragraph{Keywords}
The following identifiers are reserved keywords, and cannot be
employed otherwise:
\begin{center}
\begin{tabular}{llllll}
\verb!_!          &
\verb!as!         &
\verb!at!         &
\verb!cofix!      &
\verb!else!       &
\verb!end!        \\
%
\verb!exists!     &
\verb!exists2!    &
\verb!fix!        &
\verb!for!        &
\verb!forall!     &
\verb!fun!        \\
%
\verb!if!         &
\verb!IF!         &
\verb!in!         &
\verb!let!        &
\verb!match!      &
\verb!mod!        \\
%
\verb!Prop!       &
\verb!return!     &
\verb!Set!        &
\verb!then!       &
\verb!Type!       &
\verb!using!      \\
%
\verb!where!      &
\verb!with!       &
\end{tabular}
\end{center}


\paragraph{Special tokens}
The following sequences of characters are special tokens:
\begin{center}
\begin{tabular}{lllllll}
\verb/!/   &
\verb!%!  &
\verb!&!   &
\verb!&&!  &
\verb!(!   &
\verb!()!  &
\verb!)!   \\
%
\verb!*!   &
\verb!+!   &
\verb!++!  &
\verb!,!   &
\verb!-!   &
\verb!->!  &
\verb!.!   \\
%
\verb!.(!  &
\verb!..!  &
\verb!/!   &
\verb!/\!  &
\verb!:!   &
\verb!::!  &
\verb!:<!  \\
%
\verb!:=!  &
\verb!:>!  &
\verb!;!   &
\verb!<!   &
\verb!<-!  &
\verb!<->! &
\verb!<:!  \\
%
\verb!<=!  &
\verb!<>!  &
\verb!=!   &
\verb!=>!  &
\verb!=_D! &
\verb!>!   &
\verb!>->! \\
%
\verb!>=!  &
\verb!?!   &
\verb!?=!  &
\verb!@!   &
\verb![!   &
\verb!\/!  &
\verb!]!   \\
%
\verb!^!   &
\verb!{!   &
\verb!|!   &
\verb!|-!  &
\verb!||!  &
\verb!}!   &
\verb!~!   \\
\end{tabular}
\end{center}

Lexical ambiguities are resolved according to the ``longest match''
rule: when a sequence of non alphanumerical characters can be decomposed
into several different ways, then the first token is the longest
possible one (among all tokens defined at this moment), and so on.

\section{Terms \label{term}\index{Terms}}

\subsection{Syntax of terms}

Figures \ref{term-syntax} and \ref{term-syntax-aux} describe the basic syntax of
the terms of the {\em Calculus of Inductive Constructions} (also
called \CIC). The formal presentation of {\CIC} is given in Chapter
\ref{Cic}. Extensions of this syntax are given in chapter
\ref{Gallina-extension}. How to customize the syntax is described in Chapter
\ref{Addoc-syntax}.

\begin{figure}[htbp]
\begin{centerframe}
\begin{tabular}{lcl@{\quad~}r}  % warning: page width exceeded with \qquad 
{\term} & ::= &
         {\tt forall} {\binders} {\tt ,} {\term}  &(\ref{products})\\
 & $|$ & {\tt fun} {\binders} {\tt =>} {\term} &(\ref{abstractions})\\
 & $|$ & {\tt fix} {\fixpointbodies} &(\ref{fixpoints})\\
 & $|$ & {\tt cofix} {\cofixpointbodies} &(\ref{fixpoints})\\
 & $|$ & {\tt let} {\ident} \zeroone{\binders} {\typecstr} {\tt :=} {\term}
         {\tt in} {\term} &(\ref{let-in})\\
 & $|$ & {\tt let fix} {\fixpointbody} {\tt in} {\term} &(\ref{fixpoints})\\
 & $|$ & {\tt let cofix} {\cofixpointbody}
         {\tt in} {\term} &(\ref{fixpoints})\\
 & $|$ & {\tt let} {\tt (} \sequence{\name}{,} {\tt )} \zeroone{\ifitem}
         {\tt :=} {\term}
         {\tt in} {\term}  &(\ref{caseanalysis}, \ref{Mult-match})\\
 & $|$ & {\tt let '} {\pattern} \zeroone{{\tt in} {\term}} {\tt :=} {\term}
        \zeroone{\returntype} {\tt in} {\term} & (\ref{caseanalysis}, \ref{Mult-match})\\
 & $|$ & {\tt if} {\term} \zeroone{\ifitem} {\tt then} {\term}
         {\tt else} {\term} &(\ref{caseanalysis}, \ref{Mult-match})\\
 & $|$ & {\term} {\tt :} {\term} &(\ref{typecast})\\
 & $|$ & {\term} {\tt <:} {\term} &(\ref{typecast})\\
 & $|$ & {\term} {\tt :>} &(\ref{ProgramSyntax})\\
 & $|$ & {\term} {\tt ->} {\term} &(\ref{products})\\
 & $|$ & {\term} \nelist{\termarg}{}&(\ref{applications})\\
 & $|$ & {\tt @} {\qualid} \sequence{\term}{}
            &(\ref{Implicits-explicitation})\\
 & $|$ & {\term} {\tt \%} {\ident} &(\ref{scopechange})\\
 & $|$ & {\tt match} \nelist{\caseitem}{\tt ,}
                 \zeroone{\returntype} {\tt with} &\\
    &&   ~~~\zeroone{\zeroone{\tt |} \nelist{\eqn}{|}} {\tt end}
    &(\ref{caseanalysis})\\
 & $|$ & {\qualid} &(\ref{qualid})\\
 & $|$ & {\sort} &(\ref{Gallina-sorts})\\
 & $|$ & {\num} &(\ref{numerals})\\
 & $|$ & {\_} &(\ref{hole})\\
 & $|$ & {\tt (} {\term} {\tt )} & \\
 & & &\\
{\termarg} & ::= & {\term} &\\
 & $|$ & {\tt (} {\ident} {\tt :=} {\term} {\tt )}
         &(\ref{Implicits-explicitation})\\
%% & $|$ & {\tt (} {\num} {\tt :=} {\term} {\tt )}
%%         &(\ref{Implicits-explicitation})\\
&&&\\
{\binders} & ::= & \nelist{\binder}{}  \\
&&&\\
{\binder} & ::= &   {\name} & (\ref{Binders}) \\
 & $|$ & {\tt (} \nelist{\name}{} {\tt :} {\term} {\tt )} &\\  
 & $|$ & {\tt (} {\name} {\typecstr} {\tt :=} {\term} {\tt )} &\\
 & $|$ & {\tt '} {\pattern} &\\
& & &\\
{\name} & ::= & {\ident} &\\
 & $|$ & {\tt \_} &\\
&&&\\
{\qualid} & ::= & {\ident} & \\
 & $|$ & {\qualid} {\accessident} &\\
 & & &\\
{\sort} & ::= & {\tt Prop} ~$|$~ {\tt Set} ~$|$~ {\tt Type} &
\end{tabular}
\end{centerframe}
\caption{Syntax of terms}
\label{term-syntax}
\index{term@{\term}}
\index{sort@{\sort}}
\end{figure}



\begin{figure}[htb]
\begin{centerframe}
\begin{tabular}{lcl}
{\fixpointbodies} & ::= &
         {\fixpointbody} \\
 & $|$ & {\fixpointbody} {\tt with} \nelist{\fixpointbody}{{\tt with}}
         {\tt for} {\ident} \\
{\cofixpointbodies} & ::= &
         {\cofixpointbody} \\
 & $|$ & {\cofixpointbody} {\tt with} \nelist{\cofixpointbody}{{\tt with}}
         {\tt for} {\ident} \\
&&\\
{\fixpointbody} & ::= &
    {\ident} {\binders} \zeroone{\annotation} {\typecstr}
    {\tt :=} {\term} \\
{\cofixpointbody} & ::= & {\ident} \zeroone{\binders} {\typecstr} {\tt :=} {\term} \\
  & &\\
{\annotation} & ::= & {\tt \{ struct} {\ident} {\tt \}} \\ 
&&\\
{\caseitem} & ::= & {\term} \zeroone{{\tt as} \name}
     \zeroone{{\tt in} \qualid \sequence{\pattern}{}} \\
&&\\
{\ifitem} & ::= & \zeroone{{\tt as} {\name}} {\returntype} \\
&&\\
{\returntype} & ::= & {\tt return} {\term} \\
&&\\
{\eqn} & ::= & \nelist{\multpattern}{\tt |} {\tt =>} {\term}\\
&&\\
{\multpattern} & ::= & \nelist{\pattern}{\tt ,}\\
&&\\
{\pattern} & ::= & {\qualid} \nelist{\pattern}{}  \\
 & $|$ & {\tt @} {\qualid} \nelist{\pattern}{} \\

 & $|$ & {\pattern} {\tt as} {\ident}             \\
 & $|$ & {\pattern} {\tt \%} {\ident}         \\
 & $|$ & {\qualid}                              \\
 & $|$ & {\tt \_}                                  \\
 & $|$ & {\num}                                 \\
 & $|$ & {\tt (} \nelist{\orpattern}{,} {\tt )}     \\
\\
{\orpattern} & ::= & \nelist{\pattern}{\tt |}\\
\end{tabular}
\end{centerframe}
\caption{Syntax of terms (continued)}
\label{term-syntax-aux}
\end{figure}


%%%%%%%

\subsection{Types}

{\Coq} terms are typed. {\Coq} types are recognized by the same
syntactic class as {\term}. We denote by {\type} the semantic subclass
of types inside the syntactic class {\term}.
\index{type@{\type}}


\subsection{Qualified identifiers and simple identifiers
\label{qualid}
\label{ident}}

{\em Qualified identifiers} ({\qualid}) denote {\em global constants}
(definitions, lemmas, theorems, remarks or facts), {\em global
variables} (parameters or axioms), {\em inductive
types} or {\em constructors of inductive types}.
{\em Simple identifiers} (or shortly {\ident}) are a
syntactic subset of qualified identifiers.  Identifiers may also
denote local {\em variables}, what qualified identifiers do not.

\subsection{Numerals
\label{numerals}}

Numerals have no definite semantics in the calculus. They are mere
notations that can be bound to objects through the notation mechanism
(see Chapter~\ref{Addoc-syntax} for details). Initially, numerals are
bound to Peano's representation of natural numbers
(see~\ref{libnats}).

Note: negative integers are not at the same level as {\num}, for this
would make precedence unnatural.

\subsection{Sorts 
\index{Sorts}
\index{Type@{\Type}}
\index{Set@{\Set}}
\index{Prop@{\Prop}}
\index{Sorts}
\label{Gallina-sorts}}

There are three sorts \Set, \Prop\ and \Type.
\begin{itemize}
\item \Prop\ is the universe of {\em logical propositions}.
The logical propositions themselves are typing the proofs.
We denote propositions by {\form}. This constitutes a semantic
subclass of the syntactic class {\term}.
\index{form@{\form}}
\item \Set\ is is the universe of {\em program
types} or {\em specifications}.
The specifications themselves are typing the programs.
We denote specifications by {\specif}. This constitutes a semantic
subclass of the syntactic class {\term}.
\index{specif@{\specif}}
\item {\Type} is the type of {\Set} and {\Prop}
\end{itemize}
\noindent More on sorts can be found in Section~\ref{Sorts}.

\subsection{Binders
\label{Binders}
\index{binders}}

Various constructions such as {\tt fun}, {\tt forall}, {\tt fix} and
{\tt cofix} {\em bind} variables. A binding is represented by an
identifier. If the binding variable is not used in the expression, the
identifier can be replaced by the symbol {\tt \_}.  When the type of a
bound variable cannot be synthesized by the system, it can be
specified with the notation {\tt (}\,{\ident}\,{\tt :}\,{\type}\,{\tt
)}. There is also a notation for a sequence of binding variables
sharing the same type: {\tt (}\,{\ident$_1$}\ldots{\ident$_n$}\,{\tt
:}\,{\type}\,{\tt )}. A binder can also be any pattern prefixed by a quote,
e.g. {\tt '(x,y)}.

Some constructions allow the binding of a variable to value. This is
called a ``let-binder''. The entry {\binder} of the grammar accepts
either an assumption binder as defined above or a let-binder. 
The notation in the
latter case is {\tt (}\,{\ident}\,{\tt :=}\,{\term}\,{\tt )}. In a
let-binder, only one variable can be introduced at the same
time. It is also possible to give the type of the variable as follows:
{\tt (}\,{\ident}\,{\tt :}\,{\term}\,{\tt :=}\,{\term}\,{\tt )}.

Lists of {\binder} are allowed. In the case of {\tt fun} and {\tt
  forall}, it is intended that at least one binder of the list is an
assumption otherwise {\tt fun} and {\tt forall} gets identical. Moreover,
parentheses can be omitted in the case of a single sequence of
bindings sharing the same type (e.g.: {\tt fun~(x~y~z~:~A)~=>~t} can
be shortened in {\tt fun~x~y~z~:~A~=>~t}).

\subsection{Abstractions
\label{abstractions}
\index{abstractions}}

The expression ``{\tt fun} {\ident} {\tt :} {\type} {\tt =>}~{\term}''
defines the {\em abstraction} of the variable {\ident}, of type
{\type}, over the term {\term}. It denotes a function of the variable
{\ident} that evaluates to the expression {\term} (e.g. {\tt fun x:$A$
=> x} denotes the identity function on type $A$).
% The variable {\ident} is called the {\em parameter} of the function 
% (we sometimes say the {\em formal parameter}).
The keyword {\tt fun} can be followed by several binders as given in
Section~\ref{Binders}. Functions over several variables are
equivalent to an iteration of one-variable functions.  For instance the
expression ``{\tt fun}~{\ident$_{1}$}~{\ldots}~{\ident$_{n}$}~{\tt
:}~\type~{\tt =>}~{\term}'' denotes the same function as ``{\tt
fun}~{\ident$_{1}$}~{\tt :}~\type~{\tt =>}~{\ldots}~{\tt
fun}~{\ident$_{n}$}~{\tt :}~\type~{\tt =>}~{\term}''. If a let-binder
occurs in the list of binders, it is expanded to a let-in definition
(see Section~\ref{let-in}).

\subsection{Products
\label{products}
\index{products}}

The expression ``{\tt forall}~{\ident}~{\tt :}~{\type}{\tt
,}~{\term}'' denotes the {\em product} of the variable {\ident} of
type {\type}, over the term {\term}. As for abstractions, {\tt forall}
is followed by a binder list, and products over several variables are
equivalent to an iteration of one-variable products. 
Note that {\term} is intended to be a type.

If the variable {\ident} occurs in {\term}, the product is called {\em
dependent product}.  The intention behind a dependent product {\tt
forall}~$x$~{\tt :}~{$A$}{\tt ,}~{$B$} is twofold. It denotes either
the universal quantification of the variable $x$ of type $A$ in the
proposition $B$ or the functional dependent product from $A$ to $B$ (a
construction usually written $\Pi_{x:A}.B$ in set theory).

Non dependent product types have a special notation: ``$A$ {\tt ->}
$B$'' stands for ``{\tt forall \_:}$A${\tt ,}~$B$''. The {\em non dependent
product} is used both to denote the propositional implication and
function types.

\subsection{Applications
\label{applications}
\index{applications}}

The expression \term$_0$ \term$_1$ denotes the application of
\term$_0$ to \term$_1$.

The expression {\tt }\term$_0$ \term$_1$ ...  \term$_n${\tt}
denotes the application of the term \term$_0$ to the arguments
\term$_1$ ... then \term$_n$.  It is equivalent to {\tt (} {\ldots}
{\tt (} {\term$_0$} {\term$_1$} {\tt )} {\ldots} {\tt )} {\term$_n$} {\tt }:
associativity is to the left.

The notation {\tt (}\,{\ident}\,{\tt :=}\,{\term}\,{\tt )} for
arguments is used for making explicit the value of implicit arguments
(see Section~\ref{Implicits-explicitation}).

\subsection{Type cast
\label{typecast}
\index{Cast}}

The expression ``{\term}~{\tt :}~{\type}'' is a type cast
expression. It enforces the type of {\term} to be {\type}.

``{\term}~{\tt <:}~{\type}'' locally sets up the virtual machine for checking
that {\term} has type {\type}.

\subsection{Inferable subterms
\label{hole}
\index{\_}}

Expressions often contain redundant pieces of information. Subterms that
can be automatically inferred by {\Coq} can be replaced by the
symbol ``\_'' and {\Coq} will guess the missing piece of information.

\subsection{Let-in definitions
\label{let-in}
\index{Let-in definitions}
\index{let-in}}


{\tt let}~{\ident}~{\tt :=}~{\term$_1$}~{\tt in}~{\term$_2$} denotes
the local binding of \term$_1$ to the variable $\ident$ in
\term$_2$. 
There is a syntactic sugar for let-in definition of functions: {\tt
let} {\ident} {\binder$_1$} {\ldots} {\binder$_n$} {\tt :=} {\term$_1$}
{\tt in} {\term$_2$} stands for {\tt let} {\ident} {\tt := fun}
{\binder$_1$} {\ldots} {\binder$_n$} {\tt =>} {\term$_1$} {\tt in}
{\term$_2$}.

\subsection{Definition by case analysis
\label{caseanalysis}
\index{match@{\tt match\ldots with\ldots end}}}

Objects of inductive types can be destructurated by a case-analysis
construction called {\em pattern-matching} expression.  A
pattern-matching expression is used to analyze the structure of an
inductive objects and to apply specific treatments accordingly.

This paragraph describes the basic form of pattern-matching. See
Section~\ref{Mult-match} and Chapter~\ref{Mult-match-full} for the
description of the general form. The basic form of pattern-matching is
characterized by a single {\caseitem} expression, a {\multpattern}
restricted to a single {\pattern} and {\pattern} restricted to the
form {\qualid} \nelist{\ident}{}.

The expression {\tt match} {\term$_0$} {\returntype} {\tt with}
{\pattern$_1$} {\tt =>} {\term$_1$} {\tt $|$} {\ldots} {\tt $|$}
{\pattern$_n$} {\tt =>} {\term$_n$} {\tt end}, denotes a {\em
pattern-matching} over the term {\term$_0$} (expected to be of an
inductive type $I$).  The terms {\term$_1$}\ldots{\term$_n$} are the
{\em branches} of the pattern-matching expression. Each of
{\pattern$_i$} has a form \qualid~\nelist{\ident}{} where {\qualid}
must denote a constructor. There should be exactly one branch for
every constructor of $I$.

The {\returntype} expresses the type returned by the whole {\tt match}
expression. There are several cases.  In the {\em non dependent} case,
all branches have the same type, and the {\returntype} is the common
type of branches. In this case, {\returntype} can usually be omitted
as it can be inferred from the type of the branches\footnote{Except if
the inductive type is empty in which case there is no equation that can be 
used to infer the return type.}.

In the {\em dependent} case, there are three subcases. In the first
subcase, the type in each branch may depend on the exact value being
matched in the branch. In this case, the whole pattern-matching itself
depends on the term being matched. This dependency of the term being
matched in the return type is expressed with an ``{\tt as {\ident}}''
clause where {\ident} is dependent in the return type.
For instance, in the following example:
\begin{coq_example*}
Inductive bool : Type := true : bool | false : bool.
Inductive eq (A:Type) (x:A) : A -> Prop := eq_refl : eq A x x.
Inductive or (A:Prop) (B:Prop) : Prop :=
| or_introl : A -> or A B
| or_intror : B -> or A B.
Definition bool_case (b:bool) : or (eq bool b true) (eq bool b false)
:= match b as x return or (eq bool x true) (eq bool x false) with
   | true  => or_introl (eq bool true true) (eq bool true false)
                (eq_refl bool true)
   | false => or_intror (eq bool false true) (eq bool false false)
                (eq_refl bool false)
   end.
\end{coq_example*}
the branches have respective types {\tt or (eq bool true true) (eq
bool true false)} and {\tt or (eq bool false true) (eq bool false
false)} while the whole pattern-matching expression has type {\tt or
(eq bool b true) (eq bool b false)}, the identifier {\tt x} being used
to represent the dependency.  Remark that when the term being matched
is a variable, the {\tt as} clause can be omitted and the term being
matched can serve itself as binding name in the return type. For
instance, the following alternative definition is accepted and has the
same meaning as the previous one.
\begin{coq_eval}
Reset bool_case.
\end{coq_eval}
\begin{coq_example*}
Definition bool_case (b:bool) : or (eq bool b true) (eq bool b false)
:= match b return or (eq bool b true) (eq bool b false) with
   | true  => or_introl (eq bool true true) (eq bool true false)
                (eq_refl bool true)
   | false => or_intror (eq bool false true) (eq bool false false)
                (eq_refl bool false)
   end.
\end{coq_example*}

The second subcase is only relevant for annotated inductive types such
as the equality predicate (see Section~\ref{Equality}), the order
predicate on natural numbers % (see Section~\ref{le}) % undefined reference
or the type of
lists of a given length (see Section~\ref{listn}). In this configuration,
the type of each branch can depend on the type dependencies specific
to the branch and the whole pattern-matching expression has a type
determined by the specific dependencies in the type of the term being
matched. This dependency of the return type in the annotations of the
inductive type is expressed using a 
  ``in~I~\_~$\ldots$~\_~\pattern$_1$~$\ldots$~\pattern$_n$'' clause, where
\begin{itemize}
\item $I$ is the inductive type of the term being matched;

\item the {\_}'s are matching the parameters of the inductive type:
the return type is not dependent on them.

\item the \pattern$_i$'s are matching the annotations of the inductive
  type: the return type is dependent on them

\item in the basic case which we describe below, each \pattern$_i$ is a
  name \ident$_i$; see \ref{match-in-patterns} for the general case

\end{itemize}

For instance, in the following example:
\begin{coq_example*}
Definition eq_sym (A:Type) (x y:A) (H:eq A x y) : eq A y x :=
  match H in eq _ _ z return eq A z x with
  | eq_refl _ _ => eq_refl A x
  end.
\end{coq_example*}
the type of the branch has type {\tt eq~A~x~x} because the third
argument of {\tt eq} is {\tt x} in the type of the pattern {\tt
refl\_equal}. On the contrary, the type of the whole pattern-matching
expression has type {\tt eq~A~y~x} because the third argument of {\tt
eq} is {\tt y} in the type of {\tt H}. This dependency of the case
analysis in the third argument of {\tt eq} is expressed by the
identifier {\tt z} in the return type.

Finally, the third subcase is a combination of the first and second
subcase. In particular, it only applies to pattern-matching on terms
in a type with annotations. For this third subcase, both
the clauses {\tt as} and {\tt in} are available.
 
There are specific notations for case analysis on types with one or
two constructors: ``{\tt if {\ldots} then {\ldots} else {\ldots}}''
and ``{\tt let (}\nelist{\ldots}{,}{\tt ) := } {\ldots} {\tt in}
{\ldots}'' (see Sections~\ref{if-then-else} and~\ref{Letin}).

%\SeeAlso Section~\ref{Mult-match} for convenient extensions of pattern-matching.

\subsection{Recursive functions
\label{fixpoints}
\index{fix@{fix \ident$_i$\{\dots\}}}}

The expression ``{\tt fix} \ident$_1$ \binder$_1$ {\tt :} {\type$_1$}
\texttt{:=} \term$_1$ {\tt with} {\ldots} {\tt with} \ident$_n$
\binder$_n$~{\tt :} {\type$_n$} \texttt{:=} \term$_n$ {\tt for}
{\ident$_i$}'' denotes the $i$\nth component of a block of functions
defined by mutual well-founded recursion. It is the local counterpart
of the {\tt Fixpoint} command. See Section~\ref{Fixpoint} for more
details. When $n=1$, the ``{\tt for}~{\ident$_i$}'' clause is omitted.

The expression ``{\tt cofix} \ident$_1$~\binder$_1$ {\tt :}
{\type$_1$} {\tt with} {\ldots} {\tt with} \ident$_n$ \binder$_n$ {\tt
:} {\type$_n$}~{\tt for} {\ident$_i$}'' denotes the $i$\nth component of
a block of terms defined by a mutual guarded co-recursion. It is the
local counterpart of the {\tt CoFixpoint} command. See
Section~\ref{CoFixpoint} for more details. When $n=1$, the ``{\tt
for}~{\ident$_i$}'' clause is omitted.

The association of a single fixpoint and a local
definition have a special syntax: ``{\tt let fix}~$f$~{\ldots}~{\tt
  :=}~{\ldots}~{\tt in}~{\ldots}'' stands for ``{\tt let}~$f$~{\tt :=
  fix}~$f$~\ldots~{\tt :=}~{\ldots}~{\tt in}~{\ldots}''. The same
  applies for co-fixpoints.


\section{The Vernacular
\label{Vernacular}}

\begin{figure}[tbp]
\begin{centerframe}
\begin{tabular}{lcl}
{\sentence} & ::= & {\assumption} \\
            & $|$ & {\definition} \\
            & $|$ & {\inductive} \\
            & $|$ & {\fixpoint} \\
            & $|$ & {\assertion} {\proof} \\
&&\\
%% Assumptions
{\assumption} & ::= & {\assumptionkeyword} {\assums} {\tt .} \\
&&\\
{\assumptionkeyword} & $\!\!$ ::= & {\tt Axiom} $|$ {\tt Conjecture} \\
  & $|$  & {\tt Parameter} $|$  {\tt Parameters} \\
  & $|$  & {\tt Variable}  $|$ {\tt Variables}  \\
  & $|$  & {\tt Hypothesis}  $|$ {\tt Hypotheses}\\
&&\\
{\assums} & ::= & \nelist{\ident}{} {\tt :} {\term} \\
          & $|$ & \nelist{{\tt (} \nelist{\ident}{} {\tt :} {\term} {\tt )}}{} \\
&&\\
%% Definitions
{\definition} & ::= & 
         \zeroone{\tt Local} {\tt Definition} {\ident} \zeroone{\binders} {\typecstr} {\tt :=} {\term} {\tt .} \\
 & $|$ & {\tt Let} {\ident} \zeroone{\binders} {\typecstr} {\tt :=} {\term} {\tt .} \\
&&\\
%% Inductives
{\inductive} & ::= & 
           {\tt Inductive} \nelist{\inductivebody}{with} {\tt .} \\
 & $|$ & {\tt CoInductive} \nelist{\inductivebody}{with} {\tt .} \\
           & & \\
{\inductivebody} & ::= & 
  {\ident} \zeroone{\binders} {\tt :} {\term} {\tt :=} \\
   && ~~\zeroone{\zeroone{\tt |} \nelist{$\!${\ident}$\!$ \zeroone{\binders} {\typecstrwithoutblank}}{|}} \\
           & & \\  %% TODO: where ...
%% Fixpoints
{\fixpoint} & ::= & {\tt Fixpoint} \nelist{\fixpointbody}{with} {\tt .} \\
       & $|$ &  {\tt CoFixpoint} \nelist{\cofixpointbody}{with} {\tt .} \\
&&\\
%% Lemmas & proofs
{\assertion} & ::= &
  {\statkwd} {\ident} \zeroone{\binders} {\tt :} {\term} {\tt .} \\
&&\\
  {\statkwd} & ::= & {\tt Theorem} $|$ {\tt Lemma} \\
   & $|$ & {\tt Remark} $|$ {\tt Fact}\\
   & $|$ & {\tt Corollary} $|$ {\tt Proposition} \\
   & $|$ & {\tt Definition} $|$ {\tt Example} \\\\
&&\\
{\proof} & ::= & {\tt Proof} {\tt .} {\dots} {\tt Qed} {\tt .}\\
   & $|$ & {\tt Proof} {\tt .} {\dots} {\tt Defined} {\tt .}\\
   & $|$ & {\tt Proof} {\tt .} {\dots} {\tt Admitted} {\tt .}\\
\end{tabular}
\end{centerframe}
\caption{Syntax of sentences}
\label{sentences-syntax}
\end{figure}

Figure \ref{sentences-syntax} describes {\em The Vernacular} which is the
language of commands of \gallina.  A sentence of the vernacular
language, like in many natural languages, begins with a capital letter
and ends with a dot.

The different kinds of command are described hereafter. They all suppose
that the terms occurring in the sentences are well-typed.

%%
%% Axioms and Parameters
%%
\subsection{Assumptions
\index{Declarations}
\label{Declarations}}

Assumptions extend the environment\index{Environment} with axioms,
parameters, hypotheses or variables. An assumption binds an {\ident}
to a {\type}. It is accepted by {\Coq} if and only if this {\type} is
a correct type in the environment preexisting the declaration and if
{\ident} was not previously defined in the same module. This {\type}
is considered to be the type (or specification, or statement) assumed
by {\ident} and we say that {\ident} has type {\type}.

\subsubsection{{\tt Axiom {\ident} :{\term} .}
\comindex{Axiom}
\label{Axiom}}

This command links {\term} to the name {\ident} as its specification
in the global context. The fact asserted by {\term} is thus assumed as
a postulate.

\begin{ErrMsgs}
\item \errindex{{\ident} already exists}
\end{ErrMsgs}

\begin{Variants} 
\item \comindex{Parameter}\comindex{Parameters}
  {\tt Parameter {\ident} :{\term}.} \\
  Is equivalent to {\tt Axiom {\ident} : {\term}}

\item {\tt Parameter {\ident$_1$} {\ldots} {\ident$_n$} {\tt :}{\term}.}\\
  Adds $n$ parameters with specification {\term}

\item
 {\tt Parameter\,%
(\,{\ident$_{1,1}$} {\ldots} {\ident$_{1,k_1}$}\,{\tt :}\,{\term$_1$} {\tt )}\;%
\ldots\;{\tt (}\,{\ident$_{n,1}$}{\ldots}{\ident$_{n,k_n}$}\,{\tt :}\,%
{\term$_n$} {\tt )}.}\\ 
  Adds $n$ blocks of parameters with different specifications.

\item {\tt Local Axiom {\ident} : {\term}.}\\
\comindex{Local Axiom}
  Such axioms are never made accessible through their unqualified name by
  {\tt Import} and its variants (see \ref{Import}). You have to explicitly
  give their fully qualified name to refer to them.

\item \comindex{Conjecture}
  {\tt Conjecture {\ident} :{\term}.}\\
  Is equivalent to {\tt Axiom {\ident} : {\term}}.
\end{Variants}

\noindent {\bf Remark: } It is possible to replace {\tt Parameter} by
{\tt Parameters}.


\subsubsection{{\tt Variable {\ident} :{\term}}.
\comindex{Variable}
\comindex{Variables}
\label{Variable}}

This command links {\term} to the name {\ident} in the context of the
current section (see Section~\ref{Section} for a description of the section
mechanism). When the current section is closed, name {\ident} will be
unknown and every object using this variable will be explicitly
parametrized (the variable is {\em discharged}). Using the {\tt
Variable} command out of any section is equivalent to using {\tt
Local Parameter}.

\begin{ErrMsgs}
\item \errindex{{\ident} already exists}
\end{ErrMsgs}

\begin{Variants}
\item {\tt Variable {\ident$_1$} {\ldots} {\ident$_n$} {\tt :}{\term}.}\\
  Links {\term} to names {\ident$_1$} {\ldots} {\ident$_n$}.
\item
 {\tt Variable\,%
(\,{\ident$_{1,1}$} {\ldots} {\ident$_{1,k_1}$}\,{\tt :}\,{\term$_1$} {\tt )}\;%
\ldots\;{\tt (}\,{\ident$_{n,1}$} {\ldots}{\ident$_{n,k_n}$}\,{\tt :}\,%
{\term$_n$} {\tt )}.}\\ 
  Adds $n$ blocks of variables with different specifications.
\item \comindex{Hypothesis}
      \comindex{Hypotheses}
 {\tt Hypothesis {\ident} {\tt :}{\term}.} \\
  \texttt{Hypothesis} is a synonymous of \texttt{Variable}
\end{Variants}

\noindent {\bf Remark: } It is possible to replace {\tt Variable} by
{\tt Variables} and {\tt Hypothesis} by {\tt Hypotheses}.

It is advised to use the keywords \verb:Axiom: and \verb:Hypothesis:
for logical postulates (i.e. when the assertion {\term} is of sort
\verb:Prop:), and to use the keywords \verb:Parameter: and
\verb:Variable: in other cases (corresponding to the declaration of an
abstract mathematical entity).

%%
%% Definitions
%%
\subsection{Definitions
\index{Definitions}
\label{Basic-definitions}}

Definitions extend the environment\index{Environment} with
associations of names to terms. A definition can be seen as a way to
give a meaning to a name or as a way to abbreviate a term.  In any
case, the name can later be replaced at any time by its definition.

The operation of unfolding a name into its definition is called
$\delta$-conversion\index{delta-reduction@$\delta$-reduction} (see
Section~\ref{delta}).  A definition is accepted by the system if and
only if the defined term is well-typed in the current context of the
definition and if the name is not already used. The name defined by
the definition is called a {\em constant}\index{Constant} and the term
it refers to is its {\em body}.  A definition has a type which is the
type of its body.

A formal presentation of constants and environments is given in
Section~\ref{Typed-terms}.

\subsubsection{\tt Definition {\ident} := {\term}.
\label{Definition}
\comindex{Definition}}

This command binds {\term} to the name {\ident} in the
environment, provided that {\term} is well-typed.

\begin{ErrMsgs}
\item \errindex{{\ident} already exists}
\end{ErrMsgs}

\begin{Variants}
\item {\tt Definition} {\ident} {\tt :} {\term$_1$} {\tt :=} {\term$_2$}{\tt .}\\
  It checks that the type of {\term$_2$} is definitionally equal to
  {\term$_1$}, and registers {\ident} as being of type {\term$_1$},
  and bound to value {\term$_2$}.
\item {\tt Definition} {\ident} {\binder$_1$} {\ldots} {\binder$_n$}
       {\tt :} \term$_1$ {\tt :=} {\term$_2$}{\tt .}\\
  This is equivalent to \\
   {\tt Definition} {\ident} {\tt : forall}%
       {\binder$_1$} {\ldots} {\binder$_n$}{\tt ,}\,\term$_1$\,{\tt :=}\,%
       {\tt fun}\,{\binder$_1$} {\ldots} {\binder$_n$}\,{\tt =>}\,{\term$_2$}\,%
       {\tt .}

\item {\tt Local Definition {\ident} := {\term}.}\\
\comindex{Local Definition}
  Such definitions are never made accessible through their unqualified name by
  {\tt Import} and its variants (see \ref{Import}). You have to explicitly
  give their fully qualified name to refer to them.
\item {\tt Example {\ident} := {\term}.}\\
{\tt Example} {\ident} {\tt :} {\term$_1$} {\tt :=} {\term$_2$}{\tt .}\\
{\tt Example} {\ident} {\binder$_1$} {\ldots} {\binder$_n$}
       {\tt :} {\term$_1$} {\tt :=} {\term$_2$}{\tt .}\\
\comindex{Example}
These are synonyms of the {\tt Definition} forms.
\end{Variants}

\begin{ErrMsgs}
\item \errindex{The term {\term} has type {\type} while it is expected to have type {\type}}
\end{ErrMsgs}

\SeeAlso Sections \ref{Opaque}, \ref{Transparent}, \ref{unfold}.

\subsubsection{\tt Let {\ident} := {\term}.
\comindex{Let}}

This command binds the value {\term} to the name {\ident} in the
environment of the current section. The name {\ident} disappears
when the current section is eventually closed, and, all
persistent objects (such as theorems) defined within the
section and depending on {\ident} are prefixed by the let-in definition
{\tt let {\ident} := {\term} in}. Using the {\tt
Let} command out of any section is equivalent to using {\tt
Local Definition}.

\begin{ErrMsgs}
\item \errindex{{\ident} already exists}
\end{ErrMsgs}

\begin{Variants}
\item {\tt Let {\ident} : {\term$_1$} := {\term$_2$}.}
\item {\tt Let Fixpoint {\ident} \nelist{\fixpointbody}{with} {\tt .}.}
\item {\tt Let CoFixpoint {\ident} \nelist{\cofixpointbody}{with} {\tt .}.}
\end{Variants}

\SeeAlso Sections \ref{Section} (section mechanism), \ref{Opaque},
\ref{Transparent} (opaque/transparent constants), \ref{unfold} (tactic
    {\tt unfold}).

%%
%% Inductive Types
%%
\subsection{Inductive definitions
\index{Inductive definitions}
\label{gal-Inductive-Definitions}
\comindex{Inductive}
\label{Inductive}
\comindex{Variant}
\label{Variant}}

We gradually explain simple inductive types, simple
annotated inductive types, simple parametric inductive types, 
mutually inductive types. We explain also co-inductive types.

\subsubsection{Simple inductive types}

The definition of a simple inductive type has the following form:

\medskip
\begin{tabular}{l}
{\tt Inductive} {\ident} {\tt :} {\sort} {\tt :=} \\
\begin{tabular}{clcl}
         & {\ident$_1$} & {\tt :} & {\type$_1$} \\
 {\tt |} & {\ldots} && \\
 {\tt |} & {\ident$_n$} & {\tt :} & {\type$_n$} \\
\end{tabular}
\end{tabular}
\medskip

The name {\ident} is the name of the inductively defined type and
{\sort} is the universes where it lives.
The names {\ident$_1$}, {\ldots}, {\ident$_n$}
are the names of its constructors and {\type$_1$}, {\ldots},
{\type$_n$} their respective types. The types of the constructors have
to satisfy a {\em positivity condition} (see Section~\ref{Positivity})
for {\ident}.  This condition ensures the soundness of the inductive
definition.  If this is the case, the names {\ident},
{\ident$_1$}, {\ldots}, {\ident$_n$} are added to the environment with
their respective types.  Accordingly to the universe where
the inductive type lives ({\it e.g.} its type {\sort}), {\Coq} provides a
number of destructors for {\ident}.  Destructors are named
{\ident}{\tt\_ind}, {\ident}{\tt \_rec} or {\ident}{\tt \_rect} which
respectively correspond to elimination principles on {\tt Prop}, {\tt
Set} and {\tt Type}.  The type of the destructors expresses structural
induction/recursion principles over objects of {\ident}. We give below
two examples of the use of the {\tt Inductive} definitions.

The set of natural numbers is defined as:
\begin{coq_example}
Inductive nat : Set :=
  | O : nat
  | S : nat -> nat.
\end{coq_example}

The type {\tt nat} is defined as the least \verb:Set: containing {\tt
  O} and closed by the {\tt S} constructor. The names {\tt nat},
{\tt O} and {\tt S} are added to the environment.

Now let us have a look at the elimination principles. They are three
of them:
{\tt nat\_ind}, {\tt nat\_rec} and {\tt nat\_rect}.  The type of {\tt
  nat\_ind} is:
\begin{coq_example}
Check nat_ind.
\end{coq_example}

This is the well known structural induction principle over natural
numbers, i.e. the second-order form of Peano's induction principle.
It allows proving some universal property of natural numbers ({\tt
forall n:nat, P n}) by induction on {\tt n}.

The types of {\tt nat\_rec} and {\tt nat\_rect} are similar, except
that they pertain to {\tt (P:nat->Set)} and {\tt (P:nat->Type)}
respectively . They correspond to primitive induction principles
(allowing dependent types) respectively over sorts \verb:Set: and
\verb:Type:. The constant {\ident}{\tt \_ind} is always provided,
whereas {\ident}{\tt \_rec} and {\ident}{\tt \_rect} can be impossible
to derive (for example, when {\ident} is a proposition).

\begin{coq_eval}
Reset Initial.
\end{coq_eval}
\begin{Variants}
\item 
\begin{coq_example*}
Inductive nat : Set := O | S (_:nat).
\end{coq_example*}
In the case where inductive types have no annotations (next section
gives an example of such annotations), 
%the positivity condition implies that 
a constructor can be defined by only giving the type of
its arguments.
\end{Variants}

\subsubsection{Simple annotated inductive types}

In an annotated inductive types, the universe where the inductive
type is defined is no longer a simple sort, but what is called an
arity, which is a type whose conclusion is a sort.

As an example of annotated inductive types, let us define the
$even$ predicate:

\begin{coq_example}
Inductive even : nat -> Prop :=
  | even_0 : even O
  | even_SS : forall n:nat, even n -> even (S (S n)).
\end{coq_example}

The type {\tt nat->Prop} means that {\tt even} is a unary predicate
(inductively defined) over natural numbers.  The type of its two
constructors are the defining clauses of the predicate {\tt even}. The
type of {\tt even\_ind} is:

\begin{coq_example}
Check even_ind.
\end{coq_example}

From a mathematical point of view it asserts that the natural numbers
satisfying the predicate {\tt even} are exactly in the smallest set of
naturals satisfying the clauses {\tt even\_0} or {\tt even\_SS}. This
is why, when we want to prove any predicate {\tt P} over elements of
{\tt even}, it is enough to prove it for {\tt O} and to prove that if
any natural number {\tt n} satisfies {\tt P} its double successor {\tt
  (S (S n))} satisfies also {\tt P}. This is indeed analogous to the
structural induction principle we got for {\tt nat}.

\begin{ErrMsgs}
\item \errindex{Non strictly positive occurrence of {\ident} in {\type}}
\item \errindex{The conclusion of {\type} is not valid; it must be
built from {\ident}}
\end{ErrMsgs}

\subsubsection{Parametrized inductive types}
In the previous example, each constructor introduces a
different instance of the predicate {\tt even}. In some cases, 
all the constructors introduces the same generic instance of the
inductive definition, in which case, instead of an annotation, we use
a context of parameters which are binders shared by all the
constructors of the definition.

% Inductive types may be parameterized. Parameters differ from inductive
% type annotations in the fact that recursive invokations of inductive
% types must always be done with the same values of parameters as its
% specification.

The general scheme is:
\begin{center}
{\tt Inductive} {\ident} {\binder$_1$}\ldots{\binder$_k$} : {\term} :=
    {\ident$_1$}: {\term$_1$} | {\ldots} | {\ident$_n$}: \term$_n$
{\tt .}
\end{center}
Parameters differ from inductive type annotations in the fact that the
conclusion of each type of constructor {\term$_i$} invoke the inductive
type with the same values of parameters as its specification.



A typical example is the definition of polymorphic lists:
\begin{coq_example*}
Inductive list (A:Set) : Set :=
  | nil : list A
  | cons : A -> list A -> list A.
\end{coq_example*}

Note that in the type of {\tt nil} and {\tt cons}, we write {\tt
  (list A)} and not just {\tt list}.\\ The constructors {\tt nil} and
{\tt cons} will have respectively types:

\begin{coq_example}
Check nil.
Check cons.
\end{coq_example}

Types of destructors are also quantified with {\tt (A:Set)}.

\begin{coq_eval}
Reset Initial.
\end{coq_eval}
\begin{Variants}
\item
\begin{coq_example*}
Inductive list (A:Set) : Set := nil | cons (_:A) (_:list A).
\end{coq_example*}
This is an alternative definition of lists where we specify the
arguments of the constructors rather than their full type.
\item
\begin{coq_example*}
Variant sum (A B:Set) : Set := left : A -> sum A B | right : B -> sum A B.
\end{coq_example*}
The {\tt Variant} keyword is identical to the {\tt Inductive} keyword,
except that it disallows recursive definition of types (in particular
lists cannot be defined with the {\tt Variant} keyword). No induction
scheme is generated for this variant, unless the option
{\tt Nonrecursive Elimination Schemes} is set
(see~\ref{set-nonrecursive-elimination-schemes}).
\end{Variants}

\begin{ErrMsgs}
\item \errindex{The {\num}th argument of {\ident} must be {\ident'} in
{\type}}
\end{ErrMsgs}

\paragraph{New from \Coq{} V8.1} The condition on parameters for
inductive definitions has been relaxed since \Coq{} V8.1. It is now
possible in the type of a constructor, to invoke recursively the
inductive definition on an argument which is not the parameter itself.

One can define~:
\begin{coq_example}
Inductive list2 (A:Set) : Set :=
  | nil2 : list2 A
  | cons2 : A -> list2 (A*A) -> list2 A.
\end{coq_example}
\begin{coq_eval}
Reset list2.
\end{coq_eval}
that can also be written by specifying only the type of the arguments:
\begin{coq_example*}
Inductive list2 (A:Set) : Set := nil2 | cons2 (_:A) (_:list2 (A*A)).
\end{coq_example*}
But the following definition will give an error:
\begin{coq_example}
Fail Inductive listw (A:Set) : Set :=
  | nilw : listw (A*A)
  | consw : A -> listw (A*A) -> listw (A*A).
\end{coq_example}
Because the conclusion of the type of constructors should be {\tt
  listw A} in both cases. 

A parametrized inductive definition can be defined using
annotations instead of parameters but it will sometimes give a
different (bigger) sort for the inductive definition and will produce
a less convenient rule for case elimination.

\SeeAlso Sections~\ref{Cic-inductive-definitions} and~\ref{Tac-induction}.


\subsubsection{Mutually defined inductive types
\comindex{Inductive}
\label{Mutual-Inductive}}

The definition of a block of mutually inductive types has the form:

\medskip
{\tt 
\begin{tabular}{l}
Inductive {\ident$_1$} : {\type$_1$} :=  \\
\begin{tabular}{clcl}
   & {\ident$_1^1$}     &:& {\type$_1^1$} \\
 | & {\ldots} && \\
 | & {\ident$_{n_1}^1$} &:& {\type$_{n_1}^1$}
\end{tabular}  \\
with\\
~{\ldots} \\
with {\ident$_m$} : {\type$_m$} := \\
\begin{tabular}{clcl}
   & {\ident$_1^m$}     &:& {\type$_1^m$} \\
 | & {\ldots} \\
 | & {\ident$_{n_m}^m$} &:& {\type$_{n_m}^m$}.
\end{tabular}
\end{tabular}
}
\medskip

\noindent It has the same semantics as the above {\tt Inductive}
definition for each \ident$_1$, {\ldots}, \ident$_m$. All names
\ident$_1$, {\ldots}, \ident$_m$ and \ident$_1^1$, \dots,
\ident$_{n_m}^m$ are simultaneously added to the environment. Then
well-typing of constructors can be checked. Each one of the
\ident$_1$, {\ldots}, \ident$_m$ can be used on its own.

It is also possible to parametrize these inductive definitions.
However, parameters correspond to a local
context in which the whole set of inductive declarations is done.  For
this reason, the parameters must be strictly the same for each
inductive types The extended syntax is:

\medskip
\begin{tabular}{l}
{\tt Inductive} {\ident$_1$} {\params} {\tt :} {\type$_1$} {\tt :=}  \\
\begin{tabular}{clcl}
         & {\ident$_1^1$}    &{\tt :}& {\type$_1^1$} \\
 {\tt |} & {\ldots} && \\
 {\tt |} & {\ident$_{n_1}^1$} &{\tt :}& {\type$_{n_1}^1$}
\end{tabular}  \\
{\tt with}\\
~{\ldots} \\
{\tt with} {\ident$_m$} {\params} {\tt :} {\type$_m$} {\tt :=} \\
\begin{tabular}{clcl}
         & {\ident$_1^m$}    &{\tt :}& {\type$_1^m$} \\
 {\tt |} & {\ldots} \\
 {\tt |} & {\ident$_{n_m}^m$} &{\tt :}& {\type$_{n_m}^m$}.
\end{tabular}
\end{tabular}
\medskip

\Example
The typical example of a mutual inductive data type is the one for
trees and forests. We assume given two types $A$ and $B$ as variables.
It can be declared the following way.

\begin{coq_eval}
Reset Initial.
\end{coq_eval}
\begin{coq_example*}
Variables A B : Set.
Inductive tree : Set :=
    node : A -> forest -> tree
with forest : Set :=
  | leaf : B -> forest
  | cons : tree -> forest -> forest.
\end{coq_example*}

This declaration generates automatically six induction
principles. They are respectively 
called {\tt tree\_rec}, {\tt tree\_ind}, {\tt
  tree\_rect}, {\tt forest\_rec}, {\tt forest\_ind}, {\tt
  forest\_rect}.  These ones are not the most general ones but are
just the induction principles corresponding to each inductive part
seen as a single inductive definition.

To illustrate this point on our example, we give the types of {\tt
  tree\_rec} and {\tt forest\_rec}.

\begin{coq_example}
Check tree_rec.
Check forest_rec.
\end{coq_example}

Assume we want to parametrize our mutual inductive definitions with
the two type variables $A$ and $B$, the declaration should be done the
following way:

\begin{coq_eval}
Reset tree.
\end{coq_eval}
\begin{coq_example*}
Inductive tree (A B:Set) : Set :=
    node : A -> forest A B -> tree A B
with forest (A B:Set) : Set :=
  | leaf : B -> forest A B
  | cons : tree A B -> forest A B -> forest A B.
\end{coq_example*}

Assume we define an inductive definition inside a section.  When the
section is closed, the variables declared in the section and occurring
free in the declaration are added as parameters to the inductive
definition. 

\SeeAlso Section~\ref{Section}.

\subsubsection{Co-inductive types
\label{CoInductiveTypes}
\comindex{CoInductive}}

The objects of an inductive type are well-founded with respect to the
constructors of the type. In other words, such objects contain only a
{\it finite} number of constructors. Co-inductive types arise from
relaxing this condition, and admitting types whose objects contain an
infinity of constructors. Infinite objects are introduced by a
non-ending (but effective) process of construction, defined in terms
of the constructors of the type.

An example of a co-inductive type is the type of infinite sequences of
natural numbers, usually called streams. It can be introduced in \Coq\
using the \texttt{CoInductive} command:
\begin{coq_example}
CoInductive Stream : Set :=
    Seq : nat -> Stream -> Stream.
\end{coq_example}

The syntax of this command is the same as the command \texttt{Inductive}
(see Section~\ref{gal-Inductive-Definitions}). Notice that no
principle of induction is derived from the definition of a
co-inductive type, since such principles only make sense for inductive
ones. For co-inductive ones, the only elimination principle is case
analysis. For example, the usual destructors on streams
\texttt{hd:Stream->nat} and \texttt{tl:Str->Str} can be defined as
follows:
\begin{coq_example}
Definition hd (x:Stream) := let (a,s) := x in a.
Definition tl (x:Stream) := let (a,s) := x in s.
\end{coq_example}

Definition of co-inductive predicates and blocks of mutually
co-inductive definitions are also allowed. An example of a
co-inductive predicate is the extensional equality on streams:

\begin{coq_example}
CoInductive EqSt : Stream -> Stream -> Prop :=
    eqst :
      forall s1 s2:Stream,
        hd s1 = hd s2 -> EqSt (tl s1) (tl s2) -> EqSt s1 s2.
\end{coq_example}

In order to prove the extensionally equality of two streams $s_1$ and
$s_2$ we have to construct an infinite proof of equality, that is,
an infinite object of type $(\texttt{EqSt}\;s_1\;s_2)$. We will see
how to introduce infinite objects in Section~\ref{CoFixpoint}.

%%
%% (Co-)Fixpoints
%%
\subsection{Definition of recursive functions}

\subsubsection{Definition of functions by recursion over inductive objects}

This section describes the primitive form of definition by recursion
over inductive objects. See Section~\ref{Function} for more advanced
constructions. The command:
\begin{center}
  \texttt{Fixpoint {\ident} {\params} {\tt \{struct}
  \ident$_0$ {\tt \}} : type$_0$ := \term$_0$ 
  \comindex{Fixpoint}\label{Fixpoint}}
\end{center}
allows defining functions by pattern-matching over inductive objects 
using a fixed point construction.
The meaning of this declaration is to define {\it ident} a recursive
function with arguments specified by the binders in {\params} such
that {\it ident} applied to arguments corresponding to these binders
has type \type$_0$, and is equivalent to the expression \term$_0$. The
type of the {\ident} is consequently {\tt forall {\params} {\tt,}
  \type$_0$} and the value is equivalent to {\tt fun {\params} {\tt
    =>} \term$_0$}.

To be accepted, a {\tt Fixpoint} definition has to satisfy some
syntactical constraints on a special argument called the decreasing
argument. They are needed to ensure that the {\tt Fixpoint} definition
always terminates. The point of the {\tt \{struct \ident {\tt \}}}
annotation is to let the user tell the system which argument decreases
along the recursive calls. For instance, one can define the addition 
function as :

\begin{coq_example}
Fixpoint add (n m:nat) {struct n} : nat :=
  match n with
  | O => m
  | S p => S (add p m)
  end.
\end{coq_example}

The {\tt \{struct \ident {\tt \}}} annotation may be left implicit, in
this case the system try successively arguments from left to right
until it finds one that satisfies the decreasing condition. Note that
some fixpoints may have several arguments that fit as decreasing
arguments, and this choice influences the reduction of the
fixpoint. Hence an explicit annotation must be used if the leftmost
decreasing argument is not the desired one. Writing explicit
annotations can also speed up type-checking of large mutual fixpoints.

The {\tt match} operator matches a value (here \verb:n:) with the
various constructors of its (inductive) type. The remaining arguments
give the respective values to be returned, as functions of the
parameters of the corresponding constructor. Thus here when \verb:n:
equals \verb:O: we return \verb:m:, and when \verb:n: equals 
\verb:(S p): we return \verb:(S (add p m)):.

The {\tt match} operator is formally described
in detail in Section~\ref{Caseexpr}.  The system recognizes that in
the inductive call {\tt (add p m)} the first argument actually
decreases because it is a {\em pattern variable} coming from {\tt match
  n with}.

\Example The following definition is not correct and generates an
error message:

\begin{coq_eval}
Set Printing Depth 50.
\end{coq_eval}
% (********** The following is not correct and should produce **********)
% (*********      Error: Recursive call to wrongplus ...      **********)
\begin{coq_example}
Fail Fixpoint wrongplus (n m:nat) {struct n} : nat :=
  match m with
  | O => n
  | S p => S (wrongplus n p)
  end.
\end{coq_example}

because the declared decreasing argument {\tt n} actually does not
decrease in the recursive call.  The function computing the addition
over the second argument should rather be written:

\begin{coq_example*}
Fixpoint plus (n m:nat) {struct m} : nat :=
  match m with
  | O => n
  | S p => S (plus n p)
  end.
\end{coq_example*}

The ordinary match operation on natural numbers can be mimicked in the
following way.
\begin{coq_example*}
Fixpoint nat_match 
  (C:Set) (f0:C) (fS:nat -> C -> C) (n:nat) {struct n} : C :=
  match n with
  | O => f0
  | S p => fS p (nat_match C f0 fS p)
  end.
\end{coq_example*}
The recursive call may not only be on direct subterms of the recursive
variable {\tt n} but also on a deeper subterm and we can directly
write the function {\tt mod2} which gives the remainder modulo 2 of a
natural number.
\begin{coq_example*}
Fixpoint mod2 (n:nat) : nat :=
  match n with
  | O => O
  | S p => match p with
           | O => S O
           | S q => mod2 q
           end
  end.
\end{coq_example*}
In order to keep the strong normalization property, the fixed point
reduction will only be performed when the argument in position of the
decreasing argument (which type should be in an inductive definition)
starts with a constructor.

The {\tt Fixpoint} construction enjoys also the {\tt with} extension
to define functions over mutually defined inductive types or more
generally any mutually recursive definitions.

\begin{Variants}
\item {\tt Fixpoint} {\ident$_1$} {\params$_1$} {\tt :} {\type$_1$} {\tt :=} {\term$_1$}\\
        {\tt with} {\ldots} \\
        {\tt with} {\ident$_m$} {\params$_m$} {\tt :} {\type$_m$} {\tt :=} {\term$_m$}\\
        Allows to define simultaneously {\ident$_1$}, {\ldots},
        {\ident$_m$}.
\end{Variants}

\Example 
The size of trees and forests can be defined the following way: 
\begin{coq_eval}
Reset Initial.
Variables A B : Set.
Inductive tree : Set :=
    node : A -> forest -> tree
with forest : Set :=
  | leaf : B -> forest
  | cons : tree -> forest -> forest.
\end{coq_eval}
\begin{coq_example*}
Fixpoint tree_size (t:tree) : nat :=
  match t with
  | node a f => S (forest_size f)
  end
 with forest_size (f:forest) : nat :=
  match f with
  | leaf b => 1
  | cons t f' => (tree_size t + forest_size f')
  end.
\end{coq_example*}
A generic command {\tt Scheme} is useful to build automatically various
mutual induction principles. It is described in Section~\ref{Scheme}.

\subsubsection{Definitions of recursive objects in co-inductive types}

The command:
\begin{center}
  \texttt{CoFixpoint {\ident} : \type$_0$ := \term$_0$}
  \comindex{CoFixpoint}\label{CoFixpoint}
\end{center}
introduces a method for constructing an infinite object of a
coinduc\-tive type. For example, the stream containing all natural
numbers can be introduced applying the following method to the number
\texttt{O} (see Section~\ref{CoInductiveTypes} for the definition of
{\tt Stream}, {\tt hd} and {\tt tl}):
\begin{coq_eval}
Reset Initial.
CoInductive Stream : Set :=
    Seq : nat -> Stream -> Stream.
Definition hd (x:Stream) := match x with
                            | Seq a s => a
                            end.
Definition tl (x:Stream) := match x with
                            | Seq a s => s
                            end.
\end{coq_eval}
\begin{coq_example}
CoFixpoint from (n:nat) : Stream := Seq n (from (S n)).
\end{coq_example}

Oppositely to recursive ones, there is no decreasing argument in a
co-recursive definition. To be admissible, a method of construction
must provide at least one extra constructor of the infinite object for
each iteration. A syntactical guard condition is imposed on
co-recursive definitions in order to ensure this: each recursive call
in the definition must be protected by at least one constructor, and
only by constructors. That is the case in the former definition, where
the single recursive call of \texttt{from} is guarded by an
application of \texttt{Seq}. On the contrary, the following recursive
function does not satisfy the guard condition:

\begin{coq_eval}
Set Printing Depth 50.
\end{coq_eval}
% (********** The following is not correct and should produce **********)
% (***************** Error: Unguarded recursive call *******************)
\begin{coq_example}
Fail CoFixpoint filter (p:nat -> bool) (s:Stream) : Stream :=
  if p (hd s) then Seq (hd s) (filter p (tl s)) else filter p (tl s).
\end{coq_example}

The elimination of co-recursive definition is done lazily, i.e. the
definition is expanded only when it occurs at the head of an
application which is the argument of a case analysis expression.  In
any other context, it is considered as a canonical expression which is
completely evaluated. We can test this using the command
\texttt{Eval}, which computes the normal forms of a term:

\begin{coq_example}
Eval compute in (from 0).
Eval compute in (hd (from 0)).
Eval compute in (tl (from 0)).
\end{coq_example}

\begin{Variants}
\item{\tt CoFixpoint {\ident$_1$} {\params} :{\type$_1$} :=
  {\term$_1$}}\\ As for most constructions, arguments of co-fixpoints
  expressions can be introduced before the {\tt :=} sign.
\item{\tt CoFixpoint} {\ident$_1$} {\tt :} {\type$_1$} {\tt :=} {\term$_1$}\\
     {\tt with}\\
        \mbox{}\hspace{0.1cm} {\ldots} \\
        {\tt with} {\ident$_m$} {\tt :} {\type$_m$} {\tt :=} {\term$_m$}\\
As in the \texttt{Fixpoint} command (see Section~\ref{Fixpoint}), it
is possible to introduce a block of mutually dependent methods.
\end{Variants}

%%
%% Theorems & Lemmas
%%
\subsection{Assertions and proofs}
\label{Assertions}

An assertion states a proposition (or a type) of which the proof (or
an inhabitant of the type) is interactively built using tactics. The
interactive proof mode is described in
Chapter~\ref{Proof-handling} and the tactics in Chapter~\ref{Tactics}.
The basic assertion command is:

\subsubsection{\tt Theorem {\ident} \zeroone{\binders} : {\type}.
\comindex{Theorem}}

After the statement is asserted, {\Coq} needs a proof. Once a proof of
{\type} under the assumptions represented by {\binders} is given and
validated, the proof is generalized into a proof of {\tt forall
  \zeroone{\binders}, {\type}} and the theorem is bound to the name
{\ident} in the environment.

\begin{ErrMsgs}

\item \errindex{The term {\form} has type {\ldots} which should be Set,
    Prop or Type}

\item \errindexbis{{\ident} already exists}{already exists}
 
  The name you provided is already defined. You have then to choose
  another name.

\end{ErrMsgs}

\begin{Variants} 
\item {\tt Lemma {\ident} \zeroone{\binders} : {\type}.}\comindex{Lemma}\\
  {\tt Remark {\ident} \zeroone{\binders} : {\type}.}\comindex{Remark}\\
  {\tt Fact {\ident} \zeroone{\binders} : {\type}.}\comindex{Fact}\\
  {\tt Corollary {\ident} \zeroone{\binders} : {\type}.}\comindex{Corollary}\\
  {\tt Proposition {\ident} \zeroone{\binders} : {\type}.}\comindex{Proposition}

These commands are synonyms of \texttt{Theorem {\ident} \zeroone{\binders} : {\type}}.

\item {\tt Theorem \nelist{{\ident} \zeroone{\binders}: {\type}}{with}.}

This command is useful for theorems that are proved by simultaneous
induction over a mutually inductive assumption, or that assert mutually
dependent statements in some mutual co-inductive type. It is equivalent
to {\tt Fixpoint} or {\tt CoFixpoint}
(see Section~\ref{CoFixpoint}) but using tactics to build the proof of
the statements (or the body of the specification, depending on the
point of view). The inductive or co-inductive types on which the
induction or coinduction has to be done is assumed to be non ambiguous
and is guessed by the system. 

Like in a {\tt Fixpoint} or {\tt CoFixpoint} definition, the induction
hypotheses have to be used on {\em structurally smaller} arguments
(for a {\tt Fixpoint}) or be {\em guarded by a constructor} (for a {\tt
  CoFixpoint}).  The verification that recursive proof arguments are
correct is done only at the time of registering the lemma in the
environment. To know if the use of induction hypotheses is correct at
some time of the interactive development of a proof, use the command
{\tt Guarded} (see Section~\ref{Guarded}).

The command can be used also with {\tt Lemma},
{\tt Remark}, etc. instead of {\tt Theorem}.

\item {\tt Definition {\ident} \zeroone{\binders} : {\type}.}

This allows defining a term of type {\type} using the proof editing mode. It
behaves as {\tt Theorem} but is intended to be used in conjunction with
  {\tt Defined} (see \ref{Defined}) in order to define a
 constant of which the computational behavior is relevant.

The command can be used also with {\tt Example} instead
of {\tt Definition}.

\SeeAlso Sections~\ref{Opaque} and~\ref{Transparent} ({\tt Opaque}
and {\tt Transparent}) and~\ref{unfold} (tactic {\tt unfold}).

\item {\tt Let {\ident} \zeroone{\binders} : {\type}.}

Like {\tt Definition {\ident} \zeroone{\binders} : {\type}.} except
that the definition is turned into a let-in definition generalized over
the declarations depending on it after closing the current section.

\item {\tt Fixpoint \nelist{{\ident} {\binders} \zeroone{\annotation} {\typecstr} \zeroone{{\tt :=} {\term}}}{with}.}
\comindex{Fixpoint}

This generalizes the syntax of {\tt Fixpoint} so that one or more
bodies can be defined interactively using the proof editing mode (when
a body is omitted, its type is mandatory in the syntax). When the
block of proofs is completed, it is intended to be ended by {\tt
  Defined}.

\item {\tt CoFixpoint \nelist{{\ident} \zeroone{\binders} {\typecstr} \zeroone{{\tt :=} {\term}}}{with}.}
\comindex{CoFixpoint}

This generalizes the syntax of {\tt CoFixpoint} so that one or more bodies
can be defined interactively using the proof editing mode.

\end{Variants}

\subsubsection{{\tt Proof.} {\dots} {\tt Qed.}
\comindex{Proof}
\comindex{Qed}}

A proof starts by the keyword {\tt Proof}.  Then {\Coq} enters the
proof editing mode until the proof is completed. The proof editing
mode essentially contains tactics that are described in chapter
\ref{Tactics}. Besides tactics, there are commands to manage the proof
editing mode. They are described in Chapter~\ref{Proof-handling}. When
the proof is completed it should be validated and put in the
environment using the keyword {\tt Qed}.
\medskip

\ErrMsg
\begin{enumerate}
\item \errindex{{\ident} already exists}
\end{enumerate}

\begin{Remarks}
\item Several statements can be simultaneously asserted.
\item Not only other assertions but any vernacular command can be given
while in the process of proving a given assertion. In this case, the command is
understood as if it would have been given before the statements still to be
proved. 
\item {\tt Proof} is recommended but can currently be omitted. On the
opposite side, {\tt Qed} (or {\tt Defined}, see below) is mandatory to
validate a proof.
\item Proofs ended by {\tt Qed} are declared opaque. Their content
  cannot be unfolded (see \ref{Conversion-tactics}), thus realizing
  some form of {\em proof-irrelevance}. To be able to unfold a proof,
  the proof should be ended by {\tt Defined} (see below).
\end{Remarks}

\begin{Variants}
\item \comindex{Defined}
  {\tt Proof.} {\dots} {\tt Defined.}\\
  Same as {\tt Proof.} {\dots} {\tt Qed.} but the proof is
  then declared transparent, which means that its
  content can be explicitly used for type-checking and that it
  can be unfolded in conversion tactics (see
  \ref{Conversion-tactics}, \ref{Opaque}, \ref{Transparent}).
%Not claimed to be part of Gallina...
%\item {\tt Proof.} {\dots} {\tt Save.}\\
%  Same as {\tt Proof.} {\dots} {\tt Qed.}
%\item {\tt Goal} \type {\dots} {\tt Save} \ident \\
%  Same as {\tt Lemma} \ident {\tt :} \type \dots {\tt Save.}
%  This is intended to be used in the interactive mode.
\item \comindex{Admitted}
  {\tt Proof.} {\dots} {\tt Admitted.}\\
  Turns the current asserted statement into an axiom and exits the
  proof mode.
\end{Variants}

% Local Variables: 
% mode: LaTeX
% TeX-master: "Reference-Manual"
% End: 

