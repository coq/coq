\documentclass[a4paper]{book}
\usepackage{fullpage}
\usepackage[latin1]{inputenc}
\usepackage[T1]{fontenc}
\usepackage{amsfonts}

\newcommand\formula[1]{\begin{tt}#1\end{tt}}
\newcommand\tactic[1]{\begin{tt}#1\end{tt}}
\newcommand\command[1]{\begin{tt}#1\end{tt}}
\newcommand\term[1]{\begin{tt}#1\end{tt}}
\newcommand\library[1]{\texttt{#1}}
\newcommand\name[1]{\texttt{#1}}

\newcommand\itemrule[3]{
\paragraph{#1}
\begin{quote}
\begin{tt}
#3
\end{tt}
\end{quote}

Name: \texttt{#2}}

\newcommand\zero{\texttt{zero}}
\newcommand\op{\texttt{op}}
\newcommand\opp{\texttt{op'}}
\newcommand\oppp{\texttt{op''}}
\newcommand\phimapping{\texttt{phi}}
\newcommand\D{\texttt{D}}
\newcommand\elt{\texttt{elt}}
\newcommand\rel{\texttt{rel}}
\newcommand\relp{\texttt{rel'}}

%%%%%%%%%%%%%%%%%%%%%%%%%%%%%%%%%%%%%%%%%%%%%%%%%%%%%%%%%%%%%%%%%%%%%%

\author{The Coq Development Team\\
initial version by Hugo Herbelin}
\title{A project for the standard library of Coq\\
Draft
}

\begin{document}

\maketitle

\mbox{}
\vspace{4cm}{}

This document discusses features of an ideal Coq standard library that
could serve as a solid basis both for future extensions and
(hopefully) for ``laying formal mathematical knowledge''.

The document includes both long-term prospective aspects and technical
points about how to improve the actual library in the short term.

\tableofcontents

\chapter{What to do with the standard library?}

\section{Historical context of the standard library}

Coq is a system originating in the initial work of Thierry Coquand and
G�rard Huet on the Calculus of Constructions. Its implementation,
called CoC, was started in 1984. Version 4.10 of CoC was released in
1989.  With version 5.6, the underlying logic became the Calculus of
Inductive Constructions and changed its name to Coq. With version 8.0,
the impredicativity of the sort \term{Set} was abandoned.

The standard library of Coq was first released as such in April 1993
as part of Coq version 5.8.0. It contained libraries about natural
numbers (ARITH), boolean values (BOOL) and lists (LISTS). They were
structured pretty much the same as it still is in Coq version 8.0.  It
contained also an initial library (SYSTEM) for connectives and Peano
axioms that was automatically loaded at startup. It contained almost
empty libraries for sets (SETS), relations (RELATIONS), streams
(STREAMS) whose purpose was basically to provide examples such as the
proof of Schroeder-Bernstein theorem, the proof of Newman lemma, and
the correctness of Eratosthene sieve. A collection of examples of
algorithms correctness proofs (PROGRAMS) completed this first version.

Formerly, the same collection of results (but poorer) was simply
released as an informal collection of examples\footnote{For the
record, the directory containing the examples changed its name from
the French \texttt{exemples} to the English \texttt{EXAMPLES} at the
time of the first public release, version V4.10, in 1989.}.

Depending on the notions, we can trace their current name back as
early as the first 1984 prototype\footnote{This is for instance the
case of \texttt{sig}, \texttt{exist}, \texttt{eq},
\texttt{refl\_equal}, \texttt{sym\_equal}, \texttt{trans\_equal},
\texttt{bool}, \texttt{list}, \texttt{nat}, \texttt{plus\_0\_r},
\texttt{plus\_0\_l}, \texttt{plus\_comm}, \texttt{plus\_0\_r},
\texttt{plus\_0\_l}, \texttt{plus\_comm}...} in C, but extensions
really started around the version 4.3 with the implementation of the
first extraction mechanism and the development of correctness proofs
of algorithms (e.g. Manna's binary search algorithm), then in version
4.10 with the implementation of basic tactics, then in version 5.6
with the help of the new primitive inductive definitions and
pattern-matching constructions.

At the end, this resulted in a standard library made of a collection
of notions added heterogeneously and anarchically, with still the
traces of the different strata of Coq evolution, at periods when
rewriting, or coercions, or high-level tactics for induction, or
implicit arguments, etc. were not yet available.

As an example of the gestation process of the standard library,
Appendix~\ref{appendix-historical} shows the evolution of \texttt{sig}
and \texttt{exist} along the history of Coq.

\section{Aspects of the restructuration of the standard library}

Reworking the standard library covers several aspects that we list
below.

\subsection{The naming layer}

This first layer is a pure question of uniform naming scheme
(e.g. should we have a generic name for all reflexivity lemmas:
shouldn't le\_n be le\_refl and refl\_equal be eq\_refl, etc.). In this
level, we also include very basic conventions on naming the variables
(x,y,z or n,m,p, or n1,n2,n3, ...?), on the order of hypotheses
(e.g. where should the intermediate variable of a cotransitivity
statement be quantified?), on the file names (do we really want
Peano\_dec.v and DecBool.v), etc.

This layer, which would give a uniformity to the library, is however
complex to implement if compatibility with the existing developments
is required.

To preserve the existing developments, the only reasonable technique
is to use a translator as it was done for switching from (old) pre-V8
syntax to new V8 syntax. Would the user accept a new automatic
translation phase?

A light way to automatically translate development is in two phases:
in a first phase, compilation produces a file of localised changes
(such as: replace ident id at line xx, character yy, into id'), in a
second phase, a clever translater applies the precompiled
substitutions.


\subsection{The contents layer}

This layer is about deciding, for a given theory, what are the
meaningful lemmas that are legitimate to be present in a standard
library. This covers proving obviously missing lemmas but also removing
possible redundancies (e.g. is it desirable to have both

\verb!mult_S_lt_compat_l : forall n m p, m < p -> S n * m < S n * p!

and

\verb!mult_lt_compat_r : forall n m p, n < m -> 0 < p -> n * p < m * p!?)

and reformulating non appropriately stated lemmas.

A reasonable approach for implementing this level is that the
responsabilities are dispatched. Typically the contents of the computer
science libraries and the arithmetical, algebraic and analysis
libraries could be supervised by different persons.

\subsection{The methodological layer}

This third layer includes questions like:
\begin{itemize}

\item Should we have a library for reasoning on decidable
   propositions? (see e.g. G. Gonthier's reflection of decidable
   propositions in booleans).

\item Should we define classical real numbers on top of constructive ones or
should we have completely independent characterisations/constructions?

\item Should we define N, Z, Q, ...  axiomatically or should we construct
      them? Wouldn't it be reasonable to have both?

\item Shouldn't we use modules to factorise all properties
of, say, totally ordered rings, rather than re-proving them for each
new concrete structure? This seems reasonable.

\item Should we define distinct objects for \verb!<! and \verb!>!, or
should we just consider it is a purely notational problem? Should we
define \verb!<=! from \verb!<!, as suggested by constructive
mathematics (and actually done in CoRN), or the reverse (as done in
Arith), or shall we have different definitions depending on which
structure is considered?  Should we define < on decidable structures
as a function (as done for ZArith) or as a predicate (as done in
Arith)?  Etc.
\end{itemize}

Connected to the methodological level is the support for
automation. E.g. if we accept ring or field as granted, do we really
need to state extra technical lemmas that ring or field would solve
immediately.  On the opposite side, automation based on the auto
tactic requires a well designed set of theorems that surely exceed the
simple set of basic defining properties of the structure on
consideration (e.g. couldn't \verb!forall n m:Z, n + - m > 0 -> n > m!
be such a theorem?)

\subsection{The ``logistic'' layer}

By ``logistic'' is intended the question of which theories are
relevant for a so-called standard library (e.g. is category theory or
projective geometry relevant to build extra works or should we
consider them as leaves of the dependency graph of theories).

\subsection{The documentation layer}

\subsubsection{Metadata}

Metadatas such as authors of the files, history of the files, textbook
or paper used as references, are minimum expectations for a library with
scientific claims. In most of the files such informations are
missing.

\subsubsection{Browsing}

Coqdoc is a good tool for documentation and browsing the standard library.

\subsubsection{Natural language}

MoWGLI provided a very interesting platform for reading proofs in
natural language. It is not fully satisfactory (proofs too long,
details not put in evidence), but partly because the proof scripts
themselves are not written with natural language rendering in mind.

At the time being, the question of natural language looks immature to
me. Directions to explore before having a better feeling are:

\begin{itemize}
\item The mathematical mode developed in Nijmegen

\item The development of tactics behaving closer and closer to what the
  user has in mind.

\item A rendering based as much on the proof script than on the raw
proof-term, as partly explored by MoWGLI and Lo�c Pottier.
\end{itemize}

\subsubsection{Searching}

MoWGLI provided an excellent platform for searching formal libraries.
It has to communicate with the standard library.

\subsubsection{The symbolic language}

Coq is basically text made of Unicode characters.  How far can we go
in using more involved notations in the interactive proof development,
starting from three dimensional notations, such as indices, fractions,
etc.  The Matita~\cite{Matita} project is an interesting project to
follow on this side (see e.g. L. Padovani's work on typing mathematics
formula).

Alternatively, how to export notations from a Coq file, so that its
rendering in TeX or HTML benefits of these notations.
This raises questions such as:

\begin{itemize}
\item What is the status of indices in Coq identifiers? Shall we do
something special with the indices chart of Unicode?

\item How to render operators with side conditions such as division
that expects a non-zero proof in CoRN?

\item Can we go on with Camlp4 as it is (the LL1 restriction, no
detection of conflicts)?

\end{itemize}

\section{Methodology and connected projects}

\subsection{General remarks}

The sources of progress come from the formalisations. The early
formalisation of category theory by Amokrane Sa�bi was a motivation
for coercions, implicit arguments, canonical structures.

The need for shorten further expressions motivated extensible syntactic
notations, automatic inference of types.

The need for tactics well-suited to the human mind motivated the (new)
induction, the forward reasoning tactics and the tactics for
fine-tuned simplification of expressions.

The need for notation overloading motivated the introduction of
interpretation scopes.

The library of binary integers was crucial for the development of
further libraries.

All of these innovations had flaws. The current document lists some of
them that are related to the standard library.

The standard library project aims to both correct the flaws (when
possible) and to experiment new features.

The reworking of the standard library will certainly provide an
excellent source of problems for developing new features. We list the
ones already in sight.

\subsection{Improving the module system}

A modular development of the arithmetical properties of
ordered semi-rings, rings and field has been tempted 2 years ago.
This work has been carried on by Claudio Sacerdoti during its
post-doc.  Rewriting of abstract equalities was necessary, so he
extended the setoid rewrite mechanism. Then, expansion of the
instances of functors was necessary to go on further, so he
implemented many new features of the module system. Especially, it
appeared that the implementation/interface duality on which the Coq
module system is based was too much restricted. It has been decided to
develop a new system based on a more general notion of
refinement/subtyping (a master student is working on it).

\subsection{Improving tactics}

We list a few ideas for improvement of the basic tactics:

\begin{itemize}

\item Induction: the \texttt{induction} is an improvement over
\texttt{elim}, we have ideas to better combine inversion and
induction (as an example, the inversion steps of the proof of the
transitivity lemma in the POPLmark should be automatised).

\item Induction: mention of the occurrences on which the induction is
performed (including occurrences in context) should be attached to the
tactic invocation.

\item Cleaning hypotheses: it is most common to use hypotheses only
once. The default should be to erase the hypotheses invoked by
\tactic{apply} or \tactic{generalize}.

\item Apply: it is a pity that \texttt{apply} does not traverse
``cosmetic'' constant such as \texttt{not}, \texttt{reflexive}, or
that it does not know how to insert coercions in case of mismatch
(e.g. applying a iff statement should automatically split into trying
to apply the if or the only if part; this is typically something that
hinders the development of the standard library: one is often forced
to state the if and the only if part of an equivalence in two distinct
statements while it should be more intuitive to have only one).

\end{itemize}

G. Gonthier~\cite{Ssreflect06} proposed a concise syntax for various
tactics that it is worth to be studied. He proposed also to extend the
use of incompletely specificied terms, which is a pain of the current
implementation.

J.-C. Filli�tre proposed to refer to hypotheses by giving a pattern of
its statement (e.g. \texttt{apply :$t$} would apply the first hypothesis
that matches the  pattern $t$).

\subsection{The role of coercions}

The standard library does not use coercions at all. However, there are
many situations where coercions could be useful. Such situations
include:

\begin{itemize}
\item a coercion from \name{sumbool} to \name{bool},
\item a coercion from \name{bool} to \name{Prop}
\end{itemize}

Some forms of (non standard) coercions should be useful also in
tactics, typically when trying to apply an \name{iff} lemma (two
coercions are needed then), or more generally when trying to apply a
\name{and} statement of which it is only a component of it which is
applicable.

On the opposite side, introducing coercions in the standard library
may lead to a complexification of the learning of Coq.

\subsection{The role of the user contributions}

The user contributions contain many many interesting lemmas on various
subjects. A clarification of the licence of these developments has
been done recently and most of them are now LGPL. Some authors, among
the most interesting contributors, have already suggested that their
work could be reused (Milad Niqui, Russell O'Connor, Laurent Th�ry).

It remains to realise a smooth integration of these portions of the
contributions into a uniform consistent standard library.

\subsection{Cooperation}

Natural external possible partners for the project of revision of the
standard library include Bologna, Nijmegen, Sophia-Antipolis-Nice and
the ``Mathematical Component'' Common Research Action.

For the revision of the standard library, the Coq Development Team
hires a post-doc researcher.

\chapter{The components of the standard library}

\section{The initial state: \library{Init}}

\paragraph{Historical context}

Interestingly, the genesis of \library{Init} comes from the development
of proof of the specification of a binary-search algorithm (usable
e.g. for computing integer roots) and of the correctness of a line
formatting algorithm in CoC V4.2 (both developments were by Christine
Paulin). The standard notions on which the developments relied were
moved to a file called Base.v in CoC V4.4.

This is the time when \term{sig}, \term{sig2}, \term{sum},
\term{sumor}, \term{sumbool}, \term{prod}, and \term{and} were
created. In Coq V4.2, \term{sumor} was called \term{sum01} and the
Prop argument came first, while \term{or} was called \term{sum0}. The
respective constructors were \term{inl01}, \term{inr01}, \term{inl0}
and \term{inr0}. This was related to the fact that the informative and
non informative components of \term{Prop} were respectively called
\term{Prop1} and \term{Prop0} in V4.2.

The irregular distinctions between \term{left}/\term{right} (for
\term{sumbool}), \term{inleft}/\term{inright} (for \term{sumor}),
\term{or\_introl}/\term{or\_intror} (for \term{or}) came from V4.4.
Similarly for the distinction between \term{fst}/\term{snd} (for
\term{prod}) and \term{proj1}/\term{proj2} (for \term{and}). [Note
that \term{or\_introl}/\term{or\_intro} were also called
\term{sum\_IL} and \term{sum\_IR} in a file from V4.2.]

This volatility of names emphasizes their arbitrariness. Seeking for
more uniform names would be no sacrilege.

\paragraph{Suggested revisions}

\begin{itemize}
\item Set \term{iff} as \command{hint unfold}, or, better, define {\tt
  iff} as a primitive inductive type (suggestion of Georges Gonthier)
  and add the constructor as a {\tt hint} (leaving open the problem of
  destruring inductive types in {\tt auto}).
\item Add support for reasoning over \name{iff}: either combine
  \tactic{setoid\_rewrite} and \tactic{apply} or make that
  \tactic{apply} on a \texttt{iff} statement automatically select the
  relevant projection.
\item Move minimal terminology on relations to \library{Init} (add a
  new file \library{Init/Relations}?)
\item Move minimal operations on booleans and booleans as propositions
  to \library{Init} (add new files \library{Init/BoolOps} and
  \library{Init/BoolProp}?)
\item The question of propositional and informative connectives:
\begin{itemize}
\item Merge \term{sum}, \term{sumor} and \term{sumbool} (possible thanks
  to sort polymorphism of inductive types?
\item Having \term{\~{} A} as a notation for \term{A -> False} rather
than for \term{not A} would avoid all these problems of converting
\term{A -> False} and \term{not A} [suggested by EM].
\item If yes, what uniform naming schemes? We need at least uniform
naming schemes for \name{or} and \name{and} and for \name{prod} and
\name{sum}/\name{sumor}/\name{sumbool}.

Adopt the following?

\noindent
\begin{tabular}{lllll}
\hline & (\term{Prop},\term{Prop}) to \term{Prop} & (\term{Type},\term{Type}) to \term{Type} & (\term{Type},\term{Prop}) to \term{Type}\\
conjunction/product & and/conj/proj1/proj2 & prod/pair/fst/snd & \\
disjunction/sum & or/or\_ind/inj1/inj2 & sum/sum\_rect/inl/inr & \\
$\exists$/subtype/$\Sigma$ & ex/ex\_intro(?)/ex\_ind & ? & ? \\
\end{tabular}

Shall we reuse \name{sig} for \name{sigT} and \name{subset} for old
\name{sig}? 

We may also change \name{proj1} and \name{proj2} to 
\name{and\_elim\_l} and \name{and\_elim\_r}?

\end{itemize}
\item Give the same implicit arguments to \name{eq\_rect} and
\name{eq\_rect\_r}, etc.
\end{itemize}

\section{Arithmetics}

Coq V8.1 has the following arithmetical libraries: \library{Arith}
(Peano's arithmetic), \library{NArith} (positive and natural numbers
based on binary digits) and \library{ZArith} (integer arithmetic based
on binary digits).

The \library{Arith} library is the historical arithmetical
library. The type \name{nat} (with concrete syntax \term{0} and
\term{S} for the constructors) was defined impredicatively in the
older known archived version of CoC (namely version 1.10). The order
on \name{nat}, named \name{le} can be found in CoC V2.13 with the same
characterisation as today (reflexivity and stability wrt successor).

The \library{ZArith} library, contributed by Pierre Cr�gut, appeared
in Coq V6.2 as a component for the \tactic{omega} tactic. It gained a
great success and has been regularly extended.

At the time of Coq V8.0, \library{ZArith} was cleaned up a bit. The
parts of it that were concerned with positive and natural numbers
moved to a directory called \library{NArith}.

Still, \library{Arith} on one side and \library{NArith} and
\library{ZArith} on the other side have very different structures and
they don't share many things. Still, the theory of the types
\name{N} and \name{nat} should have the same signature.

This justifies to go in the direction of abstractly characterising the
arithmetical structures.

\paragraph{Abstract characterisation of the arithmetical structures}

Having different implementations sharing the same interface would help
in these ways.

\begin{itemize}
\item General tools can apply abstractly on some structure
      independently of its representation. For instance, if the
      ordered structures on which Fourier elimination procedure
      applies were abstractly characterized, this elimination
      procedure could be applied uniformly on $\mathbb{Z}$ (hence
      $\mathbb{N}$), $\mathbb{Q}$, $\mathbb{R}$. Depending on additional
      axioms on the density of the structure, extra tactics could be
      applied to apply either the Omega or the Fourier-Motzkin
      procedures.

\item other libraries can apply abstractly on some structure
      independently of its representation if they wish (e.g. a library
      on properties of a linear order, applicable to any arithmetical
      structure by just knowing the order is linear).

\end{itemize}

The abstract arithmetical structures can possibly be given in an
\library{Algebra} library (see section~\ref{library-algebra})

\subsection{\library{ZArith}}

This implementation of integer arithmetic took profit of the old to
new syntax translator to be cleaned up a bit.
The definitions are globally nice. Let us mention the following issues though

\begin{itemize}
\item \name{Zplus} is a bit inefficient because comparing the two
arguments twice (IMHO, \name{Zplus'} should be preferred but it may
lead to some -- few -- incompatibilities)

\item The computational definitions of order on \name{Z} is
interesting but it is unclear how to uniformly introduces orders for
all of $\mathbb{N}$, $\mathbb{N^*_+}$, $\mathbb{Z}$ and $\mathbb{Q}$
(both a definition in \term{Prop} and a definition in \term{bool}
and/or \term{comparison} should be provided, with the lemmas stating
the equivalence).

\item To have both a ``less than'' and a ``greater than'' is a source
of complication. Only one of them should be kept and the other should
be a matter of notations.

\item The proofs about the order relation should be factorized with
the ones in $\mathbb{N}$, $\mathbb{N^*_+}$, $\mathbb{Z}$, $\mathbb{Q}$ and
$\mathbb{R}$ (ongoing work from Evgeny about that).

\item Do the peripheral results need to be cleaned up?

\item About min and max: Russell O'Conor provided a nice parametric
derivation of the lattice properties of min and max, applicable to
QArith as well (and more generally to any ordered semi-lattice or
lattice).

\end{itemize}

\subsection{\library{NArith}}

Desserve to be develop as much as \library{Arith} and \library{ZArith}.
Factorise results using functors?

\subsection{\library{QArith}}

\subsection{\library{Reals}}

\paragraph{Miscellaneous remarks}

\name{INR} is not uniformly defined: there is a special case for 1
that shows up every time one want to use it.

As a matter of fact, the archimede axiom for reals is derivable from
the rest of the axioms. The current \name{archimed} axiom amounts to
specifying \name{up} (the integer upper bound of any real) as a
function.

Hints are circular (e.g. loop between \name{Rle\_ge} and
\name{Rge\_le}), or create expansions (e.g. \name{Rplus\_eq\_reg\_l}),
while some of them prove non atomic, conjonctive results (like
\name{Rplus\_ne} and \name{Rmult\_ne}). It is unclear what can be kept
from the main files of the library on reals.

\name{Rtopology.included} is redundant

\section{Data structures}

\subsection{\library{Bool}}

\paragraph{Historical context}

The type \name{bool} itself was (of course) present in CoC version 1.
Properties such as \term{(true=b $\backslash$/ false=b)},
\term{(false=b) -> (true=b)} arrived as lemmas in \library{Arith.v} in
CoC V4.10. The motivation may have been to justify the axiom
\term{true<>false} (not derivable in CoC).

First operations on \name{bool} arrived in Coq V5.10 as part of
Paulin-Werner's tautology checker case study. Operations, such as
\name{Is\_True}, and results about \name{neg}, such as
\name{no\_fixpoint\_neg} came from Huet's shuffle trick case study.

File \name{Bool.v} arrived in Coq V5.8.0 (at the same time when the
directory EXAMPLES moved to THEORIES). It collected the operations
from the tautology checker and the shuffle trick.  More lemmas about
\name{eqb} and \name{negb} arrived in V5.10.13 with applications in
Gimenez's development of the ABP protocol. First hint
(\name{diff\_true\_false}) arrived in V5.10.11.

Properties of \name{orb} arrived in file \library{Orb.v} in V5.10.14.b
for use in Lyon's contribution CIRCUITS. \library{Orb.v} have been
further integrated to \name{Bool.v} in V6.2. The libraries
\name{Zerob.v} and \name{IfProp.v} were also motivated by the
contribution CIRCUITS.

Most extensions were then made then by Patrick Loiseleur: lemmas in V6
revisions 1.11, 1.12 and 1.14 and hints in revision 1.15 (V6.2.3); by
Christine Paulin (V6 revision 1.6 and V7 revision 1.8); by
J.-C. Filli�tre (V7 revision 1.21 for reasoning on \term{b=true} in
\library{FSets}).

\paragraph{Suggested revisions}

\begin{itemize}

\item {\bf Connection between \name{bool} and \name{Prop}}

The question of suppporting reasoning in \name{bool} as a replacement
of reasoning in \term{Prop} in combination with \name{sumbool} has to be
raised.

In the first case, it is certainly possible to smoothly reason on
decidable propositions thanks to an implicit coerction from
\name{bool} to \term{Prop}. It is worth to introduce this complexity.

\bigskip

The current default embedding from \name{bool} to \name{Prop} is
\name{Is\_true} which maps \name{true} to \name{True} and \name{false}
to \name{False}. Others users seem to prefer mapping the boolean
\term{b} to \term{b=true}. The best choice, suggested by J.-M. Notin,
seems in fact to be the following inductive definition:

\begin{verbatim}
Inductive eq_true : bool -> Prop := is_eq_true : eq_true true.
\end{verbatim}

The problem of the definition \term{b=true} is that it introduces
confusions between statements intended to embed \name{bool} ito \name{Prop} 
and reasoning on pure boolean expressions. For instance, relying on
\term{b=true} led to add lemmas such as \term{negb b <> true -> b =
true} as hints, creating inconsistencies in the \tactic{auto}
mechanism (goals of the form \term{b=true} were provable while goals
of the form \term{true=b} or \term{b=false} were not).

To solve this problem of interference between the two kinds of use of
the booleans, a solution is to use a name hiding the definition of
\term{b=true} and to develop a set of hints/theorems stating the
properties of the embedding without having to clutter the theory of
booleans (e.g. with the symmetry of equality or with general
properties breaking the \term{b=true} structure such as \term{b <>
false -> b = true}.  This is the approach followed e.g. by G. Gonthier
for the proof of the four colour theorem.

However, the problem with using a name is that it eventually gets
unfolded leading to a non standard representation of the embedding.

Here is a summary of the advantage of the different solutions wrt a
choice of 4 criterion.

\noindent\begin{tabular}{lllll}
 & ability  & no theorem & stability  & hidable as  \\
 & to rewrite  & cluttering & by unfolding &  a coercion \\
\term{b=true} & OK & BAD & OK & BAD \\
\term{if b then True else False} & BAD & OK/BAD & OK & BAD \\
naming \term{b=true} & OK & OK & BAD & OK \\
naming \term{\small if b then True else False} & OK (with coercions) & OK & BAD & OK \\
\name{eq\_true} & OK & OK & OK & OK\\
\end{tabular}

One could then ask whether it would be better to have some
\name{Is\_False} to express that boolean properties are wrong, or if
it would be better to reason with \texttt{~ Is\_True}. The first case
provides a better symmetry, but we cannot have two coercions from
\name{bool} to \name{Prop}. Shall we renounce to the symmetry and
express negative conditions as, either \texttt{~ Is\_True cond}
(i.e. \texttt{~ cond} with a coercion) or \texttt{Is\_True (negb
cond)} (i.e. \texttt{negb cond} with a coercion)? Consider for
instance the following typical example in \library{IntMap}:

\verb!forall (m:Map A) (a:ad) (y:B), in_dom A a m = false -> MapDisjoint A B m (M1 B a y)!.

\item {\bf Connection between \name{sumbool} and \name{bool}}

It would be nice to have an implicit coercion from \name{sumbool}
to \name{bool}. Wouldn't be too complex for beginners?

What is the purpose of \texttt{bool\_of\_sumbool} in
\name{Sumbool.v}. Why does not it return a bool ?  In
\name{Sumbool.v}, \texttt{bool\_eq\_rec} and co are become obsolete
with the tactics we now have.

\item {\bf General schemes}

Shouldn't \term{bool\_dec} be generated automatically~(see
\ref{ind-schemes}).

Shouldn't \term{diff\_true\_false} be generated automatically~(see
\ref{ind-schemes}). Indeed \name{diff\_true\_false} and
\name{diff\_false\_true}as hints only serve to compensate the absence
of discrimination in auto (see~\ref{automation}).

Shouldn't \term{eqb} be also generated automatically?

\item {\bf The standard operators}

The standard operators \name{andb}, \name{orb}, 
\name{xorb}, \name{negb} should probably go in the initial state, together with
their notations and \texttt{bool\_scope} (but not open by default).

What's the genericity of \name{implb}? As the reflection of
(decidable) \term{->} within \name{bool}?

Is \texttt{ifb} useful (it used to be when the \texttt{if} notation
didn't exist)?

Implicit arguments dans {\tt orb\_false\_elim} and {\tt orb\_true\_elim}?

\item {\bf The boolean algebra structure}

The boolean algebra properties (\name{orb\_true\_r},
\name{orb\_negb\_r}, \name{orb\_comm}, \name{andb\_orb\_distrib\_r},
\name{absorption\_andb}, ...) should occur somewhere to say that the
booleans implements the signature of boolean algebras (to be put
somewhere in the \library{Algebra} library~\ref{library-algebra}). See
also~\ref{modules}.

Derived properties (de Morgan laws, involution of \name{negb},
idempotency of \name{orb} and \name{andb}, ...) may be either reproved
(their proofs are simpler enough) or inherited from a generic
treatment of the properties of boolean algebras.

\item {\bf Miscellaneous remarks}

Some lemmas seem rather trivial, what could they be useful for (e.g.
\name{eq\_true\_false\_abs}, \name{not\_true\_is\_false},
\name{not\_false\_is\_true})

What is the motivation for \name{leb}? Why not rather
\texttt{Is\_true (implb a b)}? Moreover, \name{leb} already exists in
\library{Compare\_dec}.

Why \texttt{eqb\_subst} rather than \name{eqb\_eq}?

Wouldn't be \name{eqb\_prop} better with \name{Is\_true} (a coercion)?

\end{itemize}

\subsection{\library{Lists}}

My feeling is that the \library{Lists} library should at least cover
the Objective Caml list library (because the Objective Caml list
library is nice, because it is in a language close to the one of Coq,
because this is a priori to have a certified list library for
Objective Caml, and finally because this is a following of the ideas
having led to the \library{FSet}/\library{FMap} library).
For instance, are missing: \name{for\_all}, \name{exists}, ...


Is the right associativity of \term{++} the right one?


\paragraph{Open questions}

How to deal with partial functions? The work on native exceptions in
CIC is not a short term work.

Starting from what we have, I would conclude that the two possible
variants have to be defined (i.e. the variant which returns an option
type and the variant which expects a default value), with connections
between them.


\paragraph{Inconsistencies and recommendations}

\begin{itemize}
\item \term{app\_nil\_end} wrong orientation
\item \term{nth} and \term{nth\_default} redundant
\end{itemize}

Concrete syntax for lists ``\texttt{[x::y::z]}''. Rarely useful in
developments about lists in general but often useful in concrete
situations that use lists as a tool.

\subsection{\library{Strings}}

\subsection{\library{FMaps} and \library{Intmap}}

\subsection{\library{FSets}}

Why isn't it in a directory distinct of \library{FMaps}?

\subsection{\library{Relations}}

\paragraph{Historical context}

The first development about relations was a proof of Newman's lemma
(probably formalised by G�rard Huet) as soon as the first version of
Coq. From CoC V4.5, the part on the transitive
closure of a relation was considered of general interest and was
isolated to a file called \library{Relations.v}. At the time the first
version of a standard library was released, the directory
\library{RELATIONS} contained only these two files. Then B. Barras
and C. Cornes started to enrich the library, and the contents
dealing with transitive closure moved to a file named
\library{Rstar.v}.

First appeared \library{Newman.v} and \library{Rstar.v} (the latter
was named \library{Relations.v} until V5.9). These files (whose
contributor was probably G�rard Huet) probably served as an emphasis
of how the Calculus of Constructions can deal with this kind of
notions.

\library{Relation\_Definitions.v} (issued from an embryon of
\library{SYSTEM/Relns.v} in V5.10.6) and the contents of
\library{Wellfounded}, \library{Relation\_Operators.v} and
\library{Operators\_Properties.v} (contributed by B. Barras and
C. Cornes) appeared in V5.10.15. It is essentially unchanged until
now.

\subsubsection{Questions}

Is it worth to have \verb!relation := A -> A -> Prop! ? Why not rather
\verb!relation := A -> B -> Prop! ? But then, how to call
\verb!A -> A -> Prop!, choose another name (e.g. \name{endorelation}) or
implements specific tools that implement common statements such as
``we will write relation$_A$ for relation$_{AB}$ when A=B''.

\subsubsection{Miscellaneous remarks}

They are redundancies between Relations\_*.v in Sets, the files in
Relations and RelationClasses in Classes. All common definitions
should a priori be linked to their type-classe version since this will
provide the general tools built on type classes.

Remove Rstar.v (inductive definition of refl-trans closure) and Newman.v.

Change the definition of commut (it is not in the intuitive way) [but
compatibility...].

How to manage the overloading of the word ``inclusion'' for relations,
unary predicates, etc? Needs some extra features in the language?

\subsubsection{Missing notions}

Standard notions of injectivity, surjectivity, etc are missing.

\subsection{\library{Wellfounded}}

\subsection{\library{Sets}}

\library{Relations\_}* files to move to \library{Relations}.

\library{Permut.v} would benefit of being presented as a module.

\subsection{\library{Setoids}}

\subsection{\library{Sorting}}

To move \library{Permutation.v} and \library{Sorting.v} in
\library{Lists} and \library{Heap.v} in contribs??

\section{Missing components}

\subsection{\library{Algebra}}
\label{library-algebra}

There is currently no library about standard algebraic structures (to
the exception of setoids).

I believe that a library that provides the basic algebraic properties
of operations and relations, and the basic structure that have these
properties is worth to be part of {\em standard} library (typically
the library \library{Relations} could be part of it.

There are several user contributions providing basic algebra,
especially in Nijmegen (constructive algebra) and in Sophia-Antipolis.

Can the constructive approach be followed smoothly without making life
more complicated for the classical mathematician?

\paragraph{Technical aspects}

Knowing that the carrier and operations of an algebraic structure are
the most used component of a structure; knowing that seeking the
carrier and operations through nested coercions increase the size of
terms and proofs and of their reduction, wouldn't it be better to keep
the carrier and operations at the top of the structure. Something
like:

\begin{verbatim}
Record is_Monoid (A:Setoid) (op:A->A->A) (zero:A) (inv:A->A) := {
  mo_ass : associative op;
  mo_unit_rht_prf : is_rht_unit op zero;
  mo_unit_lft_prf : is_lft_unit op zero;
}.
Record Monoid := {
  mo_crr : Setoid;
  mo_op : mo_crr -> mo_crr -> mo_crr;
  mo_unit : mo_crr;
  mo_prf : is_Monoid mo_crr mo_op mo_unit
}.

Record is_Group (A:Setoid) (op:A->A->A) (zero:A) (inv:A->A) := {
  gr_ass : associative op;
  gr_unit_prf : is_rht_unit op zero;
  gr_inv_prf : forall x, is_rht_inverse op zero x (inv x)
}.
Record Group := {
  gr_crr : Setoid;
  gr_op : gr_crr -> gr_crr -> gr_crr;
  gr_unit : gr_crr;
  gr_inv : gr_crr -> gr_crr;
  gr_prf : is_Group gr_crr gr_op gr_unit gr_inv
}.
\end{verbatim}

Note the use of a \name{Setoid} instead of a type for the
carrier. Equality is so central that it probably deserves to be
associated to the carrier rather than to the structure itself (it
remains the question of whether defining setoids from apartness or
from equality).

Avoiding deeply nested coercions seems to somehow follow the
mathematical practise which typically says, ``let $(A,+,0,{}^{-1})$ a
group'' meaning ``let $(A,+,0,,{}^{-1})$ be a tuple satisfying the
hypotheses of a group.

\subsection{Floating-point numbers}

This looks like a central but missing library. Still there are some
implementations of the IEEE 754 norm.

\subsection{Finite intervals}

Several users asked for such a library.

\subsection{Union-find}

\subsection{Graphs}

Which representation?

\subsection{Bounded/modulo arithmetic}


\section{Libraries dedicated to tactics}

\paragraph{Inconsistencies and recommendations}

By consistency with the naming schemes in the theories directory, the
names of contribution directories should be capitalised
(e.g. \name{Coq.Omega.OmegaLemmas} instead of
\name{Coq.omega.OmegaLemmas}).

\subsection{\library{LegacyRing}}

\library{Bool}, \library{Arith} or \library{ZARith} shouldn't have
been exported when just calling \library{Ring}. But let now
\library{Ring} end its days in peace.

\subsection{\library{LegacyField}}

\library{Bool}, \library{Arith} or \library{ZARith} shouldn't have
been exported when just calling \library{Ring}. But let now
\library{Ring} end its days in peace.

\subsection{\library{setoid\_ring}}

Should find a better name than \library{setoid\_ring} for a repository that develops \tactic{Ring} and \tactic{Field} and that could more generally provide
simplification for many different kinds of algebraic structures.

\subsection{\library{Omega and ROmega}}

Make \tactic{romega} available at large scale.

\section{Miscellaneous recommendations}

\subsection{How to name files?}

In 8.1 library, 79 files adopted juxtaposed capitalized words (as in
\texttt{BinInt.v}, 
\texttt{ProofIrrelevanceFacts.v}, \texttt{NewtonInt.v}, \texttt{FSetToFiniteSet.v},
\texttt{ReflOmegaCore.v}, \texttt{LegacyZArithRing.v}, ...).

In 8.1 library, 66 files adopted separation of words by underscores.
Of them, 45 have the second word lowercase (as in
\texttt{Constructive\_sets.v}, \texttt{Setoid\_ring\_normalize.v},
\texttt{Compare\_dec.v}, \texttt{Rtrigo\_def.v},
\texttt{Field\_theory.v}, \texttt{Fourier\_util.v},
...) and 21 have the second word uppercase (as in
\texttt{Field\_Tactic.v}, \texttt{Lexicographic\_Exponentiation.v},
\texttt{Classical\_Pred\_Set.v}, \texttt{Operators\_Properties.v},
 ...)

In the first case, the authors are CP, HH, PLe, BB, OD, ... In the
second case, the authors are CP, BB, OD, PLo, CSC, DD, LT, GK, GH, ...

Would we accept to have 
\texttt{CompareDec.v}, \texttt{RTrigoDef.v},
\texttt{FieldTheory.v}, \texttt{FourierUtil.v},
\texttt{ConstructiveSets.v}, \texttt{SetoidRingNormalize.v},
\texttt{LexicographicExponentiation.v},
\texttt{ClassicalPredSet.v}, \texttt{OperatorsProperties.v},
\texttt{FieldTactic.v}, ... ?

Would we accept to have \texttt{Bin\_Int.v} or \texttt{Bin\_int.v},
\texttt{Newton\_Int.v} or \texttt{Newton\_int.v},
\texttt{Proof\_Irrelevance\_Facts.v} or
\texttt{Proof\_irrelevance\_facts.v}, \texttt{FSet\_To\_Finite\_Set.v} or
\texttt{FSet\_to\_finite\_set.v}, ...

E.g., are'nt abbreviated words better accepted if part of joined words?

\subsection{How to name directories?}

All theories have a capitalized directory name. This is not the case
of contributions (e.g. \texttt{ring}, \texttt{setoid\_ring}, ...).
This should be made uniform. At least, in the names of contributions,
option \texttt{-R} should map \texttt{omega} to \texttt{Omega}, etc.

\subsection{How to present operators and properties?}

Should we have files for operators and other for properties (as
e.g. for \texttt{Arith} or \texttt{Relations}) ? Or should have
properties follow the definitions of operators (as e.g. in
\texttt{Lists} or \texttt{Bool})?

I'm slightly in favor of having distinct files for operators and
properties.  This would allow, for instance, to have more basic
operations in the initial state without making it growing with
properties. Unless having indexes and direct access to subpart of a
file, it is often difficult to find the (short) definition of
operators when they are lost inbetween (long) list of (long) proofs of
properties.

\section{Discussion on possible features}

\subsection{Automatisation of the generation of inductive schemes}
\label{ind-schemes}

Currently, only recursion and induction schemes are automatically
generated. Why not to generalize the process to also include the
generation of decidable equality (if any), decidability of Leibniz
equality (if any), discrimination and injection combinators,
injectivity of dependent pairs (if any) (see e.g. McBride TYPES '04
paper).

\subsection{More convenient automation tactics}
\label{automation}

It is a pity that \texttt{auto} does not recognize discriminable
equalities. More generally, it is a pity that no tactic exists to
automatically recognize that such or such goal is possibly provable by
\tactic{ring}, \tactic{omega}, \tactic{discriminate}, \tactic{tauto},
\tactic{auto}...

Sometimes {\tt trivial} is not equivalent to {\tt intros; trivial}!!
Here is a short example:

\begin{verbatim}
Goal (True -> 1=0) -> (True -> 1=0).
intros H H'; trivial. (* fails *)
intro H; trivial (* succeeds *)
trivial (* succeeds *)
\end{verbatim}

\subsection{Unification modulo $\delta$}

Some constants behave as simple abbreviation and have no computational
content.  This is the case of generic notion such \name{reflexive},
\name{relation}, etc. It would be nice that \tactic{apply} (and the
automation tactics) work smoother wrt the $\delta$-conversion of this
kind of constants. Here is a (realistic) example:

\begin{verbatim}
Definition relation (A : Type) := A -> A -> Prop.
Parameter preorder: forall A : Type, relation A -> Prop.
Parameter contains: forall A B : Type, (A->B->Prop) -> (A->B->Prop) -> Prop.
Axiom contains_preorder: forall U:Type, preorder (relation U) (contains U U).
Goal forall U:Type, preorder (U -> U -> Prop) (contains U U).
intro. apply contains_preorder. (* fails !!! *)
\end{verbatim}

\subsection{Being able to say that some .v implements some module 
specification}
\label{modules}

As in Objective Caml, only explicitly defined modules can be said to
implement such or such specification. Why couldn't we do the same for
the file-level of modules.

\appendix

\chapter{An historical example: \texttt{sig} and \texttt{exist}}
\label{appendix-historical}

The names \texttt{sig} and \texttt{exist} used to denote the type and
constructor of the inductive structure of $\Sigma$-types were present
as soon as the first implementation of CoC in 1984.

Before the first release of a system based on the Calculus of
Inductive Constructions (what coincided with the move from CoC to
Coq), inductive types were encoded as second-order types using a macro
called \verb=Inductive=. Typically, here is how the inductive
type \verb=sig= of constructor \verb=exist= was defined in CoC version
4.4 (notice that the sort \texttt{Set} was called at that time
\texttt{Spec}; there was also a sort \texttt{Data} to denote the $F_\omega$ subset of \texttt{Spec}; sectioning
was also implemented \`a la \textit{Automath} using the keyword
\verb=Discharge=).

\begin{verbatim}
Variable A : Data.
Variable P : A->Prop.
    Inductive sig : Spec = exist : (x:A)(P x)->sig.
Discharge A.
Syntax sig "<_>Sig(_)".
\end{verbatim}

In CoC version 4.3, there were no macro for generating inductive
types. There were also no fully integrated extraction mechanism so
that {\tt Set}/\texttt{Spec} was not
distinguished out of \texttt{Prop} yet\footnote{As for the
predicative unbounded type hierarchy, it was introduced in CoC version
4.8.}).  Here is how the definitions of \verb=sig= and \verb=exist=
looked like.

\begin{verbatim}
 Definition sig [A:Prop][P:A->Prop](B:Prop)((x:A)(P x)->B)->B.
Syntax sig "<_>Sig(_)".

 Variable A : Prop.
 Variable P,Q : A->Prop.
 Variable x : A.

     Theorem exist      (P x)-><A>Sig(P)
       Proof    [p:(P x)][B:Prop][f:(a:A)(P a)->B](f x p).

 Discharge A.
\end{verbatim}

Before CoC version 4.3, there were no \verb=.v= files. Definitions and
lemmas were parsed directly within a ML toplevel thanks to the syntax
extension facilities of ML. Typically, here is how \verb=sig= and
\verb=exist= were defined in CoC in version 3.4 where it appeared
first (notice the use of \verb=<< >>= to switch to the vernacular
parser from within the ML toplevel).

\begin{verbatim}
<<Definition sig        [A:Prop][P:A->Prop](B:Prop)((z:A)(P z)->B)->B;>>;;


<<Variable A : Prop;>>;;
<<
Variable P : A->Prop;>>;;
<<
Variable x : A;>>;;



<<  Lemma       Sig_I   (P x)-><A>Sig(P)
  Proof exist   [p:(P x)][B:Prop][f:(a:A)(P a)->B](f x p);>>;;


<<Discharge A;>>;;
\end{verbatim}

Just for the record, notice how the simple use of indentation improved the
presentation in version 4.1:

\begin{verbatim}
<< Definition sig       [A:Prop][P:A->Prop](B:Prop)((x:A)(P x)->B)->B;>>;;

<<Variable A : Prop;>>;;
<<Variable P : A->Prop;>>;;
<<Variable x : A;>>;;

<<  Lemma       Sig_I   (P x)-><A>Sig(P)
    Proof       exist   [p:(P x)][B:Prop][f:(a:A)(P a)->B](f x p);>>;;

<< Discharge A;>>;;
\end{verbatim}

and how the need for a concrete syntax appeared in version 4.2:

\begin{verbatim}
<< Definition sig       [A:Prop][P:A->Prop](B:Prop)((x:A)(P x)->B)->B;  >>;;
LET_SYNTAX "sig" ["<";">Sig(";")"];;
\end{verbatim}

In fact, \texttt{sig} and \texttt{exist} were already there in CoC
version 1. This version was written in ML with a different parsing
facilities.

\begin{verbatim}
% Existential quantification, or general sum %
let SIGMA = "!A.{P|A->*}!B.([x:A](P x)->B)->B";;

PROP `Sig` SIGMA;;
LET_SYNTAX `Sig` [`<`;`>Sig(`;`)`];;

% Existential Introduction %
let Sig_intro = "!A.{P|A->*}[x:A](P x)-><A>Sig(P)";;

let EXIST = "!A.{P|A->*}[x:A][y:(P x)]!B.[f:[x:A](P x)->B](f x y)";;

LET `exist` EXIST Sig_intro;;
% This permits to use a predicate over A as a space over A %
\end{verbatim}

Finally, here is the definition of \texttt{sig} and \texttt{exist} in
the 1985 implementation in C (Yacc was used for
parsing, what provided better parsing facilities).

\begin{verbatim}
Definition sig  [A:Prop][P:A->Prop](B:Prop)((x:A)(P x)->B)->B; #Syntax Binary

Variable A : Prop;
Variable P : A->Prop;
Variable x : A;

  Lemma Sig_I   (P x)-><A>Sig(P)
  Proof exist   [p:(P x)][B:Prop][f:(a:A)(P a)->B](f x p);

Discharge A;
\end{verbatim}

\chapter{Miscellaneous}

\section{A list of suggested renamings}

\begin{verbatim}
(* Everywhere *)

plus, mult -> add, mult ou plus,times

(* Logic *)
refl_equal -> eq_refl
trans_eq, trans_equal -> eq_trans
sym_eq, sym_equal -> eq_sym
sym_not_eq, sym_not_equal -> not_eq_sym
f_equal -> eq_compat (??)
f_equal2 -> eq_compat2 (??)
f_equal3 ->
f_equal4 ->
f_equal5 ->

(* uniformize names for construction/destruction of connectives *)
(* 1a- Prop * Prop to Prop *)
or_introl -> inj1      or_intro_l   (contrib: 137 of which 109 in CatsInZF)
or_intror -> inj2      or_intro_r   (contrib: 150 of which 110 in CatsInZF)
proj1 ->     proj1     and_elim_l   (contrib: 634 in 10 contribs)
proj2 ->     proj2     and_elim_r   (contrib: 553 in 8 contribs)
conj ->      conj      and_intro    (contrib: 42 in 9 contribs)

(* 1b- Type * Prop to Prop *)
ex OK
ex_intro OK                      (contrib: 65 in 5 contribs - mainly CatsInZF)
ex2 OK
ex_intro2 -> ex2_intro           (contrib: 3 in 2 contribs)

(* 2- Type * Prop to Type *)
sig OK
proj1_sig -> witness (* ? *)     (contrib: 53 in 3 contribs - mainly Godel)
proj2_sig -> proof (* ? *)       (contrib: 0)
sumor (* OK? *)
inleft -> regular (* ? *)        (contrib: 62 in 9 contribs)
inright -> irregular (* ? *)     (contrib: 38 in 8 contribs)

(* 3-a Type * Type to Type *)  (* fusionner sigT et prod ? *)
fst               (contrib: 3130 in 25 contribs - 1932 in BDDS, 568 in SMC)
snd               (contrib: 2028 in 24 contribs - 1177 in BDDS, 268 in SMC)
inl               (contrib: 162 in 8 contribs)
inr               (contrib: 156 in 8 contribs)
(* 3-b Type * dep Type to Type *)
sigT -> dprod
projT1 -> dproj1  or depfst (* d for dependent *)   (contrib: 4)
projT2 -> dproj2  or depsnd     (contrib: 0)

(* 3-c Prop + Prop to Type *)
sumbool (* ? *)
left (* OK ? *)                (contrib: 419 in 48 contribs)
right (* OK ? *)               (contrib: 375 in 48 contribs)

(* Ring *)
SR_plus_zero_left -> SR_add_0_l
SR_mult_one_left  -> SR_mult_1_l
SR_mult_zero_left -> SR_mult_0_l
SR_distr_left     -> SR_mult_add_distr_l

(* ZArith *)
Z_le_lt_eq_dec -> Zle_lt_or_eq_dec_inf
Z_lt_le_dec -> Zlt_or_le_dec_inf (mais mettre en avant un Zle_or_lt_dec_inf)

dec_eq -> Zeq_dec ??
dec_Zne -> Zne_dec ??
dec_Zle -> Zle_dec
dec_Zge -> Zge_dec
dec_Zlt -> Zlt_dec
dec_Zgt -> Zgt_dec

Zle_succ = le_n_Sn : comment nommer ? (le_S ?)
Zle_le_succ = le_S : comment nommer ? (le_S_trans ?)
lt_plus_trans      : comment nommer ?

(* Arith *)
(* Peano.v *)
plus_n_O -> add_0_r
plus_O_n -> add_0_l
le_n -> le_refl
le_S -> le_S_trans

O_S -> O_S_discr ? diff_O_S ? diff_0_succ ? (� la fois propri�t�
  d'inductif -- insistance sur O, S -- et propri�t� arithm�tique --
  insistance sur 0, succ)

implicites dans iter_nat

(* Le.v *)
le_n_S -> succ_le_monotone ?? succ_le_compat ?? S_le_compat ?? bof...
le_S_n -> succ_le_reg ??
le_lt_dec -> le_or_lt_dec_inf
le_or_lt -> le_or_lt_dec
le_lt_eq_dec -> le_lt_or_eq_inf
le_lt_or_eq -> OK 
lt_le_S -> lt_le_succ ?
lt_n_O -> lt_O_minimal ??

(* Min.v *)
min_SS -> succ_min_distr
min_l -> ?
min_r -> ?
min_dec -> min_irreducible_inf (mais garder le nom intuitif min_dec comme synonyme)

(* Max.v *)
max_SS -> succ_max_distr
max_dec -> max_irreducible_inf (mais garder le nom intuitif min_dec comme synonyme)

(* Le.v *)
le_n_S -> le_succ_mono

(* Mult.v *)
mult_S_lt_compat_l -> ??

(* Peano_dec.v *)
eq_nat_dec -> eq_nat_dec_inf (trop long ?)

min_case/Zmin_case (choisir place P)

(* Bool *)
demorgan1 -> andb_orb_distrib_r
demorgan2 -> andb_orb_distrib_l
demorgan3 -> orb_andb_distrib_r
demorgan4 -> orb_andb_distrib_l
absorption_andb -> andb_orb_absorption_l_l
absorption_orb -> orb_andb_absorption_l_l
negb_elim -> negb_involutive
negb_intro -> negb_involutive_reverse
  (* zero property *)
orb_b_true -> orb_true_r
orb_true_b -> orb_true_l
andb_b_false -> andb_false_r
andb_false_b -> andb_false_l
  (* identities *)
orb_b_false -> orb_false_r
orb_false_b -> orb_false_l
andb_b_true -> andb_true_r
andb_true_b -> andb_true_l
  (* complementation *)
orb_neg_b -> orb_negb_r
andb_neg_b -> andb_negb_r

orb_true_elim -> orb_integral_inf ? (or orb_true_elim_inf)
orb_prop -> orb_integral_inf ? (or orb_true_elim)
andb_false_elim -> andb_integral_inf ? (or andb_false_elim_inf)

orb_false_elim -> ??

negb_orb -> orb_de_morgan ??
negb_andb -> andb_de_morgan ??

eqb_reflx -> insister sur la propri�t� alg�brique (soit une forme de
  nilpotence - quoique la nilpotence dit plutot a * a = 0 o� 0 est
  l'absorbant de * [ce qui donnerait eqb_nilpotent]; soit le fait que
  l'identit� est un oppos� pour eqb [ce qui donnerait eqb_id ?? ou
  eqb_id_opposite ??]) ou sur la propri�t� ``logique'' [eqb_refl]

xorb_

(* Reals *)
Rlt_le -> Rlt_le_weak (or Rlt_le_incl, Rlt_incl_le ?)

(* Logic *)
eq_dep_intro -> eq_dep_refl

Hints:

lt_n_Sm_le: remplacable par le_S_n suivi de lt_le_S (coercion identite).
gt_le_S:    convertible avec lt_le_S (i.e. c'est une identit�)
le_lt_n_Sm: convertible avec le_n_S via lt_le_S
lt_n_Sm_le: convertible avec le_S_n via lt_le_S
gt_Sn_O:    replacer par unfold gt, puis lt_le_S
le_n_Sn:    remplacable par le_S et le_n
le_S_gt:    remplacer par unfold gt, unfold lt (i.e. lt_le_S) ?

Rle_lt_0_plus_1
le_Sn_le
\end{verbatim}

\section{On the different naming schemes for the totality of order, decidability of equality, etc.}

(as of June 2009)

\begin{itemize}
\item Total order

\begin{itemize}
\item Trichotomy

\begin{verbatim}
  Lemma Rtotal_order : forall r1 r2, r1 < r2 \/ r1 = r2 \/ r1 > r2.
  Theorem Ztrichotomy : forall n m:Z, n < m \/ n = m \/ n > m.
  Theorem NZlt_trichotomy : forall n m : NZ, n < m \/ n == m \/ m < n.
  Definition lt_eq_lt_dec n m : {n < m} + {n = m} + {m < n}. (* nat *)
  Definition gt_eq_gt_dec n m : {m > n} + {n = m} + {n > m}. (* nat *)
  Axiom total_order_T : forall r1 r2:R, {r1 < r2} + {r1 = r2} + {r1 > r2}.
  Theorem Ztrichotomy_inf : forall n m:Z, {n < m} + {n = m} + {n > m}.

  Lemma Q_dec : forall x y, {x<y} + {y<x} + {x==y}.
  Lemma Qc_dec : forall x y, {x<y} + {y<x} + {x=y}.
  Lemma Z_dec : forall n m:Z, {n < m} + {n > m} + {n = m}.
  Lemma Z_dec' : forall n m:Z, {n < m} + {m < n} + {n = m}.
\end{verbatim}

\item Conditional dichotomy

\begin{verbatim}
  Theorem not_eq : forall n m : nat, n <> m -> n < m \/ m < n.
  Theorem nat_total_order : forall n m : nat, n <> m -> n < m \/ m < n.
  Theorem not_Zeq : forall n m:Z, n <> m -> n < m \/ m < n.
  Lemma Rdichotomy : forall r1 r2, r1 <> r2 -> r1 < r2 \/ r1 > r2.
  Theorem NZlt_gt_cases : forall n m : NZ, n ~= m <-> n < m \/ n > m.
  Lemma not_Zeq_inf : forall n m:Z, n <> m -> {n < m} + {m < n}.
\end{verbatim}

\item Non standard conditional dichotomy

\begin{verbatim}
  Definition le_lt_eq_dec n m : n <= m -> {n < m} + {n = m}.
  Lemma Rle_lt_or_eq_dec : forall r1 r2, r1 <= r2 -> {r1 < r2} + {r1 = r2}.
  Definition Z_le_lt_eq_dec : x <= y -> {x < y} + {x = y}.
\end{verbatim}

\item Dichotomy

\begin{verbatim}
  Lemma Zle_or_lt : forall n m:Z, n <= m \/ m < n.
  Theorem le_or_lt : forall n m, n <= m \/ m < n.
  Lemma Rlt_or_le : forall r1 r2, r1 < r2 \/ r2 <= r1.
  Lemma Rgt_or_ge : forall r1 r2, r1 > r2 \/ r2 >= r1.
  Lemma Rle_or_lt : forall r1 r2, r1 <= r2 \/ r2 < r1.
  Lemma Rge_or_gt : forall r1 r2, r1 >= r2 \/ r2 > r1.
  Theorem NZle_gt_cases : forall n m : NZ, n <= m \/ n > m.
  Theorem NZlt_ge_cases : forall n m : NZ, n < m \/ n >= m.
  Definition lt_ge_dec : forall x y, {x < y} + {x >= y}. (* nat *)
  Definition le_lt_dec n m : {n <= m} + {m < n}. (* nat *)
  Definition le_gt_dec n m : {n <= m} + {n > m}.
  Lemma Qlt_le_dec : forall x y, {x<y} + {y<=x}.
  Lemma Qclt_le_dec : forall x y, {x<y} + {y<=x}.
  Lemma Rlt_le_dec : forall r1 r2, {r1 < r2} + {r2 <= r1}.
  Lemma Rgt_ge_dec : forall r1 r2, {r1 > r2} + {r2 >= r1}.
  Lemma Rle_lt_dec : forall r1 r2, {r1 <= r2} + {r2 < r1}.
  Lemma Rge_gt_dec : forall r1 r2, {r1 >= r2} + {r2 > r1}.
  Axiom gt_le_dec : forall x y: int, {x > y} + {x <= y}.
  Axiom ge_lt_dec :  forall x y : int, {x >= y} + {x < y}.
  Definition Z_lt_ge_dec : {x < y} + {x >= y}.
  Lemma Z_lt_le_dec : {x < y} + {y <= x}.
  Definition Z_le_gt_dec : {x <= y} + {x > y}.
  Definition Z_gt_le_dec : {x > y} + {x <= y}.
  Definition Z_ge_lt_dec : {x >= y} + {x < y}.

  Corollary
    Lemma Rcase_abs : forall r, {r < 0} + {r >= 0}. 
\end{verbatim}

\item Weak dichotomy

\begin{verbatim}
  Lemma ni_le_total : forall d d':natinf, ni_le d d' \/ ni_le d' d.
  Definition le_ge_dec n m : {n <= m} + {n >= m}.
  Theorem NZle_ge_cases : forall n m : NZ, n <= m \/ n >= m.
  Lemma le_dec : forall n m, {n <= m} + {m <= n}.
\end{verbatim}

\item Dichotomy on discrete domain

\begin{verbatim}
  Definition le_le_S_dec n m : {n <= m} + {S m <= n}.
\end{verbatim}

\end{itemize}

\item Decidability of equality

\begin{verbatim}
 Lemma Req_dec : forall r1 r2, r1 = r2 \/ r1 <> r2.
 Theorem eq_nat_decide : forall n m, {eq_nat n m} + {~ eq_nat n m}.
 Theorem eq_nat_dec : forall n m, {n = m} + {n <> m}.
 Lemma bool_dec : forall b1 b2 : bool, {b1 = b2} + {b1 <> b2}.
 Lemma eq_dec (x y: positive): {x = y} + {x <> y}.
 Theorem Qeq_dec : forall x y, {x==y} + {~ x==y}.
 Theorem Qc_eq_dec : forall x y:Qc, {x=y} + {x<>y}.
 Lemma Req_EM_T : forall r1 r2:R, {r1 = r2} + {r1 <> r2}.
 Definition ascii_dec : forall a b : ascii, {a = b} + {a <> b}.
 Definition string_dec : forall s1 s2 : string, {s1 = s2} + {s1 <> s2}.
 Axiom eq_dec : forall x y : int, { x == y } + {~ x==y }.
 Definition Z_eq_dec : {x = y} + {x <> y}.
\end{verbatim}

\item Decidability of order

\begin{verbatim}
 Lemma In_dec : forall m x, { In x m } + { ~ In x m }.
 Theorem excluded_middle_informative : forall P:Prop, {P} + {~ P}.
 Lemma Rlt_dec : forall r1 r2, {r1 < r2} + {~ r1 < r2}.
 Lemma Rle_dec : forall r1 r2, {r1 <= r2} + {~ r1 <= r2}.
 Lemma Rgt_dec : forall r1 r2, {r1 > r2} + {~ r1 > r2}.
 Lemma Rge_dec : forall r1 r2, {r1 >= r2} + {~ r1 >= r2}.
 Definition Z_lt_dec : {x < y} + {~ x < y}.
 Definition Z_le_dec : {x <= y} + {~ x <= y}.
 Definition Z_gt_dec : {x > y} + {~ x > y}.
 Definition Z_ge_dec : {x >= y} + {~ x >= y}.
 Definition Zeven_dec : forall z:Z, {Zeven z} + {~ Zeven z}.
 Definition Zodd_dec : forall z:Z, {Zodd z} + {~ Zodd z}.
\end{verbatim}

\item Definition of $\leq$ in the discrete case

\begin{verbatim}
 Lemma le_le_S_eq : forall n m, n <= m -> S n <= m \/ n = m.
 Theorem gt_S : forall n m, S n > m -> n > m \/ m = n.
 Theorem NZle_succ_r : forall n m : NZ, n <= S m <-> n <= m \/ n == S m.
 Lemma Zgt_succ_gt_or_eq : forall n m:Z, Zsucc n > m -> n > m \/ m = n.
\end{verbatim}

\item Definition of $\leq$ as <= vs <

\begin{verbatim}
 Lemma Zle_lt_or_eq : forall n m:Z, n <= m -> n < m \/ n = m.
 Theorem le_lt_or_eq : forall n m, n <= m -> n < m \/ n = m.
 Lemma Qle_lt_or_eq : forall x y, x<=y -> x<y \/ x==y.
 Lemma Qcle_lt_or_eq : forall x y, x<=y -> x<y \/ x==y.
\end{verbatim}

\item Weakening of < into $<>$

\begin{verbatim}
 Lemma Rlt_dichotomy_converse : forall r1 r2, r1 < r2 \/ r1 > r2 -> r1 <> r2.
\end{verbatim}

\item Well-ordering properties of \texttt{nat}

\begin{verbatim}
 Theorem gt_O_eq : forall n, n > 0 \/ 0 = n.
 Definition zerop n : {n = 0} + {0 < n}.
\end{verbatim}

\item \texttt{odd}/\texttt{even}

\begin{verbatim}
 Lemma even_or_odd : forall n, even n \/ odd n.
 Lemma even_odd_dec : forall n, {even n} + {odd n}.
 Definition Zeven_odd_dec : forall z:Z, {Zeven z} + {Zodd z}.
\end{verbatim}

\item Lattice properties

\begin{verbatim}
 Lemma max_spec : forall n m, m <= n /\ max n m = n \/ n <= m /\ max n m = m.
 Lemma ni_min_case : forall d d':natinf, ni_min d d' = d \/ ni_min d d' = d'.
 Lemma Rmax_Rle : forall r1 r2 r, r <= Rmax r1 r2 <-> r <= r1 \/ r <= r2.
 Lemma Zmax_le_prime : forall n m p:Z, p <= Zmax n m -> p <= n \/ p <= m.
 Lemma Zmin_irreducible : forall n m:Z, Zmin n m = n \/ Zmin n m = m.
 Lemma max_dec : forall n m, {max n m = n} + {max n m = m}.
 Lemma min_dec : forall n m, {min n m = n} + {min n m = m}.
 Lemma Nmin_choice : forall a b, {Nmin a b = a} + {Nmin a b = b}.
 Lemma Zmax_irreducible_dec : forall n m:Z, {Zmax n m = n} + {Zmax n m = m}.
 Lemma Zmin_irreducible_inf : forall n m:Z, {Zmin n m = n} + {Zmin n m = m}.
 Lemma Zmin_le_prime_inf : forall n m p:Z, Zmin n m <= p -> {n <= p} + {m <= p}.
\end{verbatim}

\item Zero-product

\begin{verbatim}
 Lemma mult_is_O : forall n m, n * m = 0 -> n = 0 \/ m = 0.
 Theorem Qmult_integral : forall x y, x*y==0 -> x==0 \/ y==0.
 Theorem Qcmult_integral : forall x y, x*y=0 -> x=0 \/ y=0.
 Lemma Rmult_integral : forall r1 r2, r1 * r2 = 0 -> r1 = 0 \/ r2 = 0.
 Theorem NZeq_mul_0 : forall n m : NZ, n * m == 0 <-> n == 0 \/ m == 0.
 Theorem Zmult_integral : forall n m:Z, n * m = Z0 -> n = Z0 \/ m = Z0.
\end{verbatim}

\item Other properties of multiplication

\begin{verbatim}
 Lemma mult_O_le : forall n m, m = 0 \/ n <= m * n.
 Lemma Rmult_eq_0_compat : forall r1 r2, r1 = 0 \/ r2 = 0 -> r1 * r2 = 0.
\end{verbatim}

\item Inversion

\begin{verbatim}
 Theorem O_or_S : forall n, {m : nat | S m = n} + {0 = n}.
 Definition Ndiscr : forall n:N, { p:positive | n = Npos p } + { n = N0 }.
\end{verbatim}
\end{itemize}

\section{A list of inconsistencies}

Implicit arguments:

eq\_rect : A is implicit

eq\_rect\_r : A, x, y are implicit

\section{A collection of examples that we'd like succeed automatically}

\begin{verbatim}
Require Import Bool List.
Fixpoint andbl l := match l with nil => true | a::l => andb a (andbl l) end.
Goal forall x l, andbl (x::l) = true -> andbl l = true.
intro H; apply H.
\end{verbatim}

\begin{verbatim}
Goal forall x y, x = y => S x = S y.
\end{verbatim}

\section{Inconsistencies and weakness in the new syntax}

\texttt{set} and \texttt{pose} have semantics swapped compared to the
intuitive one.

\texttt{autorewrite using} vs \texttt{auto with}.

Use of commas...

Use of \verb=%type= in, e.g., \verb=(nat * nat)%type= is not very elegant.

The confusion between $(x_1,\ldots,x_n)$ meaning $C(x_1,\ldots,x_n)$
in destructing let and $(x_1,\ldots,x_n)$ meaning the iterated tuple
$(x_1,(x_2,\ldots,x_n)\ldots)$ in terms and \texttt{match} patterns.

How then to extend the notation to nested destructing \texttt{let} or,
further, to destructing \texttt{fun} (i.e. \texttt{fun ((x,y),(z,t))
=> \ldots} meaning \texttt{fun u => let ((x,y),(z,t)) := u in \ldots}.

Implicit arguments in Bvectors missing. Implicit arguments of
left/right, inleft/inright etc missing.

\section{About Hints}

From file {\tt rith/Power.v}, the following hints are liable to be used to simplify hypotheses (some {\tt Hint Assert} (or {\tt Hint Apply in}):

\begin{verbatim}
Hint Assert power_le_mono_inv : arith2.
Hint Assert power_lt_mono_inv1 : arith2.
Hint Assert power_lt_mono_inv2 : arith2.
\end{verbatim}

\section{Tactic wishes}

That apply and elim default to automatically removing used hypotheses
?

\subsection{auto}

{\tt auto} does not handle \texttt{iff}.

\subsection{tauto}

{\tt tauto} is not uniform in how it deals with
conjunction/disjunction when in contravariant position of an
implicative hypothesis(in which case, it must be the standard and/prod
and or/sum/sumbool/sumor) and when in an hypothesis or conclusion,
where it can be any form of conjunction/disjunction.

It also exceeds the scope of propositional logic by solving
\texttt{n=n} (due to a too tolerant notion of {\em unit} type.

\subsection{injection}

\texttt{injection} reduces more than necessary. E.g. for

\begin{verbatim}
Fixpoint div2 n : nat :=
  match n with
  | O => 0
  | S O => 0
  | S (S n') => S (div2 n')
  end.

Goal forall n, S (div2 (S n)) = S (div2 n) -> True.
intros n H. injection H.
(*
match n with
| 0 => 0
| S n' => S (div2 n')
end = div2 n -> True.
*)
\end{verbatim}



\subsection{case}

The tactic {\tt case} invents non-robust names for the branches even
for non-dependent ones. Example:

\begin{verbatim}
Goal {x : nat | x = x} -> True.
intros (?,?).
(*
  x : nat
  e : x = x
  ============================
   True
*)
\end{verbatim}

The problem comes from the {\tt prim\_refiner} which calls {\tt
  type\_case\_branches\_with\_names} which invents names depending on
the type of the variable. For instance, the names for {\tt {x:A|P x}
  are not always the same, since they will depend on the particular
  instance of {\tt P}.

\subsection{intuition}

\begin{itemize}

\item Explodes too many negations.

\item Do not clear hypotheses of the form \term{False -> A}.

\end{itemize}

\section{Language wishes}

That existential/meta variables are named after the dependent variable that
create them.

Extending the \term{\{~...~\}} notation for record types ?


\begin{verbatim}
{ x1 : A1;
  x2 : A2(x1);
  _ : T;   (* no available projection *)
  y;       (* infered type *)
  ...      (* optional colon for last field *)
}
\end{verbatim}

Having a \term{\{~...~\}} notation for record components? (allowing
field in any (dependency-compatible) order?)

\begin{verbatim}
{ x1 = O;
  x2 : A2(x1) = v1;
  _ = v2;
  ...
}
\end{verbatim}

\section{Implementation issues}

Rename \texttt{rawconstr} into \texttt{glob\_constr} or \texttt{gconstr}.

Rename term.ml into constr.ml, rawterm.ml into globconstr.ml ?

Confusion with pcoq.ml4. Rename it to ?

Replace calls with ``sigma env'' into calls with ``env sigma''.

Confusion PatVar and APatVar/CPatVar: rename the laters into AMeta/CMeta

Ltac environment: use constr or glob\_constr

Statut des hidden tactiques: autoriser ltac dedans ? Comment y
representer les arguments de tactiques

\bibliographystyle{alpha}
\bibliography{coq-biblio}

\end{document}
