
%% -*-french-tex-*-

\documentclass{article}

\usepackage{verbatim}
\usepackage[T1]{fontenc}
\usepackage[utf8]{inputenc}
\usepackage[french]{babel}
\usepackage{amsmath}
\usepackage{amssymb}
\usepackage{array}


\author{B.~Barras}
\title{Proposition de syntaxe pour Coq}

%% Le _ est un caractère normal
\catcode`\_=13
\let\subscr=_
\def_{\ifmmode\sb\else\subscr\fi}

%% Macros pour les grammaires
\def\NT#1{\langle\textit{#1}\rangle}
\def\TERM#1{\textsf{#1}}
\def\STAR#1{#1\!*}
\def\PLUS#1{#1\!+}

%% Tableaux de definition de non-terminaux
\newenvironment{cadre}
        {\begin{array}{|c|}\hline\\}
        {\\\\\hline\end{array}}
\newenvironment{rulebox}
        {$$\begin{cadre}\begin{array}{r@{~}c@{~}l@{}r}}
        {\end{array}\end{cadre}$$}
\def\DEFNT#1{\NT{#1} & ::= &}
\def\EXTNT#1{\NT{#1} & ::= & ... \\&|&}
\def\RNAME#1{(\textsc{#1})}
\def\SEPDEF{\\\\}
\def\nlsep{\\&|&}


\begin{document}

\maketitle

\section{Grammaire des tactiques}
\label{tacticsyntax}

La réflexion de la rénovation de la syntaxe des tactiques n'est pas
encore aussi poussée que pour les termes (section~\ref{constrsyntax}),
mais cette section vise à énoncer les quelques principes que l'on
souhaite suivre.

\begin{itemize}
\item Réutiliser les mots-clés de la syntaxe des termes (i.e. en
  minuscules) pour les constructions similaires de tactiques (let_in,
  match, and, etc.). Le connecteur logique \texttt{and} n'étant que
  rarement utilisé autrement que sous la forme \texttt{$\wedge$} (sauf
  dans le code ML), on pourrait dégager ce mot-clé.
\item Les arguments passés aux tactiques sont principalement des
  termes, on préconise l'utilisation d'un symbole spécial (par exemple
  l'apostrophe) pour passer une tactique ou une expression
  (AST). L'idée étant que l'on écrit plus souvent des tactiques
  prenant des termes en argument que des tacticals.
\end{itemize}

\begin{figure}
\begin{rulebox}
\DEFNT{tactic}
       \NT{tactic} ~\TERM{\&} ~\NT{tactic}            & \RNAME{then}
\nlsep \TERM{[} ~\NT{tactic}~\TERM{|}~...
      ~\TERM{|}~\NT{tactic}~\TERM{]}                  & \RNAME{par}
\nlsep \NT{ident} ~\STAR{\NT{tactic-arg}}   ~~~       & \RNAME{apply}
\nlsep \TERM{fun} ~....                               & \RNAME{function}
\nlsep \NT{simple-tactic}
\SEPDEF
\DEFNT{tactic-arg}
       \NT{constr}
\nlsep \TERM{'} ~\NT{tactic}
\SEPDEF
\DEFNT{simple-tactic}
       \TERM{Apply} ~\NT{binding-term}
\nlsep \NT{elim-kw} ~\NT{binding-term}
\nlsep \NT{elim-kw} ~\NT{binding-term} ~\TERM{using} ~\NT{binding-term}
\nlsep \TERM{Intros} ~\NT{intro-pattern}
\SEPDEF
\DEFNT{elim-kw}
       \TERM{Elim} ~\mid~ \TERM{Case} ~\mid~ \TERM{Induction}
       ~\mid~ \TERM{Destruct}
\end{rulebox}
\caption{Grammaire des tactiques}
\label{tactic}
\end{figure}


\subsection{Arguments de tactiques}

La syntaxe actuelle des arguments de tactiques est que l'on parse par
défaut une expression de tactique, ou bien l'on parse un terme si
celui-ci est préfixé par \TERM{'} (sauf dans le cas des
variables). Cela est gênant pour les utilisateurs qui doivent écrire
des \TERM{'} pour leurs tactiques.

À mon avis, il n'est pas souhaitable pour l'utilisateur de l'obliger à
marquer une différence entre les tactiques ``primitives'' (en fait
``système'') et les tactiques définies par Ltac. En effet, on se
dirige inévitablement vers une situation où il existera des librairies
de tactiques et il va devenir difficile de savoir facilement s'il faut
ou non mettre des \TERM{'}.



\subsection{Bindings}

Dans un premier temps, les ``bindings'' sont toujours considérés comme
une construction du langage des tactiques, mais il est intéressant de
prévoir l'extension de ce procédé aux termes, puisqu'il s'agit
simplement de construire un n{\oe}ud d'application dans lequel on
donne les arguments par nom ou par position, les autres restant à
inférer. Le principal point est de trouver comment combiner de manière
uniforme ce procédé avec les arguments implicites.

Il est toutefois important de réfléchir dès maintenant à une syntaxe
pour éviter de rechanger encore la syntaxe.

Intégrer la notation \TERM{with} aux termes peut poser des problèmes
puisque ce mot-clé est utilisé pour le filtrage: comment parser (en
LL(1)) l'expression:
\begin{verbatim}
Cases x with y ...
\end{verbatim}

Soit on trouve un autre mot-clé, soit on joue avec les niveaus de
priorité en obligeant a parenthéser le \TERM{with} des ``bindings'':
\begin{verbatim}
Cases (x with y) with (C z) => ...
\end{verbatim}
ce qui introduit un constructeur moralement équivalent à une
application situé à une priorité totalement différente (les
``bindings'' seraient au plus haut niveau alors que l'application est
à un niveau bas).


\begin{figure}
\begin{rulebox}
\DEFNT{binding-term}
       \NT{constr} ~\TERM{with} ~\STAR{\NT{binding}}
\SEPDEF
\DEFNT{binding}
       \NT{constr}
\end{rulebox}
\caption{Grammaire des bindings}
\label{bindings}
\end{figure}

\subsection{Enregistrements}

Il faudrait aménager la syntaxe des enregistrements dans l'optique
d'avoir des enregistrements anonymes (termes de première classe), même
si pour l'instant, on ne dispose que d'enregistrements définis a
toplevel.

Exemple de syntaxe pour les types d'enregistrements:
\begin{verbatim}
{ x1 : A1;
  x2 : A2(x1);
  _ : T;   (* Pas de projection disponible *)
  y;       (* Type infere *)
  ...      (* ; optionnel pour le dernier champ *)
}
\end{verbatim}

Exemple de syntaxe pour le constructeur:
\begin{verbatim}
{ x1 = O;
  x2 : A2(x1) = v1;
  _ = v2;
  ...
}
\end{verbatim}
Quant aux dépendences, une convention pourrait être de considérer les
champs non annotés par le type comme non dépendants.

Plusieurs interrogations:
\begin{itemize}
\item l'ordre des champs doit-il être respecté ?
  sinon, que faire pour les champs sans projection ?
\item autorise-t-on \texttt{v1} a mentionner \texttt{x1} (comme dans
  la définition d'un module), ce qui se comporterait comme si on avait
  écrit \texttt{v1} à la place. Cela pourrait être une autre manière
  de déclarer les dépendences
\end{itemize}

La notation pointée pour les projections pose un problème de parsing,
sauf si l'on a une convention lexicale qui discrimine les noms de
modules des projections et identificateurs: \texttt{x.y.z} peut être
compris comme \texttt{(x.y).z} ou texttt{x.(y.z)}.


\section{Grammaire des termes}
\label{constrsyntax}

\subsection{Quelques principes}

\begin{enumerate}
\item Diminuer le nombre de niveaux de priorité en regroupant les
  règles qui se ressemblent: infixes, préfixes, lieurs (constructions
  ouvertes à droite), etc.
\item Éviter de surcharger la signification d'un symbole (ex:
  \verb+( )+ comme parenthésage et produit dans la V7).
\item Faire en sorte que les membres gauches (motifs de Cases, lieurs
  d'abstraction ou de produits) utilisent une syntaxe compatible avec
  celle des membres droits (branches de Cases et corps de fonction).
\end{enumerate}

\subsection{Présentation de la grammaire}

\begin{figure}
\begin{rulebox}
\DEFNT{paren-constr}
       \NT{cast-constr}~\TERM{,}~\NT{paren-constr}     &\RNAME{pair}
\nlsep \NT{cast-constr}
\SEPDEF
\DEFNT{cast-constr}
       \NT{constr}~\TERM{\!\!:}~\NT{cast-constr}       &\RNAME{cast}
\nlsep \NT{constr}
\SEPDEF
\DEFNT{constr}
       \NT{appl-constr}~\NT{infix}~\NT{constr}      &\RNAME{infix}
\nlsep \NT{prefix}~\NT{constr}                      &\RNAME{prefix}
\nlsep \NT{constr}~\NT{postfix}                     &\RNAME{postfix}
\nlsep \NT{appl-constr}
\SEPDEF
\DEFNT{appl-constr}
       \NT{appl-constr}~\PLUS{\NT{appl-arg}}           &\RNAME{apply}
\nlsep \TERM{@}~\NT{global}~\PLUS{\NT{simple-constr}}  &\RNAME{expl-apply}
\nlsep \NT{simple-constr}
\SEPDEF
\DEFNT{appl-arg}
       \TERM{@}~\NT{int}~\TERM{\!:=}~\NT{simple-constr}  &\RNAME{impl-arg}
\nlsep \NT{simple-constr}
\SEPDEF
\DEFNT{simple-constr}
       \NT{atomic-constr}
\nlsep \TERM{(}~\NT{paren-constr}~\TERM{)}
\nlsep \NT{match-constr}
\nlsep \NT{fix-constr}
%% \nlsep \TERM{<\!\!:ast\!\!:<}~\NT{ast}~\TERM{>\!>} &\RNAME{quotation}
\end{rulebox}
\caption{Grammaire des termes}
\label{constr}
\end{figure}

\begin{figure}
\begin{rulebox}
\DEFNT{prefix}
       \TERM{!}~\PLUS{\NT{binder}}~\TERM{.}~    &\RNAME{prod}
\nlsep \TERM{fun} ~\PLUS{\NT{binder}} ~\TERM{$\Rightarrow$} &\RNAME{lambda}
\nlsep \TERM{let}~\NT{ident}~\STAR{\NT{binder}} ~\TERM{=}~\NT{constr}
        ~\TERM{in}  &\RNAME{let}
%\nlsep \TERM{let (}~\NT{comma-ident-list}~\TERM{) =}~\NT{constr}
%        ~\TERM{in}   &~~~\RNAME{let-case}
\nlsep \TERM{if}~\NT{constr}~\TERM{then}~\NT{constr}~\TERM{else}
      &\RNAME{if-case}
\nlsep \TERM{eval}~\NT{red-fun}~\TERM{in}       &\RNAME{eval}
\SEPDEF
\DEFNT{infix}
      \TERM{$\rightarrow$}                     & \RNAME{impl}
\SEPDEF
\DEFNT{atomic-constr}
      \TERM{_}
\nlsep \TERM{?}\NT{int}
\nlsep \NT{sort}
\nlsep \NT{global}
\SEPDEF
\DEFNT{binder}
       \NT{ident}                              &\RNAME{infer}
\nlsep \TERM{(}~\NT{ident}~\NT{type}~\TERM{)}  &\RNAME{binder}
\SEPDEF
\DEFNT{type}
       \TERM{\!:}~\NT{constr}
\nlsep \epsilon
\end{rulebox}
\caption{Grammaires annexes aux termes}
\label{gram-annexes}
\end{figure}

La grammaire des termes (correspondant à l'état \texttt{barestate})
est décrite figures~\ref{constr} et~\ref{gram-annexes}. On constate
par rapport aux précédentes versions de Coq d'importants changements
de priorité, le plus marquant étant celui de l'application qui se
trouve désormais juste au dessus\footnote{La convention est de
considérer les opérateurs moins lieurs comme ``au dessus'',
c'est-à-dire ayant un niveau de priorité plus élévé (comme c'est le
cas avec le niveau de la grammaire actuelle des termes).} des
constructions fermées à gauche et à droite.

La grammaire des noms globaux est la suivante:
\begin{eqnarray*}
\DEFNT{global}
      \NT{ident}
%% \nlsep \TERM{\$}\NT{ident}
\nlsep \NT{ident}\TERM{.}\NT{global}
\end{eqnarray*}

Le $\TERM{_}$ dénote les termes à synthétiser. Les métavariables sont
reconnues au niveau du lexer pour ne pas entrer en conflit avec le
$\TERM{?}$ de l'existentielle.

Les opérateurs infixes ou préfixes sont tous au même niveau de
priorité du point de vue de Camlp4. La solution envisagée est de les
gérer à la manière de Yacc, avec une pile (voir discussions plus
bas). Ainsi, l'implication est un infixe normal; la quantification
universelle et le let sont vus comme des opérateurs préfixes avec un
niveau de priorité plus haut (i.e. moins lieur). Il subsiste des
problèmes si l'on ne veut pas écrire de parenthèses dans:
\begin{verbatim}
 A -> (!x. B -> (let y = C in D))
\end{verbatim}

La solution proposée est d'analyser le membre droit d'un infixe de
manière à autoriser les préfixes et les infixes de niveau inférieur,
et d'exiger le parenthésage que pour les infixes de niveau supérieurs.

En revanche, à l'affichage, certains membres droits seront plus
lisibles s'ils n'utilisent pas cette astuce:
\begin{verbatim}
(fun x => x) = fun x => x
\end{verbatim}

La proposition est d'autoriser ce type d'écritures au parsing, mais
l'afficheur écrit de manière standardisée en mettant quelques
parenthèses superflues: $\TERM{=}$ serait symétrique alors que
$\rightarrow$ appellerait l'afficheur de priorité élevée pour son
sous-terme droit.

Les priorités des opérateurs primitifs sont les suivantes (le signe
$*$ signifie que pour le membre droit les opérateurs préfixes seront
affichés sans parenthèses quel que soit leur priorité):
$$
\begin{array}{c|l}
$symbole$ & $priorité$ \\
\hline
\TERM{!}      & 200\,R* \\
\TERM{fun}    & 200\,R* \\
\TERM{let}    & 200\,R* \\
\TERM{if}     & 200\,R \\
\TERM{eval}   & 200\,R \\
\rightarrow   & 90\,R*
\end{array}
$$

Il y a deux points d'entrée pour les termes: $\NT{constr}$ et
$\NT{simple-constr}$. Le premier peut être utilisé lorsqu'il est suivi
d'un séparateur particulier. Dans le cas où l'on veut une liste de
termes séparés par un espace, il faut lire des $\NT{simple-constr}$.



Les constructions $\TERM{fix}$ et $\TERM{cofix}$ (voir aussi
figure~\ref{gram-fix}) sont fermées par end pour simplifier
l'analyse. Sinon, une expression de point fixe peut être suivie par un
\TERM{in} ou un \TERM{and}, ce qui pose les mêmes problèmes que le
``dangling else'': dans
\begin{verbatim}
fix f1 x {x} = fix f2 y {y} = ... and ... in ...
\end{verbatim}
il faut définir une stratégie pour associer le \TERM{and} et le
\TERM{in} au bon point fixe.

Un autre avantage est de faire apparaitre que le \TERM{fix} est un
constructeur de terme de première classe et pas un lieur:
\begin{verbatim}
fix f1 ... and f2 ...
in f1 end x
\end{verbatim}
Les propositions précédentes laissaient \texttt{f1} et \texttt{x}
accolés, ce qui est source de confusion lorsque l'on fait par exemple
\texttt{Pattern (f1 x)}.

Les corps de points fixes et co-points fixes sont identiques, bien que
ces derniers n'aient pas d'information de décroissance. Cela
fonctionne puisque l'annotation est optionnelle. Cela préfigure des
cas où l'on arrive à inférer quel est l'argument qui décroit
structurellement (en particulier dans le cas où il n'y a qu'un seul
argument).

\begin{figure}
\begin{rulebox}
\DEFNT{fix-expr}
       \TERM{fix}~\NT{fix-decls}    ~\NT{fix-select} ~\TERM{end} &\RNAME{fix}
\nlsep \TERM{cofix}~\NT{cofix-decls}~\NT{fix-select} ~\TERM{end} &\RNAME{cofix}
\SEPDEF
\DEFNT{fix-decls}
       \NT{fix-decl}~\TERM{and}~\NT{fix-decls}
\nlsep \NT{fix-decl}
\SEPDEF
\DEFNT{fix-decl}
       \NT{ident}~\PLUS{\NT{binder}}~\NT{type}~\NT{annot}
       ~\TERM{=}~\NT{constr}
\SEPDEF
\DEFNT{annot}
       \TERM{\{}~\NT{ident}~\TERM{\}}
\nlsep \epsilon
\SEPDEF
\DEFNT{fix-select}
       \TERM{in}~\NT{ident}
\nlsep \epsilon
\end{rulebox}
\caption{Grammaires annexes des points fixes}
\label{gram-fix}
\end{figure}

La construction $\TERM{Case}$ peut-être considérée comme
obsolète. Quant au $\TERM{Match}$ de la V6, il disparaît purement et
simplement.

\begin{figure}
\begin{rulebox}
\DEFNT{match-expr}
       \TERM{match}~\NT{case-items}~\NT{case-type}~\TERM{with}~
       \NT{branches}~\TERM{end}        &\RNAME{match}
\nlsep \TERM{match}~\NT{case-items}~\TERM{with}~
       \NT{branches}~\TERM{end}        &\RNAME{infer-match}
%%\nlsep \TERM{case}~\NT{constr}~\NT{case-predicate}~\TERM{of}~
%%      \STAR{\NT{constr}}~\TERM{end}   &\RNAME{case}
\SEPDEF
\DEFNT{case-items}
       \NT{case-item} ~\TERM{\&} ~\NT{case-items}
\nlsep \NT{case-item}
\SEPDEF
\DEFNT{case-item}
       \NT{constr}~\NT{pred-pattern} &\RNAME{dep-case}
\nlsep \NT{constr}                   &\RNAME{nodep-case}
\SEPDEF
\DEFNT{case-type}
       \TERM{$\Rightarrow$}~\NT{constr}
\nlsep \epsilon
\SEPDEF
\DEFNT{pred-pattern}
       \TERM{as}~\NT{ident} ~\TERM{\!:}~\NT{constr}
\SEPDEF
\DEFNT{branches}
       \TERM{|} ~\NT{patterns} ~\TERM{$\Rightarrow$}
        ~\NT{constr} ~\NT{branches}
\nlsep \epsilon
\SEPDEF
\DEFNT{patterns}
       \NT{pattern} ~\TERM{\&} ~\NT{patterns}
\nlsep \NT{pattern}
\SEPDEF
\DEFNT{pattern} ...
\end{rulebox}
\caption{Grammaires annexes du filtrage}
\label{gram-match}
\end{figure}

De manière globale, l'introduction de définitions dans les termes se
fait avec le symbole $=$, et le $\!:=$ est réservé aux définitions au
niveau vernac. Il y avait un manque de cohérence dans la
V6, puisque l'on utilisait $=$ pour le $\TERM{let}$ et $\!:=$ pour les
points fixes et les commandes vernac.

% OBSOLETE: lieurs multiples supprimes
%On peut remarquer que $\NT{binder}$ est un sous-ensemble de
%$\NT{simple-constr}$, à l'exception de $\texttt{(a,b\!\!:T)}$: en tant
%que lieur, {\tt a} et {\tt b} sont tous deux contraints, alors qu'en
%tant que terme, seul {\tt b} l'est. Cela qui signifie que l'objectif
%de rendre compatibles les membres gauches et droits est {\it presque}
%atteint.

\subsection{Infixes}

\subsubsection{Infixes extensibles}

Le problème de savoir si la liste des symboles pouvant apparaître en
infixe est fixée ou extensible par l'utilisateur reste à voir.

Notons que la solution où les symboles infixes sont des
identificateurs que l'on peut définir paraît difficilement praticable:
par exemple $\texttt{Logic.eq}$ n'est pas un opérateur binaire, mais
ternaire. Il semble plus simple de garder des déclarations infixes qui
relient un symbole infixe à un terme avec deux ``trous''. Par exemple:

$$\begin{array}{c|l}
$infixe$ &  $identificateur$ \\
\hline
= & \texttt{Logic.eq _ ?1 ?2} \\
== & \texttt{JohnMajor.eq _ ?1 _ ?2}
\end{array}$$

La syntaxe d'une déclaration d'infixe serait par exemple:
\begin{verbatim}
Infix "=" 50 := Logic.eq _ ?1 ?2;
\end{verbatim}


\subsubsection{Gestion des précédences}

Les infixes peuvent être soit laissé à Camlp4, ou bien (comme ici)
considérer que tous les opérateurs ont la même précédence et gérer
soit même la recomposition des termes à l'aide d'une pile (comme
Yacc).


\subsection{Extensions de syntaxe}

\subsubsection{Litéraux numériques}

La proposition est de considerer les litéraux numériques comme de
simples identificateurs. Comme il en existe une infinité, il faut un
nouveau mécanisme pour leur associer une définition. Par exemple, en
ce qui concerne \texttt{Arith}, la définition de $5$ serait
$\texttt{S}~4$. Pour \texttt{ZArith}, $5$ serait $\texttt{xI}~2$.

Comme les infixes, les constantes numériques peuvent être qualifiées
pour indiquer dans quels module est le type que l'on veut
référencer. Par exemple (si on renomme \texttt{Arith} en \texttt{N} et
\texttt{ZArith} en \texttt{Z}): \verb+N.5+, \verb+Z.5+.

\begin{eqnarray*}
\EXTNT{global}
      \NT{int}
\end{eqnarray*}

\subsubsection{Nouveaux lieurs}

$$
\begin{array}{rclr}
\EXTNT{constr}
       \TERM{ex}~\PLUS{\NT{binder}}~\TERM{.}~\NT{constr}  &\RNAME{ex}
\nlsep \TERM{ex}~\PLUS{\NT{binder}}~\TERM{.}~\NT{constr}~\TERM{,}~\NT{constr}
        &\RNAME{ex2}
\nlsep \TERM{ext}~\PLUS{\NT{binder}}~\TERM{.}~\NT{constr}  &\RNAME{exT}
\nlsep \TERM{ext}~\PLUS{\NT{binder}}~\TERM{.}~\NT{constr}~\TERM{,}~\NT{constr}
        &\RNAME{exT2}
\end{array}
$$

Pour l'instant l'existentielle n'admet qu'une seule variable, ce qui
oblige à écrire des cascades de $\TERM{ex}$.

Pour parser les existentielles avec deux prédicats, on peut considérer
\TERM{\&} comme un infixe intermédiaire et l'opérateur existentiel en
présence de cet infixe se transforme en \texttt{ex2}.

\subsubsection{Nouveaux infixes}

Précédences des opérateurs infixes (les plus grands associent moins fort):
$$
\begin{array}{l|l|c|l}
$identificateur$      & $module$     & $infixe/préfixe$ & $précédence$ \\
\hline
\texttt{iff}          & $Logic$      & \longleftrightarrow & 100 \\
\texttt{or}           & $Logic$      & \vee     & 80\, R \\
\texttt{sum}          & $Datatypes$  & +        & 80\, R \\
\texttt{and}          & $Logic$      & \wedge   & 70\, R \\
\texttt{prod}         & $Datatypes$  & *        & 70\, R \\
\texttt{not}          & $Logic$      & \tilde{} & 60\, L \\
\texttt{eq _}         & $Logic$      & =        & 50 \\
\texttt{eqT _}        & $Logic_Type$ & =        & 50 \\
\texttt{identityT _}  & $Data_Type$  & =        & 50 \\
\texttt{le}           & $Peano$      & $<=$     & 50 \\
\texttt{lt}           & $Peano$      & $<$      & 50 \\
\texttt{ge}           & $Peano$      & $>=$     & 50 \\
\texttt{gt}           & $Peano$      & $>$      & 50 \\
\texttt{Zle}          & $zarith_aux$   & $<=$   & 50 \\
\texttt{Zlt}          & $zarith_aux$   & $<$    & 50 \\
\texttt{Zge}          & $zarith_aux$   & $>=$   & 50 \\
\texttt{Zgt}          & $zarith_aux$   & $>$    & 50 \\
\texttt{Rle}          & $Rdefinitions$ & $<=$   & 50 \\
\texttt{Rlt}          & $Rdefinitions$ & $<$    & 50 \\
\texttt{Rge}          & $Rdefinitions$ & $>=$   & 50 \\
\texttt{Rgt}          & $Rdefinitions$ & $>$    & 50 \\
\texttt{plus}         & $Peano$        & +      & 40\,L \\
\texttt{Zplus}        & $fast_integer$ & +      & 40\,L \\
\texttt{Rplus}        & $Rdefinitions$ & +      & 40\,L \\
\texttt{minus}        & $Minus$        & -      & 40\,L \\
\texttt{Zminus}       & $zarith_aux$   & -      & 40\,L \\
\texttt{Rminus}       & $Rdefinitions$ & -      & 40\,L \\
\texttt{Zopp}         & $fast_integer$ & -      & 40\,L \\
\texttt{Ropp}         & $Rdefinitions$ & -      & 40\,L \\
\texttt{mult}         & $Peano$        & *      & 30\,L \\
\texttt{Zmult}        & $fast_integer$ & *      & 30\,L \\
\texttt{Rmult}        & $Rdefinitions$ & *      & 30\,L \\
\texttt{Rdiv}         & $Rdefinitions$ & /      & 30\,L \\
\texttt{pow}          & $Rfunctions$   & \hat   & 20\,L \\
\texttt{fact}         & $Rfunctions$   & !      & 20\,L \\
\end{array}
$$

Notons qu'il faudrait découper {\tt Logic_Type} en deux car celui-ci
définit deux égalités, ou alors les mettre dans des modules différents.

\subsection{Exemples}

\begin{verbatim}
Definition not (A:Prop) := A->False;
Inductive eq (A:Set) (x:A) : A->Prop :=
    refl_equal : eq A x x;
Inductive ex (A:Set) (P:A->Prop) : Prop :=
    ex_intro : !x. P x -> ex A P;
Lemma not_all_ex_not : !(P:U->Prop). ~(!n. P n) -> ?n. ~ P n;
Fixpoint plus n m : nat {struct n} :=
  match n with
    O     => m
  | (S k) => S (plus k m)
  end;
\end{verbatim}

\subsection{Questions ouvertes}

Voici les points sur lesquels la discussion est particulièrement
ouverte:
\begin{itemize}
\item choix d'autres symboles pour les quantificateurs \TERM{!} et
  \TERM{?}. En l'état actuel des discussions, on garderait le \TERM{!}
  pour la qunatification universelle, mais on choisirait quelquechose
  comme \TERM{ex} pour l'existentielle, afin de ne pas suggérer trop
  de symétrie entre ces quantificateurs (l'un est primitif, l'autre
  pas).
\item syntaxe particulière pour les \texttt{sig}, \texttt{sumor}, etc.
\item la possibilité d'introduire plusieurs variables du même type est
  pour l'instant supprimée au vu des problèmes de compatibilité de
  syntaxe entre les membres gauches et membres droits. L'idée étant
  que l'inference de type permet d'éviter le besoin de déclarer tous
  les types.
\end{itemize}

\subsection{Autres extensions}

\subsubsection{Lieur multiple}

L'écriture de types en présence de polymorphisme est souvent assez
pénible:
\begin{verbatim}
Check !(A:Set) (x:A) (B:Set) (y:B). P A x B y;
\end{verbatim}

On pourrait avoir des déclarations introduisant à la fois un type
d'une certaine sorte et une variable de ce type:
\begin{verbatim}
Check !(x:A:Set) (y:B:Set). P A x B y;
\end{verbatim}

Noter que l'on aurait pu écrire:
\begin{verbatim}
Check !A x B y. P A (x:A:Set) B (y:B:Set);
\end{verbatim}

\section{Syntaxe des tactiques}

\subsection{Questions diverses}

Changer ``Pattern nl c ... nl c'' en ``Pattern [ nl ] c ... [ nl ] c''
pour permettre des chiffres seuls dans la catégorie syntaxique des
termes.

Par uniformité remplacer ``Unfold nl c'' par ``Unfold [ nl ] c'' ?

Même problème pour l'entier de Specialize (ou virer Specialize ?) ?

\subsection{Questions en suspens}

\verb=EAuto= : deux syntaxes différentes pour la recherche en largeur
et en profondeur ? Quelle recherche par défaut ?

\section*{Remarques pêle-mêle (HH)}

Autoriser la syntaxe

\begin{verbatim}
Variable R (a : A) (b : B) : Prop.
Hypotheses H (a : A) (b : B) : Prop; Y (u : U) : V.
Variables H (a : A) (b : B), J (k : K) : nat; Z (v : V) : Set.
\end{verbatim}

Renommer eqT, refl_eqT, eqT_ind, eqT_rect, eqT_rec en eq, refl_equal, etc.
Remplacer == en =.

Mettre des \verb=?x= plutot que des \verb=?1= dans les motifs de ltac ??

\section{Moulinette}

\begin{itemize}

\item Mettre \verb=/= et * au même niveau dans R.

\item Changer la précédence du - unaire dans R.

\item Ajouter Require Arith par necessite si Require ArithRing ou Require ZArithRing.

\item Ajouter Require ZArith par necessite si Require ZArithRing ou Require Omega.

\item Enlever le Export de Bool, Arith et ZARith de Ring quand inapproprié et
l'ajouter à côté des Require Ring.

\item Remplacer "Check n" par "n:Check ..."

\item Renommer Variable/Hypothesis hors section en Parameter/Axiom.

\item Renommer les \verb=command0=, \verb=command1=, ... \verb=lcommand= etc en
\verb=constr0=, \verb=constr1=, ... \verb=lconstr=.

\item Remplacer les noms Coq.omega.Omega par Coq.Omega ...

\item Remplacer AddPath par Add LoadPath (ou + court)

\item Unify + and \{\}+\{\} and +\{\} using Prop $\leq$ Set ??

\item Remplacer Implicit Arguments On/Off par Set/Unset Implicit Arguments.

\item La syntaxe \verb=Intros (a,b)= est inutile, \verb=Intros [a b]= fait l'affaire.

\item Virer \verb=Goal= sans argument (synonyme de \verb=Proof= et sans effets).

\item Remplacer Save. par Qed.

\item Remplacer \verb=Zmult_Zplus_distr= par \verb=Zmult_plus_distr_r=
et \verb=Zmult_plus_distr= par \verb=Zmult_plus_distr_l=.

\end{itemize}

\end{document}
